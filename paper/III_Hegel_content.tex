\phantomsection
\subsection*{III—Hegel: From Object-Desire to Recognition, Mastery's Emptiness, and the Truth of
	Work}
\addcontentsline{toc}{subsection}{Hegel: Recognition and Mastery's Emptiness}
\label{ssec:iii-hegel}
G.W.F. Hegel's decisive move is to show why fulfillment through possession cannot settle
desire. In the \emph{Phenomenology of Spirit}, self-consciousness first appears as desire that
negates
otherness, but it learns that consuming things can never yield self-certainty:
``self-consciousness achieves its satisfaction only in another self-consciousness''
\parencite[\S 175]{HegelPhenomenology1977}. The thing I devour does not look back; it cannot
recognize me. If ``Everyone gets everything he wants'' is read as a promise of objects and
outcomes, Hegel's rejoinder is that objects are the wrong currency for the desire at stake.
The later clause—``\ldots for my sins I got one''—sounds like the experience of having obtained
the wrong coin.

Hegel dramatizes this transition in the struggle for recognition culminating in lordship and
bondage \parencite[\S\S 178--196]{HegelPhenomenology1977}. The combatants risk death because
only a being who risks its life shows that it is not bound to bare preservation. The so-called
Lord ``wins,'' but his victory is hollow: the Bondsman's submission is not free recognition
\parencite[\S\S 187--189]{HegelPhenomenology1977}. Mastery therefore ``gets what it
wants''—dominion—and
finds it empty of the very confirmation it sought. This emptiness is not psychological
disappointment; it is structural. Recognition that counts must be mutual between free subjects.
Where a project's logic—administrative or militarized—reduces others to functions, the more
perfectly it attains its end, the more sharply its lack of recognition appears.

The truth of self-consciousness, Hegel says, lies not with the Lord but with the Bondsman, who,
through fear, service, and formative work (\emph{Bildung}), mediates self and world
\parencite[\S 196]{HegelPhenomenology1977}. Work transforms the given without annihilating it;
it commits the self to a shared, durable world. The river journey's serial procedures—secure a
beach, clear a waterway, enforce a protocol—have the outer form of work, yet the world they
leave is not stabilized as a space of mutual recognition. The cycle at the bridge—construction
by day, erasure by night—parodies \emph{Bildung}: it produces, but it does not found.
Fulfillment here punishes by revealing the absence of the only recognition that could have
satisfied the desire that set the project in motion.

Hegel's dialectic also clarifies why the most ``efficient'' victories feel airless. Dominating
the other as instrument silences the very freedom from which recognition must come. Each success,
then, intensifies the contradiction: the more complete the procedure, the more total the other's
silencing, and the less possible the confirmation the agent craves. ``Everyone gets everything
he wants'' becomes tragic because the want was mis-specified: it sought certitude about self
through the mute success of operations. ``\ldots For my sins I got one'' is the moment mastery
confesses its own null confirmation.
