\phantomsection
\section*{VII. Conclusion: The Mission and the Mirror}
\addcontentsline{toc}{section}{VII. Conclusion}
\label{sec:vii-conclusion}

Captain Willard's line---``Everyone gets everything he wants. I wanted a mission, and for my 
sins they gave me one''---has proven interpretable across radically different frameworks. 
Conrad's literary structure (Section II) provides the journey's template; theological and 
Buddhist traditions (Section III) read fulfillment as moral or karmic disclosure; twelve 
Western philosophers (Section IV) diagnose the will's metaphysical, existential, and ethical 
structures; critical theorists (Section V) historicize those structures within colonial 
bureaucracy and representational violence; psychologists (Section VI) reveal desire as 
death-denial and fantasy-support. Each reading exposes a dimension of why getting what one 
wants reveals what the wanting was.

Yet the traditions converge on a single claim: the problem is not that desire is denied but 
that it is granted. Punishment comes not from deprivation but from delivery. This is the 
film's thesis, enacted through Willard's journey: each fulfilled objective (the beach secured, 
the protocol followed, the mission completed) strips another alibi until only the will's 
complicity remains visible. The horror is not what happens but what wanting it reveals about 
the will that wanted.

None of these traditions counsel resignation. Each proposes a discipline of wanting: Augustine's 
rightly ordered love, the Buddha's Noble Path, Kant's categorical imperative, Beauvoir's 
reciprocity, Levinas's responsibility for the face, Arendt's plural action, Frankl's search 
for meaning beyond the self. The common thread is reformation: not the annihilation of desire 
but its reorientation away from possession and toward responsibility, away from fantasy and 
toward lucidity.

The film offers no such reformation. Willard completes the mission and inherits Kurtz's place, 
but nothing changes. The will remains trapped in the structure it exposed. Perhaps that is the 
final teaching: interpretation alone does not redeem. Understanding why fulfillment punishes 
does not break the pattern. The question the essay cannot answer---because the film does 
not---is whether Willard, having seen what his wanting was, can want differently. The mission 
is the mirror. What the will does with the reflection remains open.
