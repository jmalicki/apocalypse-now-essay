\phantomsection
\section*{VII. Conclusion: The Mission and the Mirror}
\addcontentsline{toc}{section}{VII. Conclusion: The Mission and the Mirror}
\label{sec:vii-conclusion-the-mission-and-the-mirror}

Across theology (Section III), philosophy (Section IV), critical theory (Section V), and depth 
psychology (Section VI), a single structure recurs: fulfillment is disclosure. To ``get what 
one wants'' is to discover what one's wanting already was. In biblical terms, the sinner is 
given over to his desire; in Buddhist terms, craving reproduces suffering; in Schopenhauer's 
terms, satisfaction resets the pendulum; in Kant's terms, success proves nothing about duty; 
in Sartre's terms, the project exposes bad faith; in Foucault's terms, the will reveals its 
normalization; in Lacan's terms, demand is met but desire persists; in Žižek's terms, cynical 
knowledge does not break the fantasy. Each tradition diagnoses why ``for my sins I got one'' 
follows from ``everyone gets everything he wants.''

\subsection*{Toward a Discipline of Wanting}
\label{ssec:3-toward-a-discipline-of-wanting}
The traditions surveyed here do not counsel quietism but discipline. The biblical path names it charity and obedience; the Buddhist path, the Noble Eightfold Path; Kant, duty; Beauvoir, reciprocity; Levinas, responsibility for the face of the Other; Fanon, the struggle for recognition without annihilation. Each proposes that the solution to fulfilled desire is not more desire, nor its annihilation, but the \emph{re-formation} of willing: detachment from possession, attachment to responsibility. The altern...
\subsection*{4. Last Words}
\label{ssec:4-last-words}
Poetry often says last things best. Eliot’s reflection that ``the end of all our exploring / Will be to arrive where we started / And know the place for the first time'' \gbparencite[Eliot, 1969, ``Little Gidding'']{EliotCollected1969} names the arc of this essay: the mission returns us to the origin of will. Rilke’s admonition---``You must change your life''---arrives in another register \parencite[Snow trans., 2009]{RilkeElegies2009}. If Willard’s sentence is the mission’s epitaph, this is its coda. The...
