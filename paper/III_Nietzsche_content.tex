\subsection*{III—Nietzsche: Transvaluation versus the Mask of “Truth”}
\label{ssec:iii-nietzsche}
Nietzsche contests Schopenhauer’s resignation while preserving recurrence: what corrodes willing is not recurrence itself but a \emph{valuation} that reduces willing to consumption. The task is transvaluation—recasting recurrence as creative affirmation. His warning that the “desire for ‘truth’ ” can operate as a disguised will to command (\parencite[\S34]{NietzscheBGE1990}) is dramatized by the Saigon dossier’s sanitary rhetoric: cognition as a mask for domination. In that light, “everyone gets everything he wants” reads as the triumph of a will to order; because no transvaluation occurs upriver, recurrence returns as nausea, not joy. “…for my sins I got one” marks the recognition that correct outcomes without revalued ends deepen rather than cure the malaise.
