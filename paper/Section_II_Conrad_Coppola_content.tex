\phantomsection
\section*{II. Conrad and Coppola: The Journey Upriver as Structure}
\addcontentsline{toc}{section}{II. Conrad and Coppola}
\label{sec:ii-conrad-and-coppola}

\textit{Apocalypse Now} refracts Joseph Conrad's \textit{Heart of Darkness} through Cold War
geopolitics and the American war in Vietnam, translating the desire for a ``mission'' into the
bureaucratized pursuit of domination. In both narratives, the will gets what it
wants---empire, recognition, power---and what it wants unmasks the will. The horror is not
merely the violence of conquest but the disclosure that conquest was the will's secret object
all along.

\subsection*{1. Conrad’s Modernity: Illumination and Horror}
\label{ssec:1-conrad-s-modernity-illumination-and-horror}
Conrad’s novella organizes colonial conquest as an epistemological allegory: the voyage upriver promises knowledge, but the attainment of knowledge reveals only the vacuum at its core. In Marlow’s opening demystification of imperial rhetoric, conquest is ``robbery with violence, aggravated murder on a great scale'' \parencite[ConradHOD1990]{ConradHOD1990}. Kurtz’s dying judgment---``The horror! The horror!''---is the paradoxical consummation of his civilizing project: he gets everything he wants (ivory, absolute command, the image of European virtue) and discovers that desire fulfilled negates the desiring self \parencite{ConradHOD1990}. The novella’s formal strategy---a frame narrative in which the tale loops back on itself---mirrors this structure of return: fulfillment is not progress but recursion.

\subsection*{2. From Conrad to Coppola: Bureaucracy as a Technology of Will}
\label{ssec:2-from-conrad-to-coppola-bureaucracy-as-a-technology-of-will}
Coppola’s adaptation transposes Conrad’s private empire into a military bureaucracy that routinizes transgression. Willard’s orders are typed, briefed, and accompanied by dossiers; Kurtz’s poetry is replaced by radio logs and classified memoranda. The mission is thus an artifact of files, not of metaphysical vocation. In Willard’s terms, ``for my sins they gave me one'': the administrative system internalizes the will’s desire for a task and returns it as obligation \parencite{CoppolaApocalypse1979}. The film’s mise-en-scène—the air cavalry’s Wagnerian assault, the Playboy USO show, the bridge to nowhere—presents modern fulfillment as spectacle: an economy of images where desire circulates as command and entertainment at once.

Adorno and Horkheimer called this dialectic in advance: the Enlightenment’s will to demystify nature turns into domination of humans by technical reason \parencite{AdornoHorkheimer2002}. The same rationality that frees us from myth installs the world as object of control. In \textit{Apocalypse Now}, the mission is in this sense an \emph{Enlightenment object}: planned, staged, justified. Its fulfillment exposes the subject as functionary. Arendt’s analysis of modern totalitarianism clarifies the disjunction between action and responsibility here: systems generate outcomes that no single agent intends, but which implicate every actor caught within them \parencite{ArendtOrigins1973}. Willard’s errand-boy status embodies this structure of dispersed agency.
