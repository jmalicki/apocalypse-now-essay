\phantomsection
\section*{Appendix B: Film Scene Reference Guide}
\addcontentsline{toc}{section}{Appendix B: Film Scene Reference}
\label{app:scene-reference}

The following scenes from \textit{Apocalypse Now: Final Cut} \parencite{CoppolaApocalypse2019}
are referenced throughout this essay. Scene descriptions are provided to refresh readers'
memory and facilitate verification of the analysis.

\subsection*{Opening: Saigon Hotel Room}
\label{scene:saigon-opening}

Captain Willard lies intoxicated in a Saigon hotel room, haunted by memories and desperate for
a mission. The opening montage, set to The Doors' ``The End,'' crosscuts helicopter imagery
with Willard's breakdown. His voice-over introduces the thesis: ``Everyone gets everything he
wants. I wanted a mission, and for my sins they gave me one.''

\subsection*{Mission Briefing}
\label{scene:briefing}

Willard receives his orders from military intelligence officers. He is shown a dossier on
Colonel Walter Kurtz, a decorated Special Forces officer who has ``gone insane'' and now
commands his own army in Cambodia. The mission: ``terminate with extreme prejudice.'' The
briefing presents Kurtz as an administrative problem requiring surgical removal.

\subsection*{Upriver Journey (Meta-Scene)}
\label{scene:upriver-journey}

The central narrative arc of the film: Willard and his patrol boat crew (Chief, Chef, Lance,
Clean) travel up the Nung River from South Vietnam toward Cambodia to reach Kurtz. The journey
functions as a progressive stripping away of pretense and rationalization. Each checkpoint
(Kilgore's beach, the sampan, the Playboy show, Do Lung Bridge, the French plantation)
delivers a form of ``success'' or fulfillment that immediately reveals emptiness or generates
new lack. The upriver structure embodies the essay's thesis: getting what one wants (progress,
clarity, completion of objectives) exposes the wanting itself as problematic.

\subsection*{Kilgore's Beach Assault}
\label{scene:kilgore-beach}

Lieutenant Colonel Kilgore (Robert Duvall) leads an Air Cavalry helicopter assault on a coastal
village, playing Wagner's ``Ride of the Valkyries'' from loudspeakers. After securing the
beach, Kilgore famously declares, ``I love the smell of napalm in the morning.'' The scene
combines spectacular violence with aesthetic ritual, Kilgore treating the battle as
performance.

\subsection*{Sampan Search and Shooting}
\label{scene:sampan}

The patrol boat crew stops a sampan for inspection. A civilian woman moves suddenly toward a
hidden basket, and Chief panics, ordering the crew to open fire. All civilians are killed. The
basket contained a puppy. Clean, the youngest crew member, is horrified; Chief insists they
had no choice. The scene exposes how procedural ``success'' can coincide with moral horror.

\subsection*{Playboy Bunny USO Show}
\label{scene:playboy-show}

The crew attends a chaotic USO show featuring Playboy Playmates performing for thousands of
soldiers. The performance promises gratification but delivers only spectacle and frustration.
The show is abruptly evacuated as the crowd becomes unruly. The scene dramatizes craving
amplified rather than satisfied.

\subsection*{Do Lung Bridge}
\label{scene:do-lung-bridge}

The crew arrives at a nightmarish outpost where a bridge is built during the day and destroyed
by the Viet Cong each night, only to be rebuilt the next morning. No one is in charge; chaos
and violence continue without purpose. The scene embodies repetition without progress, samsaric
cycle without escape.

\subsection*{Reading Kurtz's Dossier (Resupply)}
\label{scene:dossier-reading}

After leaving Do Lung Bridge, the crew receives resupply including mail and additional
intelligence materials for Willard. Willard opens the package containing updated files on Kurtz,
including new photographs and information about his unauthorized operations. Through voiceover,
Willard processes this expanding portrait of Kurtz's trajectory from exemplary officer to rogue
commander. The scene stages surveillance and documentation as disciplinary apparatus: Willard is
both subject (reading the file) and object (himself filed, tracked, positioned by the system).
The updated dossier constructs Kurtz as simultaneously exemplary and aberrant, and positions
Willard as the reliable instrument for correction. This shows how institutional knowledge
continuously shapes and redirects desire.

\subsection*{French Plantation}
\label{scene:french-plantation}

The crew reaches a French colonial plantation maintained by the descendants of French colonists
who refuse to leave Vietnam despite the fall of French Indochina. The family has buried their
dead in the land they claim as theirs. Over an elaborate dinner, the French discuss their
history, their refusal to abandon what they built, and the American war as continuation of
colonial failure. Willard meets Roxanne, a widowed Frenchwoman. The scene stages colonial
nostalgia as ghostly persistence: empire achieved its object (possession of land, wealth,
legacy) and discovered that possession cannot justify or redeem. The plantation is a monument
to fulfillment that became haunting.

\subsection*{Kurtz's Compound}
\label{scene:kurtz-compound}

Willard finally reaches Kurtz's jungle temple, surrounded by indigenous Montagnard followers
and littered with bodies and severed heads. Kurtz (Marlon Brando) appears as both prophet and
tyrant, speaking in riddles about horror, judgment, and the hypocrisy of moral pretense. The
compound represents the endpoint of the journey: what Willard wanted to find, and what finding
it reveals.

\subsection*{Final Assassination}
\label{scene:assassination}

Willard emerges from the river, machete in hand, and kills Kurtz during a ritual slaughter.
The killing is crosscut with the sacrifice of a water buffalo, fusing Willard's act with
primal ritual. Kurtz's final words---``The horror! The horror!''---echo Conrad's novella.
After the killing, Willard walks through Kurtz's followers, who part to let him leave. He has
completed the mission but inherits Kurtz's position without resolution.
