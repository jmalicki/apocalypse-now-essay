\subsection*{III—Schopenhauer: Will as Lack and Fulfillment as Deflation}
\label{ssec:iii-schopenhauer}
Schopenhauer’s analysis of willing gives the line’s first clause its basic grammar. Will is lack; representation supplies objects that momentarily still the pressure without curing it. Hence satisfaction “at once makes room for a new one,” and life “swings like a pendulum” between suffering and boredom (\parencite[pp.~312, 319]{SchopenhauerWWR1969}). The film’s “wins”—a beach secured, a sampan inspected, a bridge crossed—illustrate the structure. Each procedural success breeds the need for another. In this light, “everyone gets everything he wants” is descriptive of delivery, not redemption.
On this view, “…and for my sins I got one” does not confess a single misdeed so much as the mistake of expecting from fulfillment what fulfillment cannot give. The “sin” is metaphysical: to confuse delivery with deliverance. The line’s two halves are internally connected. The first clause names the world’s competence at delivering objects; the second registers the punishment endemic to a will that takes attainment as cure.
