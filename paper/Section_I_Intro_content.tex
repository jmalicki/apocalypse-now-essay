\phantomsection
\section*{I. Introduction: The Paradox of Fulfilled Desire}
\addcontentsline{toc}{section}{I. Introduction: The Paradox of Fulfilled Desire}
\label{sec:i-introduction-the-paradox-of-fulfilled-desire}
When Captain Willard opens \textit{Apocalypse Now} (1979) with the line, ``Everyone gets everything he wants. I wanted a mission, and for my sins they gave me one,'' he states a moral law.\footnote{Where available, bibliographic entries include links to Google Books digitized copies of public domain and other works. These are provided for reader convenience. Page numbers reference the print editions listed in the bibliography.} Beneath the soldier’s irony lies a metaphysical claim: that desire fulfilled is inseparable from punishment. The first clause universalizes fulfillment as an inevitable structure; the second localizes it as judgment. This essay traces that paradox across traditions: biblical justice and Buddhist causality; Western philosophy’s metaphysics of will; Conrad’s colonial modernity refracted through Coppola; and modern psychology’s confrontation with mortality. Willard’s line, stripped of theology but charged with fatalism, speaks for the modern self: one who always gets what he wants—and must live with what it means.
