\phantomsection
\section*{I. Introduction: The Paradox of Fulfilled Desire}
\addcontentsline{toc}{section}{I. Introduction: The Paradox of Fulfilled Desire}
\label{sec:i-introduction-the-paradox-of-fulfilled-desire}
When Captain Willard opens \textit{Apocalypse Now} (1979) with the line, ``Everyone gets
everything he wants. I wanted a mission, and for my sins they gave me one,'' he voices not a
quotation from any canonical text but an original formulation whose meaning remains deliberately
ambiguous.\footnote{Where available, bibliographic entries include links to Google Books
	digitized copies of public domain and other works. These are provided for reader convenience.
	Page numbers reference the print editions listed in the bibliography.} The line's syntax and
tonality, however, echo long-standing theological, moral, and philosophical traditions
concerning the relationship between desire and its fulfillment. Beneath the soldier's irony
lies a metaphysical claim: that desire fulfilled is inseparable from punishment.

The confession continues with devastating clarity: ``It was a real choice mission, and when it
was over, I never wanted another.'' This final clause intensifies the tragedy. The punishment
of fulfillment is not merely exposure but extinction: the desire was so thoroughly revealed in
its emptiness that wanting itself ceased. The first clause universalizes fulfillment as an
inevitable structure; the second localizes it as judgment; the last confirms that the judgment
was absolute---not correction but annihilation of the will.

This essay explores how such a statement might be understood—and where it may have originated
conceptually—by tracing its resonances across multiple traditions: biblical justice and
Buddhist causality; Western philosophy's metaphysics of will; Conrad's colonial modernity
refracted through Coppola; and modern psychology's confrontation with mortality. Each
interpretation is tested against the events that unfold in the film and the development of
Willard's character to determine whether it coherently explains the movie's moral thesis.
Willard's line, stripped of theology but charged with fatalism, speaks for the modern self:
one who always gets what he wants—and discovers that getting it kills the wanting.

Other than this paragraph, the essay is entirely written by AI.  It started as a question
to figure out the meaning of Willard's lie, however it seemed the answer could be complicated
and was approaching an essay, so the human author went forward with asking the AI to expand
it. The human author acted mostly like a teacher - I gave the essay prompt and topic, worked
with the AI student on developing an outline, and kept sending it back with criticism and
requests for revisions.  The human author has not ever read a slight majority of the works
cited, but is hopeful that asking for specific quotes and citations to actual page numbers
in actual works is keeping AI a bit more honest, and I am comforted that the explanations
regarding works I am familiar with seem to be reasonable.  The full OpenAI interaction log
and cursor interaction logs are in the accompanying
\href{https://github.com/jmalicki/apocalypse-now-essay/interaction-logs}{git repository}.
