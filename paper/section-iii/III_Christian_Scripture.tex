\phantomsection
\subsection*{III.2—Christian Scripture: Handed Over to Desire}
\addcontentsline{toc}{subsection}{III.2—Christian Scripture}
\label{ssec:iii-christian-scripture}

The Christian Testament inherits and intensifies the Hebrew pattern. Romans 1:18--32 provides
the clearest articulation: divine ``wrath'' is revealed not as external intervention but as
permissive abandonment. Three times Paul writes, ``God gave them over'' (Rom 1:24, 26, 28):
first to the lusts of their hearts, then to dishonorable passions, finally to a debased mind.
The grammar is juridical but the mechanism is organic: suppressed truth about God leads to
idolatry, which distorts desire, and the distorted desire is then permitted its own object.
The punishment is getting what was wanted under the conditions that made the wanting corrupt.

Joseph Fitzmyer notes that the phrase ``God gave them over'' (\emph{paredōken autous ho theos})
echoes covenant-curses in Deuteronomy and uses legal terminology for handing a prisoner to
executioners---yet here the ``executioner'' is the unchecked desire itself
\parencite{FitzmyerRomans1993}. The structure matches Psalm 106:15: outward grant, inward
leanness. Notably, Paul does not claim desire is extinguished or thwarted; rather, it achieves
its aims and the achievement exposes the will's disorder. The ``wrath'' is not retributive
bolt but revelatory hand-over.

Jesus' teaching in the Synoptic Gospels complicates this with eschatological urgency. In Mark
8:36, he asks, ``For what shall it profit a man, if he shall gain the whole world, and lose
his own soul?'' The conditional is not hypothetical but diagnostic: maximal fulfillment
(``gaining the world'') can coincide with maximal loss. The Parable of the Rich Fool (Luke
12:16--21) dramatizes this: abundant harvest, ample storage, planned leisure---and death that
same night. The man got everything; the getting was judgment. Similarly, the Rich Young Ruler
(Mark 10:17--22) receives the answer he sought (``sell all, follow me''), yet ``went away
sorrowful'' because he had great possessions. Fulfillment of the inquiry revealed what could
not be relinquished.

The Johannine literature adds a mystical dimension. In John 12:25, Jesus declares, ``He that
loveth his life shall lose it; and he that hateth his life in this world shall keep it unto
life eternal.'' The paradox is structural: possessive attachment (``loveth his life'') achieves
its object and loses it in the keeping; self-dispossession opens eschatological retention.
First John 2:15--17 frames worldly desire---lust of the flesh, lust of the eyes, pride of
life---as inherently passing: ``he that doeth the will of God abideth for ever.'' The contrast
is not between frustrated and satisfied desire but between desire-structures: one that grasps
finitude and perishes with it, another that loves God and participates in eternity.

Within Christian canon, then, the pattern is consistent but variously emphasized. Synoptic
tradition warns eschatologically (gain the world, lose the soul). Pauline tradition diagnoses
juridically (handed over to the lusts). Johannine tradition frames mystically (loving life
loses it). All agree: fulfillment can be judgment when desire is disordered, and the disorder
is exposed not by denial but by grant.
