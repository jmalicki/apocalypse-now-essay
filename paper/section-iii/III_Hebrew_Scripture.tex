\phantomsection
\subsection*{III.1—Hebrew Scripture: The Gift That Judges}
\addcontentsline{toc}{subsection}{III.1—Hebrew Scripture}
\label{ssec:iii-hebrew-scripture}

The Hebrew Bible establishes a pattern where divine grant can function as judgment. The
paradigmatic text is Psalm 106:15: ``And he gave them their request; but sent leanness into
their soul'' (KJV). The verse refers to Israel's demand for meat in the wilderness: God
provides quail in abundance, yet the fulfillment brings plague. The structure is not arbitrary
punishment but disclosure---the desire for provision as if ultimate, rather than trust in
covenant faithfulness, reveals and punishes itself through its own satisfaction. The dual
movement---outward abundance, inward thinning---becomes paradigmatic for how Scripture reads
desire fulfilled.

Ecclesiastes extends this diagnostic to all earthly fulfillment: ``I have seen all the works
that are done under the sun; and, behold, all is vanity and vexation of spirit'' (Eccl.
1:14). The Preacher catalogs achievements---building, planting, accumulating---and finds that
``there is no profit under the sun'' (Eccl. 2:11). Notably, he does not lack capacity to
fulfill desire: ``whatsoever mine eyes desired I kept not from them'' (Eccl. 2:10). The
judgment comes \emph{through} the getting, not despite it. The will's misdirection toward
finite goods as if infinite produces the ``vanity.'' A person may receive ``riches, wealth,
and honour, so that he wanteth nothing for his soul,'' yet it remains ``vanity and vexation
of spirit'' (Eccl. 6:2). The gift is given; the gift exposes.

This pattern threads through Hebrew Scripture: fulfillment can reveal the disorder of the
wanting. Later interpretive traditions will systematize this into a theology of desire's
orientation, but the scriptural logic is already present: getting what one wants, when the
wanting is misdirected, does not bless but judges.
