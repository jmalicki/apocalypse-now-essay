\phantomsection
\subsection*{III.1—Hebrew Scripture: The Gift That Judges}
\addcontentsline{toc}{subsection}{III.1—Hebrew Scripture}
\label{ssec:iii-hebrew-scripture}

The Hebrew Bible establishes a pattern where divine grant can function as judgment. The
paradigmatic text is Psalm 106:15: ``And he gave them their request; but sent leanness into
their soul'' (KJV). The verse refers to Israel's demand for meat in the wilderness (Numbers
11): God provides quail in abundance, yet the fulfillment brings plague. The structure is not
arbitrary punishment but disclosure---the desire for provision as if ultimate, rather than
trust in covenant faithfulness, reveals and punishes itself through its own satisfaction.

This pattern recurs across the canon. In 1 Samuel 8, Israel demands a king ``like all the
nations.'' God grants the request through Samuel, but warns of the consequences: the king will
conscript sons, tax harvests, and make the people cry out ``because of your king which ye
shall have chosen'' (1 Sam 8:18). The gift is given; the gift judges. The wanting itself---to
be ordered like the nations rather than by covenant---contains its own privation.

Ecclesiastes extends the diagnostic to all earthly fulfillment: ``I have seen all the works
that are done under the sun; and, behold, all is vanity and vexation of spirit'' (Eccl. 1:14).
The Preacher catalogs achievements---building, planting, accumulating---and finds that ``there
is no profit under the sun'' (Eccl. 2:11). Notably, he does not lack capacity to fulfill
desire: ``whatsoever mine eyes desired I kept not from them'' (Eccl. 2:10). The judgment comes
\emph{through} the getting, not despite it. The will's misdirection toward finite goods as if
infinite produces the ``vanity.''

The prophetic tradition adds a corporate dimension. Hosea frames Israel's political alliances
(seeking security from Egypt and Assyria) as covenant-betrayal, and the alliances themselves
become the instruments of judgment (Hos. 5:13, 8:9--10). The nation ``gets'' the strategic
partnerships it wanted and finds them unable to save. Similarly, Ezekiel depicts God
``giving'' the people over to destructive statutes as a response to idolatry (Ezek. 20:25)---a
hard saying that Jewish and Christian interpreters read as either permissive judgment or
ironic exposure of futility.

Jewish interpretive tradition emphasizes that this is not divine sadism but the logical
consequence of free will oriented away from Torah. Maimonides argues that providential
``punishment'' is often the natural result of abandoning rational-moral order: the sinner is
not struck down but experiences the intrinsic emptiness of misdirected appetite
\parencite{MaimonidesGuide1963}. The structure holds: fulfillment that bypasses righteousness
punishes through its own success.
