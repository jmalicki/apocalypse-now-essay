\phantomsection
\subsection*{III.4—Buddhist Causal Analysis: Craving Fulfilled, Suffering Renewed}
\addcontentsline{toc}{subsection}{III.4—Buddhist Causal Analysis}
\label{ssec:iii-buddhist-causal-analysis}

Early Buddhist teaching locates suffering not in moral disorder but in the causal structure of
craving itself. The diagnosis is phenomenological and impersonal: no divine judge is required;
suffering arises whenever grasping meets impermanence.

The \emph{Dhammapada} states the principle concisely: ``From craving arises grief, from
craving arises fear; for one who is free of craving there is no grief or fear'' (Dhp 216;
\parencite{BuddharakkhitaDhp1993}). The logic is causal, not punitive. Craving (\emph{taṇhā})
seeks to secure what is inherently unstable (pleasure, possession, status), and the mismatch
between the fixity craving demands and the flux of conditioned phenomena generates suffering
(\emph{dukkha}). Fulfillment does not resolve this; it triggers the next cycle.

The \emph{Saṃyutta Nikāya} formalizes the structure through dependent origination
(\emph{paṭiccasamuppāda}). The twelve links are:
\begin{enumerate}
	\item Ignorance (\emph{avijjā}) conditions
	\item volitional formations (\emph{saṅkhārā}), which condition
	\item consciousness (\emph{viññāṇa}), which conditions
	\item name-and-form (\emph{nāmarūpa}), \ldots\ through the sense bases, contact, feeling, to
	\item craving (\emph{taṇhā}), which conditions
	\item clinging (\emph{upādāna}), which conditions
	\item becoming (\emph{bhava}), which conditions
	\item birth (\emph{jāti}), which conditions
	\item aging-and-death (\emph{jarāmaraṇa}), sorrow, lamentation, pain, grief, and despair.
\end{enumerate}
(SN 12.2; \parencite{BodhiSN2000}, p.~536). The key insight is that craving, once gratified,
does not terminate but ``conditions becoming''---i.e., it generates the ground for further
birth and thus further suffering. Fulfillment feeds the chain rather than breaking it.

The \emph{Majjhima Nikāya} dramatizes this with the image of the leper who warms himself by
fire: the heat momentarily relieves the itch but aggravates the underlying condition (MN 75;
\parencite{NanamoliBodhiMN1995}, p.~608). Similarly, sensual pleasure offers transient relief
yet deepens dependency. The Buddha declares that even if one were to ``conquer the earth and
sea,'' the lust for gain would not be sated; satisfaction is structurally impossible because
the craving is not for \emph{a} thing but for \emph{permanence in impermanence}.

The Fire Sermon (SN 35.28) uses visceral language: ``The eye is burning, forms are burning,
eye-consciousness is burning \ldots\ burning with what? Burning with the fire of lust, with
the fire of hate, with the fire of delusion'' (\parencite{BodhiSN2000}, p.~1143). The
metaphor is crucial: fire consumes its fuel. Craving, granted its object, does not rest but
continues to burn because the burning \emph{is} the craving. Buddhaghosa's image in the
\emph{Visuddhimagga} sharpens this: craving clings to its object ``like meat sticking to a hot
iron pan''---the very grasping causes the pain (\parencite{BuddhaghosaVisuddhi1956}).
Fulfillment cannot cool what is intrinsically aflame.

Later Madhyamaka analysis, particularly Nāgārjuna's \emph{Mūlamadhyamakakārikā}, adds an
epistemic dimension: craving presupposes that objects possess inherent existence
(\emph{svabhāva}), yet analysis reveals all phenomena as empty (\emph{śūnya}) of such essence.
When one grasps an ``inherently existent'' goal and attains it, the goal's emptiness is
disclosed---not as mere absence but as the collapse of the reified expectation. The shock of
that collapse is suffering. Thus, for Nāgārjuna, ignorance of emptiness fuels craving, craving
achieves reified aims, and the aims' non-inherence wounds the one who grasped
\parencite{NagarjunaMMK2013}. Fulfillment punishes not morally but ontologically: the object
cannot be what craving demanded it to be.

The Buddhist remedy differs fundamentally from Biblical reordering. Rather than redirecting
desire toward a highest good, the path aims at cessation (\emph{nirodha}). The Noble Eightfold
Path cultivates right view, intention, speech, action, livelihood, effort, mindfulness, and
concentration---not to intensify desire for something better but to extinguish the fires
through disenchantment (\emph{nibbidā}), dispassion (\emph{virāga}), and liberation
(\emph{vimutti}) (\parencite{Rahula1959}; \parencite{Gethin1998}). Walpola Rahula summarizes:
``The cessation of dukkha is the cessation of taṇhā'' \parencite{Rahula1959}. There is no
reorientation toward an eternal Good but a cooling, an unbinding (\emph{nibbāna}).

Coppola's river journey maps this phenomenology without endorsing its metaphysics. Do Lung
Bridge is rebuilt daily, destroyed nightly---samsaric repetition without progress. The Playboy
show promises gratification, amplifies agitation, and leaves the crew more restless: craving
fed, suffering renewed. The mission itself, completed, does not bring rest but disillusionment.
In Buddhist terms, each ``success'' is another link in the chain, and the chain's logic is
exposure: what you wanted cannot satisfy, because wanting is the problem.
