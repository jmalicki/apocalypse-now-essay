\phantomsection
\subsection*{Buddhist Causal Analysis: Craving Fulfilled, Suffering Renewed}
\addcontentsline{toc}{subsection}{Buddhist Causal Analysis}
\label{ssec:iii-buddhist-causal-analysis}

Early Buddhist teaching locates suffering not in moral disorder but in the causal structure of
craving itself. The \emph{Dhammapada} states the principle: ``From craving arises grief, from
craving arises fear; for one who is free of craving there is no grief or fear'' (Dhp. 216;
\parencite{BuddharakkhitaDhp1993}). The logic is causal, not punitive. Craving (\emph{taṇhā})
seeks to secure what is inherently unstable, and fulfillment does not resolve this---it
triggers the next cycle.

The \emph{Saṃyutta Nikāya}'s teaching of dependent origination
(\emph{paṭiccasamuppāda}) formalizes this: craving conditions clinging, clinging conditions
becoming, becoming conditions suffering (SN 12.2; \parencite[p.~536]{BodhiSN2000}). The key
insight is that craving, once gratified, does not terminate
but perpetuates the chain. Satisfaction is structurally impossible because the craving is not
for any particular thing but for permanence in what is impermanent. Fulfillment feeds the
chain rather than breaking it.

The Fire Sermon makes this visceral: ``The eye is burning \ldots\ burning with the fire of
lust, with the fire of hate, with the fire of delusion'' (SN 35.28;
\parencite[p.~1143]{BodhiSN2000}). Fire consumes its fuel. Craving, granted its object, does
not rest but continues to burn because the burning \emph{is} the craving. Fulfillment cannot
cool what is intrinsically aflame.

The remedy differs fundamentally from Biblical reordering. Rather than redirecting desire
toward a highest good, the Noble Eightfold Path aims at cessation (\emph{nirodha})---
extinguishing the fires through disenchantment, dispassion, and liberation
\parencite[pp.~45--50]{Rahula1959}. There is no reorientation toward an eternal Good but a
cooling, an unbinding (\emph{nibbāna}).

Coppola's \hyperref[scene:upriver-journey]{river journey} maps this phenomenology without
endorsing its metaphysics. \hyperref[scene:do-lung-bridge]{Do Lung Bridge} is rebuilt daily,
destroyed nightly---samsaric repetition without progress. The
\hyperref[scene:playboy-show]{Playboy show} promises gratification, amplifies agitation, and
leaves the crew more restless: craving fed, suffering renewed. The
\hyperref[scene:french-plantation]{French plantation} dramatizes clinging (\emph{upādāna}): the
colonists achieved their object---land, dynasty, legacy---and will not let go even as
fulfillment becomes living death. They cling to what they won, and the clinging perpetuates
suffering. The mission itself, completed, does not bring rest but disillusionment. In Buddhist
terms, each ``success'' is another link in the chain, and the chain's logic is exposure: what
you wanted cannot satisfy, because wanting is the problem.
