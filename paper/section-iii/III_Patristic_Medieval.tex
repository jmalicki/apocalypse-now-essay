\phantomsection
\subsection*{III.3—Patristic and Medieval Synthesis: Disorder, Privation, Telos}
\addcontentsline{toc}{subsection}{III.3—Patristic and Medieval Synthesis}
\label{ssec:iii-patristic-medieval}

The Church Fathers and medieval scholastics systematize the scriptural pattern into a
psychology and metaphysics of desire. Their contribution is to distinguish not \emph{that} we
desire but \emph{how} we desire---the orientation, object, and telos of willing.

Augustine of Hippo interiorizes the diagnosis in the \emph{Confessions}. His famous line,
``Every disordered affection is its own punishment'' (\emph{omnis inordinatus animus sibi ipsi
	supplicium est}), appears early in the work (2.2.2) as he reflects on adolescent theft
\parencite[p.~47]{AugustineConfessions1998}. The point is not that theft was punished by
external penalty but that the act of willing against order produced internal fragmentation.
Later, Augustine develops the distinction between \emph{cupiditas} (possessive desire, ``use''
of what should be enjoyed, orientation toward finite goods as ultimate) and \emph{caritas}
(rightly ordered love, ``enjoyment'' of God and ``use'' of creatures toward that end). In
\emph{De doctrina christiana}, he writes: ``To enjoy something is to cling to it with love for
its own sake. To use something, however, is to employ it in obtaining that which you love''
\parencite{AugustineDeDoctrina1958}. Fulfillment of \emph{cupiditas} punishes because the
finite cannot bear the weight of ultimacy; only God, as infinite Good, can satisfy without
remainder.

In \emph{City of God}, Augustine extends this to social and political order. The ``earthly
city'' organizes itself around self-love (\emph{amor sui}), seeking dominion and glory. It
succeeds---Rome conquers the Mediterranean---and the success exposes the emptiness: endless
wars, internal factionalism, the anxiety of preserving what was won
\parencite{AugustineCity2003}. The city ``gets everything it wants'' (imperium, tribute,
fame) and finds that the getting binds rather than frees. By contrast, the ``heavenly city''
orients love toward God (\emph{amor Dei}), which alone reorders desire so that fulfillment
participates in rather than exhausts the Good. Crucially, for Augustine, this reordering
requires grace: the will cannot fix itself by willing harder but must be healed by the One
toward whom it should tend.

Thomas Aquinas formalizes the metaphysics in the \emph{Summa Theologiae}. Every act of will,
he argues, aims at some good, real or apparent (ST I--II, q.~19, a.~1). Sin occurs not when
the will targets evil \emph{as such} (nothing wills its own negation) but when it turns away
from the immutable Good toward a mutable one as if it were ultimate---a movement Aquinas calls
\emph{aversio a Deo} (aversion from God) coincident with \emph{conversio ad creaturam}
(turning toward the creature) (ST I--II, q.~19, a.~9; \parencite{AquinasST1947}). The
problem is not the creature---finite goods are genuinely good---but the absolutizing.
Fulfillment of this misdirected will is privative: the good achieved cannot deliver what was
implicitly demanded (ultimacy), and the gap between expectation and reality is punishment.
Aquinas adds that this structure applies even to virtuous acts done for vainglory: the act
succeeds, recognition is gained, and the recognition reveals that honor sought \emph{for its
	own sake} cannot satisfy the rational appetite for the Absolute (ST II--II, q.~132).

Gregory of Nyssa complicates the Augustinian synthesis with his doctrine of \emph{epektasis}:
the soul's infinite ascent toward the infinite God. In \emph{Life of Moses}, Gregory interprets
Moses' request to see God's glory and the divine reply (``Thou canst not see my face'') as a
paradox: the vision is granted by being perpetually deferred, so that desire intensifies
rather than satiates \parencite{GregoryMoses1978}. Later, in the \emph{Homilies on the
	Beatitudes}, he argues that the blessed, having fulfilled one level of virtue, discover a
higher one, ad infinitum \parencite[p.~31]{GregoryBeatitudes1954}. This might seem to
contradict the ``fulfillment punishes'' structure, but Gregory's point is that
\emph{possessive} fulfillment (desire that grasps and exhausts) fails, while participatory
desire (asymptotic approach to an inexhaustible Good) continually fulfills without depleting.
Punishment still attaches to the former, not the latter.

The patristic-medieval consensus, then, is teleological: desire must be ordered toward an
appropriate end. When finite goods are taken as ultimate, their attainment exposes the
disjunction between what they are (genuinely but limitedly good) and what was demanded
(ultimate satisfaction). Augustine names this ``disordered affection''; Aquinas calls it
``aversion from God.'' Both agree: the disorder is its own punishment, and grace must
intervene to reorient the will toward its proper end.
