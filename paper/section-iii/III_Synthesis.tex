\phantomsection
\subsection*{Comparative Synthesis: Binding, Punishment, Healing}
\addcontentsline{toc}{subsection}{Comparative Synthesis}
\label{ssec:iii-synthesis}

Hebrew Scripture, Christian New Testament, patristic-medieval theology, and early Buddhism
converge on the paradox that fulfillment can be judgment---yet they diverge profoundly in
ontology, diagnosis, and remedy.

\textbf{What binds?} Biblical tradition names the bondage as \emph{sin}: love disordered
toward finite goods as if ultimate. Psalm 106 shows Israel getting what it wants and
discovering the want itself was misdirected. Augustine systematizes this in the
\emph{Confessions} as \emph{cupiditas} (possessive desire) versus \emph{caritas} (rightly
ordered love) \parencite[p.~47]{AugustineConfessions1998}. Aquinas locates the disorder in
\emph{aversio a Deo}: turning from the infinite Good toward creatures absolutized (ST I--II,
q.~19, a.~9; \parencite{AquinasST1947}). The problem is not desire \emph{per se} but its
target and manner.

Buddhist tradition names the bondage as \emph{taṇhā-upādāna}: craving and clinging that arise
from ignorance of impermanence and non-self. The \emph{Dhammapada} (Dhp. 216) and
\emph{Saṃyutta Nikāya} (SN 12.2) diagnose suffering as endogenous to grasping
\parencite{BuddharakkhitaDhp1993}; \parencite[p.~536]{BodhiSN2000}. The problem is not
misdirection but \emph{direction-as-such}---any attachment to conditioned phenomena
perpetuates becoming.

\textbf{What punishes?} In Biblical-patristic idiom, the will is ``handed over'' to its object
(Romans 1:24), or fulfillment brings ``leanness of soul'' (Psalm 106:15). Augustine writes
that disordered affection is its own punishment; Aquinas that turning from God entails
privation. The punishment is not external retribution but the intrinsic failure of finite goods
to bear infinite demand. The mechanism is teleological: a will aimed wrongly cannot reach its
true end, and the mismatch is experienced as emptiness, fragmentation, or despair.

In Buddhist idiom, gratification feeds the next link in dependent origination. Craving
conditions clinging, clinging conditions becoming, becoming conditions birth, birth conditions
aging-and-death (SN 12.2). The punishment is causal and impersonal: no judge assigns penalty,
but the structure of conditioned arising guarantees that satisfied craving generates new
suffering. The Fire Sermon's imagery---sense faculties ``burning'' with lust, hate,
delusion---shows that the fire is not fuel-dependent; it \emph{is} the craving itself, so
``feeding'' it intensifies rather than extinguishes it.

Both traditions understand Coppola's \hyperref[scene:sampan]{sampan episode} identically in
form but differently in ground: the mission ``succeeds,'' the
\hyperref[scene:sampan]{unarmed civilians are killed}, the crew's affect goes hollow. Biblical
interpretation reads this as disordered desire
(securing the mission as ultimate) exposed through its own success. Buddhist interpretation
reads it as craving (security, mission-completion) perpetuating suffering through another link
in the chain. The
phenomenology is the same; the metaphysics differ.

\textbf{What heals?} Biblical-patristic tradition proposes grace re-ordering love. Augustine
insists in the \emph{City of God} that the will, enslaved to \emph{cupiditas}, cannot free
itself but must be healed by participation in divine love (DCD XIV.28;
\parencite{AugustineCity2003}). Aquinas agrees: only when desire is rightly ordered toward God
as ultimate end can finite goods be enjoyed without privation (ST I--II, q.~19;
\parencite{AquinasST1947}). Gregory of Nyssa's \emph{epektasis} in the \emph{Life of Moses}
suggests that infinite desire directed toward the infinite God becomes non-possessive ascent
rather than grasping \parencite[pp.~113--114]{GregoryMoses1978}. The remedy is
\emph{reorientation}: not ceasing to desire but desiring rightly.

Buddhist tradition proposes \emph{cessation}. The Noble Eightfold Path cultivates
disenchantment with conditioned phenomena, leading to the cooling of craving and liberation
from the cycle \parencite[pp.~45--50]{Rahula1959}. There is no reorientation toward a higher
Good but a progressive unbinding from all objects of attachment.

The two paths are not easily reconciled. Biblical-patristic thought insists that the soul is
made for God and only infinite Good can satisfy; desire rightly ordered ascends eternally
without exhaustion. Early Buddhism insists that any conditioned object, even ``God,'' remains
within the cycle and thus subject to dukkha; only cessation of craving brings peace. One
infinitizes desire toward the Absolute; the other extinguishes grasping altogether.

Yet both agree on the diagnostic: finite fulfillment, sought possessively, punishes. The
administrative mission that Willard receives promises clarity, purpose, even redemption. It
delivers exposure: of complicity (Biblical reading) or of samsaric repetition (Buddhist
reading). ``Everyone gets everything he wants''---the structure is universal. ``For my sins I
got one''---the judgment is intrinsic. The film offers no remedy, only the vision of
fulfillment as wound. Whether healing requires reordered love or cessation of craving remains
theologically contested. What the \hyperref[scene:upriver-journey]{river journey} confirms is
that getting what one wants, when the wanting is disordered or grasping, does not liberate. It
binds---and the binding is the punishment.
