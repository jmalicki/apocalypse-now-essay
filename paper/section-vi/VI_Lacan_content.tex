\phantomsection
\subsection*{Lacan: Desire, Demand, and the Object That Is Always Missing}
\addcontentsline{toc}{subsection}{Lacan: Desire and the Missing Object}
\label{ssec:vi-lacan}

Jacques Lacan radicalizes Freud by insisting that desire is not a biological drive but a
structural effect of language and the symbolic order. His formula is stark: desire is ``the
desire of the Other''---we want to be desired, to be the object that would complete the
Other's lack \parencite{LacanEcrits2006}. But this means desire is fundamentally
unsatisfiable, because it is not oriented toward any object in the world but toward a
recognition that can never be secured. Demand can be met (give me water, give me a mission),
but desire persists beyond every satisfied demand as an irreducible remainder.

Lacan distinguishes \emph{need}, \emph{demand}, and \emph{desire}. Need is biological (hunger,
thirst); demand is the articulation of need in language, which always asks for more than the
object---it asks for love, recognition, the Other's presence. Desire is what remains when
demand is subtracted from need: the excess that no object can fill
\parencite{LacanEcrits2006}. When Willard demands a mission, the institution supplies one. But
what he \emph{desires}---orientation, selfhood, confirmation that his existence
matters---cannot be delivered by any dossier. ``For my sins I got one'' is the recognition
that the object satisfied the demand but left desire untouched.

The concept of \emph{objet petit a} (object-cause of desire) is central here. This is not the
object we want but the object that \emph{causes} wanting---a structural void around which
desire circulates \parencite[p.~103]{LacanSeminarXI1991}. It ``is never the object of need or
demand,'' and its attainment is impossible because it does not exist as a thing; it is a gap.
The mission functions as \emph{objet a}: Willard treats it as if possessing it (completing it)
would fill the lack, but the mission is only the placeholder for a desire that has no
terminus. Getting what he wanted exposes the wanting as a relation to an absence, not a goal.

Lacan's mirror stage further clarifies the film's visual economy. The infant sees its image in
the mirror and misrecognizes itself as whole, unified, masterful---a fiction that founds the
ego but is always alienated (the image is outside, not me) \parencite{LacanEcrits2006}.
Willard's journey is structured by mirrors: the \hyperref[scene:briefing]{dossier} presents
Kurtz's image, the \hyperref[scene:kurtz-compound]{photojournalist} reflects back a narrative,
Kurtz himself becomes the screen onto which Willard projects. When
\hyperref[scene:assassination]{Willard kills Kurtz}, he does not escape the mirror; he steps
into it. The ego wanted to see itself as sovereign agent;
it got that image and discovered the image is hollow.

Lacan also theorizes the ``split subject'' ($\cancel{S}$)---barred from itself by language,
always mediated, never coinciding with its own speech \parencite{LacanEcrits2006}. The
subject cannot
possess itself fully because it is constituted through signifiers that belong to the Other.
Willard's narration performs this split: he speaks himself, but the words are institutional,
procedural, borrowed. ``I wanted a mission'' sounds like self-disclosure, but the wanting and
the wanting's vocabulary are effects of the symbolic order (military training, cultural
narratives of heroism, Cold War ideology). Fulfillment punishes because the delivery of the
object---the mission accomplished---reveals that the subject who wanted was never a unified
origin but a split produced by the Other's discourse.

Finally, Lacan insists that analysis does not cure desire; it transforms the subject's relation
to it. To ``traverse the fantasy'' is to stop imagining that the object would complete you
\parencite{LacanSeminarXI1991}. Willard's ending offers no such traversal. He completes the
mission and inherits Kurtz's place, but the fantasy (that completing it would mean something)
remains untouched. The lack persists. In Lacan's terms, the subject got everything demanded
and learned that desire was never in the demand.
