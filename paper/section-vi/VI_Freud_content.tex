\phantomsection
\subsection*{Freud: The Death Drive, Repetition Compulsion, and Civilization's Discontents}
\addcontentsline{toc}{subsection}{Freud: Death Drive and Repetition}
\label{ssec:vi-freud}

Sigmund Freud's later work confronts phenomena that unsettle his earlier pleasure-principle
model: why do neurotics compulsively repeat traumatic experiences? Why do patients resist cure?
In \textit{Beyond the Pleasure Principle} (1920), Freud hypothesizes a ``death drive''
(\emph{Todestrieb})---a tendency in all organic life to ``restore an earlier state of things,''
ultimately the inanimate \parencite[p.~38]{FreudBeyond1955}. This drive does not seek pleasure
but discharge, not satisfaction but return to zero. The ``daemonic'' character of repetition
``overrides the pleasure principle'' \parencite[p.~22]{FreudBeyond1955}: the subject
re-enacts precisely what wounded him, as if compelled to master through repetition what he
could not master in the original experience.

Willard's opening confession---``I wanted a mission''---reads, in this frame, as a repetition
compulsion. He has already completed missions; they did not satisfy. Yet he wants another, as
if the next iteration might finally discharge the tension the previous ones installed. The
\hyperref[scene:upriver-journey]{upriver journey} literalizes this: each checkpoint repeats
the pattern (encounter, violence, continuation), and none resolves. The
\hyperref[scene:do-lung-bridge]{Do Lung Bridge}, rebuilt nightly only to be destroyed, is
Freud's fort-da game writ large---the child throws the spool away and reels it back,
mastering
absence through symbolic repetition \parencite[pp.~14--16]{FreudBeyond1955}. But mastery never
arrives; the compulsion only intensifies.

In \textit{Civilization and Its Discontents} (1930), Freud reframes the contradiction
socially: civilization requires renunciation of instinct, and the price is guilt. The more
successfully culture sublimates aggression into work and law, the more the leftover aggression
turns inward as superego \parencite[pp.~70--97]{FreudCivilization1961}. This is why ``getting
what one wants'' (order, justice, security) feels punitive: the very achievements of
civilization intensify the psychic burden. The
\hyperref[scene:french-plantation]{French plantation} stages this perfectly: the colonists
achieved civilization (order, culture, refinement, dynastic continuity) and dwell in a space
saturated with death. They literally live in a graveyard, maintaining elaborate ritual
(formal dinners, cultural sophistication) atop buried bodies. Civilization delivered, and the
death drive's presence became unavoidable. Willard's mission is both sublimation (redirecting
violent impulse into ``lawful'' assassination) and the return of what was sublimated (he must
kill to preserve the order that forbids killing). The guilt is structural, not contingent.

Freud also distinguishes Eros (binding, connection) from Thanatos (unbinding, aggression).
Civilization is the battleground where these forces clash
\parencite[pp.~81--92]{FreudCivilization1961}.
The mission-form channels Thanatos under the sign of Eros: violence is presented as care
(``terminate to restore order''), destruction as construction. When the mission is fulfilled,
the Eros-claim collapses, leaving only the Thanatos that was operative all along. ``For my
sins I got one'' is the moment the sublimation fails and the drive's true orientation is
revealed. Fulfillment punishes because it strips the Eros-mask from Thanatos.

Finally, Freud insists that the unconscious knows no negation and no time: repressed content
persists, undischarged, awaiting return \parencite[pp.~166--171]{FreudBeyond1955}. Willard's
voiceover has this timeless, circular quality---he narrates the past as if it were still
present, the mission as if it were still ahead. The death drive would predict that
satisfaction is always deferred, always returned as renewed lack. Yet Willard's full
confession includes a crucial coda: ``when it was over, I never wanted another.'' This presents
a puzzle for Freud's model. Either the death drive was finally satisfied (unlikely, as it
seeks the inanimate, not mission-completion), or the exposure was so complete it extinguished
the compulsion---not through cure but through exhaustion. The wanting stopped not because the
object delivered but because the will itself was killed.
