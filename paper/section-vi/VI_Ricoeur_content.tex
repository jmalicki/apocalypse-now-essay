\phantomsection
\subsection*{Ricoeur: The Hermeneutics of Suspicion and Second Naïveté}
\addcontentsline{toc}{subsection}{Ricoeur: Hermeneutics of Suspicion}
\label{ssec:vi-ricoeur}

Paul Ricoeur's \textit{Freud and Philosophy: An Essay on Interpretation} (1970) bridges
psychoanalytic theory and philosophical hermeneutics. He identifies Freud, Marx, and Nietzsche
as the ``masters of suspicion''---thinkers who teach us to read consciousness as symptom
rather than transparent self-knowledge \parencite{RicoeurSymbol1970}. For Freud, consciousness
masks unconscious drives; for Marx, it masks class interest; for Nietzsche, it masks will to
power. All three demand a hermeneutic that interprets the manifest content to reveal latent
structures. The point is not to destroy meaning but to refashion understanding by confronting
what consciousness disavows.

This essay has practiced such a hermeneutic on Willard's line. Reading it through Augustine
reveals desire's orientation toward or away from God; through Freud, the death drive's
repetition; through Foucault, disciplinary normalization. Each reading is suspicious: it
refuses to take the line at face value and instead asks what structure it symptomatizes.
Ricoeur's contribution is to insist that suspicion is not the end. After exposure comes the
``second naïveté''---a post-critical reception of symbols that can take them up
\emph{through} interpretation, not in innocent ignorance but in educated commitment
\parencite{RicoeurSymbol1970}.

Ricoeur argues that symbols ``give rise to thought''---they are not mere illustrations of
concepts but occasions for philosophical reflection \parencite[p.~347]{RicoeurSymbol1970}.
Willard's line is such a symbol. Its surface is simple (a soldier's ironic complaint); its
depth is inexhaustible (theological judgment, metaphysical necessity, historical complicity,
psychological defense). The multiplicity of readings is not a failure of determinacy but the
symbol's richness. Each tradition reveals a dimension of the truth.

Yet Ricoeur also warns against reducing symbols to abstractions. The symbol retains a
``surplus of meaning'' that resists conceptual capture \parencite{RicoeurSymbol1970}. To
translate ``for my sins I got one'' entirely into Schopenhauer's pendulum or Lacan's
\emph{objet a} is to lose the concrete particularity of Willard's voice, the film's images,
the historical specificity of Vietnam. The task of hermeneutics is not to exhaust the symbol
but to let it disclose a world. The symbol's power is that it says more than any single
interpretation can recover.

Ricoeur's method also clarifies the essay's structure. Each section has practiced a different
hermeneutic: theological (divine judgment), metaphysical (the will's ontology), historical
(colonial systems), psychological (death-denial). No single reading is complete; together,
they triangulate the symbol's meaning. This is not relativism (``all interpretations are
equal'') but a disciplined pluralism: each lens reveals what the others miss, and the symbol
is richer than any reduction.

Finally, Ricoeur's concept of second naïveté suggests a way forward that the film itself does
not provide. After reading Willard's line through these traditions, one cannot return to
innocent hearing. The symbol has been exposed as layered: theological, existential,
ideological, psychological. But one can receive it again as a claim on conscience---not as a
timeless law but as a question: \emph{How should I want, knowing that getting what I want will
	reveal what my wanting was?} Ricoeur does not answer the question, but he insists that the
hermeneutic work prepares the ground for asking it responsibly \parencite{RicoeurSymbol1970}.
The mission cannot be undone, but the will can be re-formed. That is the second naïveté's
wager.
