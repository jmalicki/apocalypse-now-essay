\phantomsection
\subsection*{VI.5—Frankl: Meaning as Antidote, and Its Limits}
\addcontentsline{toc}{subsection}{VI.5—Frankl: Meaning and Its Limits}
\label{ssec:vi-frankl}

Viktor Frankl's logotherapy, born from his experience in Nazi concentration camps, offers a
corrective to Freud's emphasis on drive and Jung's on individuation. Frankl argues that the
primary human motivation is not pleasure (Freud) or power (Adler) but \emph{meaning}.
``Those who have a 'why' to live can bear with almost any 'how,''' he writes, quoting
Nietzsche approvingly \parencite[p.~104]{FranklMeaning2006}. Even in extremity, the search
for meaning can sustain life. Suffering becomes bearable when it is understood as purposeful.

Yet Frankl insists meaning cannot be willed directly; it must be discovered as a by-product of
commitment to something beyond oneself---love, work, a cause \parencite[pp.~110--115]{FranklMeaning2006}.
This is why the pursuit of happiness for its own sake fails: happiness is a side effect of
meaningful engagement, not an achievable target. Applied to Willard's confession, the mission
furnishes a \emph{how} (procedural steps, objectives, action) without furnishing a \emph{why}
(a transcendent purpose that would justify the how). The structure is meaning's shell without
its substance.

Frankl's concept of the ``existential vacuum'' describes precisely Willard's Saigon malaise: a
state of inner emptiness where the absence of meaning manifests as boredom, aggression, or
addiction \parencite[pp.~127--129]{FranklMeaning2006}. The mission appears as a cure for this
vacuum, and in one sense it works---it dispels the aimlessness. But Frankl distinguishes
\emph{provisional} meaning (tied to a finite project) from \emph{ultimate} meaning (tied to
values that survive the project's end). The mission is purely provisional. When it is
completed, the vacuum returns, because the mission never connected to anything beyond itself.

Frankl also addresses the question of guilt. He argues that guilt can be meaningful if it
leads to change, to turning toward responsibility \parencite[pp.~131--133]{FranklMeaning2006}.
But guilt that only confirms one's worthlessness is pathological. Willard's line---``for my
sins I got one''---could be read either way. If it leads to transformation, it is meaningful
guilt; if it only deepens the sense of being trapped in a meaningless cycle, it is existential
neurosis. The film refuses the redemptive reading. No transformation follows. Willard's
completion of the mission does not reorient him toward love or creation; it only delivers him
to the next iteration of lack.

Frankl's most important caution is that meaning is \emph{found}, not \emph{made}. One cannot
manufacture significance through achievement alone; significance emerges from the orientation
of one's acts toward something worthy of commitment \parencite[pp.~115--121]{FranklMeaning2006}.
In this light, the mission is a counterfeit: it has the form of meaning (narrative, struggle,
completion) but lacks the substance (commitment to a value that transcends the self).
``Everyone gets everything he wants'' here becomes: the world supplies projects, even heroic
ones. But projects do not automatically bear meaning. ``For my sins I got one'' acknowledges
that the delivered project was not meaningless because it failed, but because it was
instrumentalized from the start---a means without an end worthy of a human life.
