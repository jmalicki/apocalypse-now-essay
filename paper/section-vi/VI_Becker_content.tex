\phantomsection
\subsection*{VI.4—Becker: Terror Management and Immortality Projects}
\addcontentsline{toc}{subsection}{VI.4—Becker: Terror Management}
\label{ssec:vi-becker}

Ernest Becker's \textit{The Denial of Death} (1973) synthesizes psychoanalysis, anthropology,
and existentialism into a single thesis: human culture is fundamentally a system for denying
mortality. ``The irony of man's condition is that the deepest need is to be free of the
anxiety of death and annihilation; but it is life itself which awakens it''
\parencite[p.~66]{BeckerDenial1973}. We are animals who know we will die, and this knowledge
is unbearable. Culture provides ``hero-systems''---symbolic frameworks (religion, nation,
profession, art) through which we can achieve significance that transcends bodily death. We
want missions, achievements, legacies---anything that promises to inscribe us beyond the flesh.

Becker argues that all human striving is, at bottom, an immortality project. The neurotic
seeks heroism through private myth; the well-adjusted finds it through cultural roles; the
creative seeks it through art. But every hero-system is a lie, because death is non-negotiable
\parencite[pp.~26--27]{BeckerDenial1973}. Fulfillment within a hero-system therefore cannot
satisfy the terror it was built to mask. When the hero ``gets what he wants''---the trophy,
the mission completed, the name remembered---he also gets the knowledge that the system was a
defense, not a solution. The trophy does not make him immortal; it only makes the lie visible.

Willard's desire for a mission is, in Becker's terms, a bid for heroic transcendence. Saigon's
aimlessness is unbearable not because it is boring but because it confronts him with
meaninglessness, which is a cipher for death. A mission supplies narrative: a beginning
(orders received), a middle (trials overcome), an end (objective achieved). This narrative
structure is itself a symbolic conquest of time and death---it gives shape to existence,
implying that one's acts matter beyond their moment. ``For my sins I got one'' is the moment
the narrative delivers, and the delivery exposes the terror it was meant to hide. Completion
does not transcend; it only clarifies that transcendence was never on offer.

Becker also insists that hero-systems turn murderous when challenged. If my immortality
project depends on a particular myth (American exceptionalism, military honor, the righteousness
of the mission), then anyone who threatens that myth threatens my symbolic life. I must
eliminate the threat to preserve the defense \parencite[pp.~123--125]{BeckerDenial1973}. This
explains the peculiar violence of ideological wars: they are not about territory but about
competing hero-systems. Kurtz represents a different myth---sovereignty through unmasked
will---and the institution cannot tolerate it. Eliminating him is not strategic; it is
psychological---the system protecting its own immortality project from a rival narrative.

Yet Becker does not counsel despair. He distinguishes ``morbid'' heroism (neurotic, private,
destructive) from ``natural'' heroism (creative, self-transcending, life-affirming)
\parencite[pp.~153--175]{BeckerDenial1973}. Natural heroism acknowledges death without denial,
finds meaning in finite acts, and does not require domination to sustain itself. Willard's
journey offers no such heroism. The mission is morbid from the start: a role that requires
denying others' humanity to preserve one's own symbolic significance. When it succeeds, the
morbidity is exposed. The will wanted to be heroic and discovered it was only performing a
script written by terror.
