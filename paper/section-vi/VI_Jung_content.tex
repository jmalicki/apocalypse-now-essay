\phantomsection
\subsection*{Jung: Shadow, Persona, and the Dark Double}
\addcontentsline{toc}{subsection}{Jung: Shadow and the Dark Double}
\label{ssec:vi-jung}

Carl Jung's analytical psychology reframes neurosis as failed integration. The psyche is not a
unified ego but a collective of structures: the \emph{persona} (social mask), the
\emph{shadow} (repressed traits), the \emph{anima/animus} (contrasexual complement), and the
\emph{Self} (the totality toward which individuation aims) \parencite{JungArchetypes1969}.
Pathology arises when the ego identifies with the persona and disowns the shadow---projecting
it onto others, denying its own darkness. Jung's claim is therapeutic but also ethical: what
you refuse to integrate, you will enact unconsciously.

Willard and Kurtz are Jungian doubles. Kurtz represents everything Willard's persona (the
disciplined soldier, the procedural executor) must disown: excess, sovereignty,
unrationalized violence. Yet the dossier describes Kurtz as exemplary before his
breakdown---decorated, brilliant, a model officer. The shadow, Jung notes, often contains not
only vice but repressed potentials for greatness \parencite{JungArchetypes1969}. Kurtz is what
Willard might become if the persona cracks. The mission to ``terminate'' Kurtz is therefore an
attempt to kill the shadow, but Jung warns that such attempts only strengthen the projection.
When \hyperref[scene:assassination]{Willard completes the mission}, he does not exorcise the
shadow; he integrates it---he becomes Kurtz.

Jung's process of \emph{individuation} requires confronting the shadow, recognizing it as part
of oneself, and incorporating it into a larger wholeness \parencite{JungArchetypes1969}. This
is not moral relativism (``accept your darkness'') but psychological realism: denied contents
return as compulsion. The film's structure enacts this process literally: the
\hyperref[scene:upriver-journey]{journey upriver} is a descent into the psyche, each checkpoint
revealing more of what the ego disavowed (the
\hyperref[scene:sampan]{sampan}: murderous efficiency; the
\hyperref[scene:do-lung-bridge]{bridge}: meaningless repetition; the
\hyperref[scene:playboy-show]{Playboy show}: libidinal consumption).
\hyperref[scene:kurtz-compound]{Kurtz} is the final station---the shadow fully constellated.

Yet Jung also warns against \emph{inflation}: identifying the ego with the Self, imagining
oneself to be the totality rather than a fragment within it. In \textit{Aion}, he argues that
such inflation generates moral catastrophe because the inflated ego loses the capacity for
self-critique \parencite{JungAion1969}. Kurtz's pronouncement---``the horror''---sounds like
an inflated verdict, a claim to have seen the whole when in fact he has only seen his own
shadow absolutized. Willard, by contrast, does not pronounce. His silence at the end might be
read as non-inflation: he knows he has confronted the shadow but refuses to mistake that
confrontation for wholeness.

Jung's theory also explains why fulfillment feels like judgment. The shadow contains not only
what is evil but what is incompatible with the persona's self-image. To ``get what one
wants''---in this case, to complete the heroic mission---is to activate the shadow material
the mission entailed: complicity, instrumentalization, the capacity for killing. The persona
wanted to be the clean executor; the shadow reveals the executor as implicated. ``For my sins
I got one'' reads as the ego's recognition that the mission constellated precisely the
contents it worked to deny. Integration is possible, but only if the ego stops projecting and
owns what it has done. The film offers no such ownership, only recursion.
