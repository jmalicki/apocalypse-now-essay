\phantomsection
\subsection*{The Psychological Convergence: Desire as Death-Denial}
\addcontentsline{toc}{subsection}{The Psychological Convergence}
\label{ssec:vi-synthesis}

Where theology diagnoses sin and philosophy diagnoses metaphysical lack, psychology diagnoses
defense. The convergence across Freud, Lacan, Jung, Becker, and even Frankl is stark: what we
consciously want (the mission, the object, the achievement) masks what we unconsciously flee
(death, void, meaninglessness). Fulfillment punishes because getting the object exposes the
futility of the defense. The mission cannot grant immortality, the fantasy cannot fill the
lack, the persona cannot eliminate the shadow.

Yet the psychologists diverge on remedy. Freud offers no exit from the drives' antagonism;
Lacan teaches traversing the fantasy; Jung demands integration of the shadow; Becker
distinguishes morbid from natural heroism; Frankl insists meaning must be found, not willed;
Ricoeur proposes second naïveté through interpretation. {\v{Z}}i{\v{z}}ek adds the darkest
note: even knowing the fantasy is a lie, we enjoy our symptom and persist. The disagreement is
not trivial---it determines whether Willard's ending is tragic necessity (Freud), missed
opportunity for integration (Jung), or cynical complicity ({\v{Z}}i{\v{z}}ek).

Applied to the film, the psychological lens reveals why Willard's tone is so affectless: the
ego has exhausted its defenses. Each checkpoint stripped another layer (the sampan: moral
purity; the bridge: rational order; Kurtz: sovereign mastery), until only the bare structure
remains---a will that wanted symbolic transcendence and received mortality's mirror. The
mission delivered everything demanded and nothing desired. What returns is not guilt
(theology), not existential clarity (philosophy), not institutional critique (critical theory),
but the psyche's exposure to its own terror. The will got what it wanted and learned it was
running from death all along.
