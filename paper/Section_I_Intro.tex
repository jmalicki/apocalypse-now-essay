\documentclass[12pt]{article}
\usepackage[margin=1in]{geometry}
\usepackage{setspace}
\usepackage{parskip}
\usepackage{csquotes}
\usepackage{hyperref}
\usepackage{fontspec} %% Compile with XeLaTeX or LuaLaTeX
\usepackage[style=apa,backend=biber]{biblatex}
\addbibresource{everyone.bib}

\setmainfont{Libertinus Serif}
\setsansfont{Libertinus Sans}
\setmonofont{Libertinus Mono}
\defaultfontfeatures{Ligatures=TeX}
\linespread{1.5}

\title{“Everyone Gets Everything He Wants”: Desire, Fulfillment, and the Tragic Logic of Will in \textit{Apocalypse Now}}
\author{(Your Name)}
\date{\today}

\begin{document}
\maketitle

\section*{I. Introduction: The Paradox of Fulfilled Desire}
When Captain Willard opens \textit{Apocalypse Now} (1979) with the line, ``Everyone gets everything he wants. I wanted a mission, and for my sins they gave me one,'' he states a moral law. Beneath the soldier’s irony lies a metaphysical claim: that desire fulfilled is inseparable from punishment. The first clause universalizes fulfillment as an inevitable structure; the second localizes it as judgment. This essay traces that paradox across traditions: biblical justice and Buddhist causality; Western philosophy’s metaphysics of will; Conrad’s colonial modernity refracted through Coppola; and modern psychology’s confrontation with mortality. Willard’s line, stripped of theology but charged with fatalism, speaks for the modern self: one who always gets what he wants—and must live with what it means.

\printbibliography
\end{document}
