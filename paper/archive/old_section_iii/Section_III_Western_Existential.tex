
\documentclass[12pt]{article}
\usepackage[margin=1in]{geometry}
\usepackage{setspace}
\usepackage{parskip}
\usepackage{csquotes}
\usepackage{hyperref}
\usepackage{fontspec} %% XeLaTeX or LuaLaTeX
\usepackage[style=apa,backend=biber]{biblatex}
\addbibresource{everyone.bib}

\setmainfont{Libertinus Serif}
\setsansfont{Libertinus Sans}
\setmonofont{Libertinus Mono}
\defaultfontfeatures{Ligatures=TeX}
\linespread{1.5}

\title{“Everyone Gets Everything He Wants”: Desire, Fulfillment, and the Tragic Logic of Will in \textit{Apocalypse Now}}
\author{(Your Name)}
\date{\today}

\begin{document}
\maketitle

\section*{III. Western Philosophy and Existentialism: Fulfillment as Exposure of Freedom (Revised, Integrated)}

If the Biblical and Buddhist traditions read fulfillment as moral or causal disclosure, modern Western philosophy reads it as the exposure of a freedom that cannot complete itself. From Schopenhauer to Levinas, getting what one wants reveals that what one wanted was either structurally unsatisfiable or ethically misconceived. Willard’s sentence---``Everyone gets everything he wants. I wanted a mission, and for my sins they gave me one''---thus becomes a vernacular précis: fulfillment is \emph{the moment fr...
\subsection*{1. Lack, Power, Law: Schopenhauer, Nietzsche, Kant}

Schopenhauer diagnoses willing as lack: ``All willing springs from lack, from deficiency, and therefore from suffering'' (\parencite{SchopenhauerWWR1969}, p.~196). Satisfaction cannot end willing; ``every satisfied desire at once makes room for a new one'' (\parencite{SchopenhauerWWR1969}, p.~319). Fulfillment punishes by revealing the will’s \emph{insatiable structure}. In cinematic terms, the river’s episodic ``missions'' replicate this grammar: completion does not resolve; it resets.

Nietzsche overturns renunciation by affirming the will. Better to will nothingness than not will at all (\parencite[III.28, p.~162]{NietzscheGenealogy1994}); fulfillment becomes self-overcoming. Yet the paradox persists: the will requires ever-new objects to remain itself. ``The desire for `truth' has hitherto been the most dangerous of all possessions'' (\parencite[\S 34]{NietzscheBGE1990}). When Willard accepts the assignment, the \emph{form} of willing is affirmed; the content (kill Kurtz) is merely fres...
Kant separates moral worth from happiness: ``only the good will is good without qualification'' (\parencite{KantCPrR1996}, p.~27). Fulfillment of inclination does not confer moral status; law does. In that sense, the mission’s successful completion has no inherent ethical weight. If anything, the film’s bureaucratic legality accentuates the gap between \emph{legality} and \emph{morality}.

\subsection*{2. Despair and Narrative Selfhood: Kierkegaard and Dostoevsky}

Kierkegaard understands the self as a relation ``that relates itself to itself'' and can therefore be in despair (\parencite[pp.~49--52]{KierkegaardSUD1980}). To will oneself apart from God---to absolutize a contingent project---is to experience fulfillment as self-loss. Willard’s ``sins'' can be read as precisely this absolutization: the mission is taken as identity. The result is the melancholic thinning that follows each tactical ``win.''

Dostoevsky’s Underground Man insists: ``Man only wants independent desire, whatever that independence may cost'' (\parencite{DostoevskyNFU1994}, p.~131). The value is not in the object attained but in asserting agency against mechanism: ``To hell with two times two makes four!'' (\parencite{DostoevskyNFU1994}, p.~129). The film’s recurrent refusals of sense---pressing upriver through absurdity---match this existential appetite for agency even at the expense of well-being.

\subsection*{3. Condemnation to Project: Sartre, Beauvoir, Camus}

Sartre’s formula is canonical: ``Man is condemned to be free'' (\parencite{SartreBN2003}, p.~34). Freedom is not a gift but an inescapable task; each act forges essence without guarantee. The ``fulfilled'' desire is illusory insofar as the project aims at a God-like synthesis of facticity and transcendence that cannot be achieved (\parencite[p.~604]{SartreBN2003}). Willard’s desire for a mission thus yields condemnation: he must own the act that reveals what he is.

Beauvoir converts ontology into ethics: to will oneself free entails willing others free (\parencite{Beauvoir1976}, p.~73). Fulfillment pursued as solitary sovereignty degenerates into tyranny. This sharpens the moral critique of the assignment: a mission executed without reciprocity becomes inherently suspect, whatever its ``success.''

For Camus, the absurd arises when human need meets the world’s silence (\parencite[p.~28]{CamusMyth1991}). Fulfillment is impossible because meaning is not given; yet rebellion---lucid action without appeal---remains. To ``imagine Sisyphus happy'' (\parencite[p.~123]{CamusMyth1991}) is to accept the project as project. The film’s final sequences show a similar lucidity: completion without consolation.

\subsection*{4. Being-toward-Death and the Face of the Other: Heidegger and Levinas}

Heidegger individualizes freedom through finitude. Authenticity requires ``anticipatory resoluteness'' toward death (\parencite[p.~307]{HeideggerBT1962}); ``as soon as man comes to life, he is at once old enough to die'' (\parencite[p.~298]{HeideggerBT1962}). Fulfillment cannot complete Dasein because death discloses the structural impossibility of completion. The river journey’s temporality---a string of ``nows'' without totalizing narrative---is phenomenologically apt for this exposure.

Levinas, by contrast, makes the Other the site of transcendence. ``Desire is desire for the absolutely other'' (\parencite[p.~33]{LevinasTI1969}); the face ``forbids us to kill'' (\parencite[p.~199]{LevinasTI1969}). Here the punishment of fulfilled desire is ethical: a will that seeks completion in its own project discovers itself as violence when it meets the command of the Other. The film’s late confrontations---with captives, with Kurtz---puncture the mission’s self-sufficiency and reveal a demand that the...
\subsection*{5. Synthesis: What Fulfillment Reveals}

Across these positions, ``getting what one wants'' functions as an \emph{experiment} in self-revelation:

\begin{itemize}
	\item For Schopenhauer, it reveals \emph{insatiability}; for Nietzsche, the need for \emph{ever-new self-overcoming}.
	\item For Kant, the gap between \emph{inclination} and \emph{moral law}.
	\item For Kierkegaard, the danger of \emph{despair} through absolutized projects.
	\item For Sartre/Beauvoir, the \emph{condemnation} of freedom and the ethical necessity of \emph{reciprocity}.
	\item For Camus, the inevitability of \emph{lucid rebellion}.
	\item For Heidegger, the \emph{finitude} that voids completion; for Levinas, the \emph{infinite responsibility} that punctures solipsistic ends.
\end{itemize}

These are not identical; they are complementary exposures of the same knot: will, world, and other. Integrated back into Willard’s line: \emph{Everyone} (universality of structure) \emph{gets} (experiment of action) \emph{everything} (totalizing aims) he \emph{wants} (projected freedom). ``For my sins I got one'' translates into: the project fulfilled displayed, without alibi, what my willing had been all along.

\printbibliography
\end{document}
