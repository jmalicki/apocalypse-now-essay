\documentclass[12pt]{article}
\usepackage[margin=1in]{geometry}
\usepackage{setspace}
\usepackage{parskip}
\usepackage{csquotes}
\usepackage{hyperref}
\usepackage{fontspec} %% XeLaTeX or LuaLaTeX
\usepackage[style=apa,backend=biber]{biblatex}
\addbibresource{everyone.bib}

\setmainfont{Libertinus Serif}
\setsansfont{Libertinus Sans}
\setmonofont{Libertinus Mono}
\defaultfontfeatures{Ligatures=TeX}
\linespread{1.5}

\title{“Everyone Gets Everything He Wants”: Desire, Fulfillment, and the Tragic Logic of Will in \textit{Apocalypse Now}}
\author{(Your Name)}
\date{\today}

\begin{document}
\maketitle

\section*{V. Modern Psychology and Death: Desire, Symbolism, and the Shadow}

If Section III located fulfillment’s burden in the ontology of freedom, modern depth psychology shows why fulfillment so often feels like punishment: the psyche seeks symbolic mastery of death. From Freud’s death drive to Lacan’s endless deferral of satisfaction, from Jung’s shadow to Becker’s hero-systems, modern theory repeatedly argues that desire is a defensive formation against mortality. Willard’s admission---``I wanted a mission, and for my sins they gave me one''---thus reads as a clinical precis...
\subsection*{1. Freud: Repetition, Drive, and Discontent}
In \textit{Beyond the Pleasure Principle}, Freud confronts the anomaly of compulsive repetition. Rather than seeking pleasure, the neurotic repeats the very experience that wounds him; a ``daemonic'' character of repetition ``overrides the pleasure principle'' \parencite[p.~22]{FreudBeyond1955}. Freud hypothesizes a ``death drive'' that ``endeavours to lead organic life back into the inanimate state'' \parencite[p.~38]{FreudBeyond1955}. Fulfillment here cannot satisfy because the drive is not oriented to pl...
Three decades later, \textit{Civilization and Its Discontents} reframes the contradiction socially: culture requires renunciation, and the price of justice is repression \parencite{FreudCivilization1961}. The subject must sublimate instinct into work and love, but the leftover aggression returns as guilt, which ``represents the most important problem in the development of civilization'' \parencite[p.~97]{FreudCivilization1961}. In Willard’s case, the ``mission'' becomes both a vehicle of sublimation and a ...
\subsection*{2. Rank and Fromm: Will, Escape, and Character}
Rank’s late work recovers will as a positive, creative power rather than merely a symptom of repression. ``The will to create is the will to become'' \parencite[p.~xx]{RankWill1978}; yet creativity is haunted by separation anxiety. The artist and the hero attempt to ``birth'' themselves symbolically, staging separations that turn passivity into authorship. This clarifies why missions appeal: they promise individuation through action. Fromm, by contrast, reads modern authoritarianism as a characterologica...
\subsection*{3. Becker: Immortality Projects and the Terror of Death}
Becker synthesizes psychoanalysis, anthropology, and existentialism into a single claim: culture is a ``hero-system'' that denies death \parencite{BeckerDenial1973}. ``The irony of man’s condition is that the deepest need is to be free of the anxiety of death and annihilation; but it is life itself which awakens it'' \parencite[p.~66]{BeckerDenial1973}. We seek symbolic immortality through achievement, love, or nation; we want missions capable of insulating us from finitude. Hence fulfillment feels like pu...
\subsection*{4. Jung: Shadow, Persona, and the Coniunctio}
Jung reframes the problem as one of psychic integration. The persona secures social recognition; the shadow contains repressed traits; the Self symbolizes wholeness \parencite{JungArchetypes1969}. Fulfillment constellates the shadow: when we gain what we consciously want, we also summon disowned potentials. Individuation therefore requires confrontation. In \textit{Aion} Jung argues that the ego’s inflation---its identification with the Self---generates moral catastrophe; the only cure is dialectical recon...
\subsection*{5. Lacan: Desire Beyond Demand}
For Lacan, desire is not biological appetite but a structural effect of language: it is ``the desire of the Other'' \parencite{LacanEcrits2006}. Demand can be satisfied; desire cannot. The object-cause of desire---\textit{objet petit a}---functions as a remainder that ``is never the object of need or demand'' \parencite[p.~103]{LacanSeminarXI1991}. Fulfillment is punishment because the subject receives the demanded object only to rediscover that desire persists as lack. The mirror stage shows how ego-ide...
\subsection*{6. Frankl: Meaning as Antidote---and Limit}
As a corrective to psychopathology, Frankl’s logotherapy proposes meaning as the human being’s primary motivation: ``Those who have a 'why' to live can bear with almost any 'how' '' \parencite[p.~104]{FranklMeaning2006}. Yet even Frankl cautions that meaning cannot be \textit{willed} directly; it must be found as a by-product of commitment and love. In Willard’s case, the mission furnishes a ``how'' without a ``why''---a structure of action without a transcendent end---and thus cannot supply redemption. Me...
\subsection*{7. Synthesis: The Mission as Psychological Symptom}
Across these approaches, fulfillment functions as a diagnostic device. Psychoanalysis reads the ``mission'' as repetition and displacement; Rank and Becker interpret it as a defense against death; Jung and Lacan see in it the activation of shadow and the return of lack; Frankl reframes it as a failure of teleology. Willard receives exactly what he wants and thereby learns what he is. The diagnosis is severe but not hopeless: the only exit these traditions allow is not more fulfillment but a different rela...

\printbibliography
\end{document}
