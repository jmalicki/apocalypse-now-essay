\phantomsection
\section*{II. Conrad: Heart of Darkness as Source}
\addcontentsline{toc}{section}{II. Conrad: Heart of Darkness}
\label{sec:ii-conrad}

Joseph Conrad's \textit{Heart of Darkness} (1899) provides the structural template for 
\textit{Apocalypse Now}: a journey upriver toward a mysterious figure whose fulfillment reveals 
horror rather than resolution. Both narratives share a fundamental pattern: the will gets what 
it wants---knowledge, empire, recognition---and what it wants unmasks the will itself.

\subsection*{Conrad's Modernity: Illumination and Horror}
\label{ssec:conrad-s-modernity-illumination-and-horror}
Conrad’s novella organizes colonial conquest as an epistemological allegory: the voyage upriver promises knowledge, but the attainment of knowledge reveals only the vacuum at its core. In Marlow’s opening demystification of imperial rhetoric, conquest is ``robbery with violence, aggravated murder on a great scale'' \parencite[ConradHOD1990]{ConradHOD1990}. Kurtz’s dying judgment---``The horror! The horror!''---is the paradoxical consummation of his civilizing project: he gets everything he wants (ivory, absolute command, the image of European virtue) and discovers that desire fulfilled negates the desiring self \parencite{ConradHOD1990}. The novella’s formal strategy---a frame narrative in which the tale loops back on itself---mirrors this structure of return: fulfillment is not progress but recursion.
