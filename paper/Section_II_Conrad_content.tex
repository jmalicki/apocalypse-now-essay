\phantomsection
\section*{II. Conrad: Heart of Darkness as Source}
\addcontentsline{toc}{section}{II. Conrad: Heart of Darkness}
\label{sec:ii-conrad}

\textit{Heart of Darkness} \parencite{ConradHOD1990} provides the structural template for
\textit{Apocalypse Now}: a journey upriver toward a mysterious figure whose fulfillment reveals
horror rather than resolution. Both narratives share a fundamental
pattern: the will gets what it wants---knowledge, empire, recognition---and what it wants
unmasks the will itself.

\phantomsection
\subsection*{The Shared Allegory: From the Congo to Vietnam}
\addcontentsline{toc}{subsection}{The Shared Allegory}
\label{ssec:the-shared-allegory}
Coppola transposes Conrad's tale from the Congo Free State to the Vietnam War, but preserves
the allegorical architecture. Marlow becomes Captain Willard; the Company becomes the U.S.
military; Kurtz remains Kurtz. In both, the protagonist is sent upriver to confront a figure
who has ``gone native'' and abandoned institutional norms, yet represents the logical
conclusion of those norms' inner violence. Marlow travels by steamboat through Belgian colonial
stations; Willard by patrol boat through American firebase chaos. Both journeys move through
escalating absurdity—administrative pretense dissolving into arbitrary brutality—toward a
figure who has dispensed with pretense altogether.

The parallels extend to narrative structure. Both protagonists narrate retrospectively, already
knowing what they will find but unable to forestall it. Both are sent with orders to
``terminate'' or retrieve, yet the mission's clarity dissolves as the journey reveals that
Kurtz's ``methods'' merely make explicit what the empire practices covertly. Willard's
voiceover, like Marlow's frame tale, establishes ironic distance: the teller knows the mission
cannot deliver what it promises, yet he completes it anyway. The form itself enacts the
structure of fulfillment-as-exposure.

\phantomsection
\subsection*{Conrad's Modernity: Illumination and Horror}
\addcontentsline{toc}{subsection}{Conrad's Modernity}
\label{ssec:conrad-s-modernity-illumination-and-horror}
Conrad's novella organizes colonial conquest as an epistemological allegory: the voyage upriver
promises knowledge—knowledge of Kurtz, of Africa, of empire's truth—but the attainment of
knowledge reveals only the vacuum at its core. Marlow's opening demystification of imperial
rhetoric frames this from the start: conquest is ``robbery with violence, aggravated murder on
a great scale'' \parencite{ConradHOD1990}. The novella exposes how the language of
``civilization,'' ``trade,'' and ``light'' masks extractive domination. Kurtz, the exemplary
agent of progress, writes a report advocating ``Exterminate all the brutes!''—the id of empire
speaking without superego \parencite{ConradHOD1990}.

Kurtz's dying judgment---``The horror! The horror!''---is the paradoxical consummation of his
civilizing project: he gets everything he wants (ivory, absolute command, the image of European
virtue) and discovers that desire fulfilled negates the desiring self \parencite{ConradHOD1990}.
What he sees at the end is not moral failure but ontological exposure: the will's object was
domination all along, and attaining it strips away the rationalizations that made willing
bearable. The novella's formal strategy---a frame narrative in which the tale loops back on
itself, Marlow returning to tell what cannot be told—mirrors this structure of return:
fulfillment is not progress but recursion. One cannot ``un-know'' what the journey reveals.

Coppola preserves this structure in Willard's arc. The \hyperref[scene:briefing]{dossier}
promises clarity: Kurtz is an aberration, a problem to be solved. The journey upstream reveals
that Kurtz is the empire's truth-teller, the figure who refused to lie about what he wanted.
When \hyperref[scene:assassination]{Willard completes the mission}—gets what he wanted—he
inherits Kurtz's knowledge without inheriting a script for what
to do with it. Both Conrad and Coppola locate the tragedy not in failure but in success:
getting what one wants reveals that the wanting itself was the problem.
