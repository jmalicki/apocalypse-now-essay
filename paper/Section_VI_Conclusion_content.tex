\phantomsection
\section*{VI. Conclusion: The Mission and the Mirror}
\addcontentsline{toc}{section}{VI. Conclusion: The Mission and the Mirror}
\label{sec:vi-conclusion-the-mission-and-the-mirror}

Across theology, Buddhist doctrine, modern philosophy, postcolonial critique, and depth psychology, a single structure recurs: fulfillment is disclosure. To ``get what one wants'' is to discover what one’s wanting already was. In biblical terms, the sinner is given over to his desire; in Buddhist terms, craving reproduces the causes of suffering; in existential terms, freedom shows itself as obligation without guarantor; in colonial modernity, the will returns as an administrative function; in the psyc...
\subsection*{1. Hermeneutics of Guilt and the Work of Understanding}
\label{ssec:1-hermeneutics-of-guilt-and-the-work-of-understanding}
Ricoeur’s ``hermeneutics of suspicion'' does not end in exposure alone; it moves toward a ``second naivet{\'e}'' in which symbols are received again through interpretation \parencite{RicoeurSymbol1970}. The point of reading Willard’s line through Augustine, the Nik{\={a}}yas, Schopenhauer, Conrad, or Freud is not merely to diagnose complicity; it is to refashion understanding so that desire can be taken up differently. Symbols ``give rise to thought,'' Ricoeur insists (p.~347); they also give rise to will ...
\subsection*{2. Desire and Ideology}
\label{ssec:2-desire-and-ideology}
{\v{Z}}i{\v{z}}ek’s Lacanian account of ideology clarifies how ``everyone gets everything he wants'' can be true at the level of fantasy while remaining false in life. Ideology, he argues, is not what we consciously believe but what structures our desiring \parencite{ZizekSublime1999}. The subject ``knows very well'' that fulfillment is impossible, ``but still'' he acts as if the mission would complete him (p.~32). In Willard’s world, the mission-form functions as this fantasy-support: a script that dema...
\subsection*{3. Toward a Discipline of Wanting}
\label{ssec:3-toward-a-discipline-of-wanting}
The traditions surveyed here do not counsel quietism but discipline. The biblical path names it charity and obedience; the Buddhist path, the Noble Eightfold Path; Kant, duty; Beauvoir, reciprocity; Levinas, responsibility for the face of the Other; Fanon, the struggle for recognition without annihilation. Each proposes that the solution to fulfilled desire is not more desire, nor its annihilation, but the \emph{re-formation} of willing: detachment from possession, attachment to responsibility. The altern...
\subsection*{4. Last Words}
\label{ssec:4-last-words}
Poetry often says last things best. Eliot’s reflection that ``the end of all our exploring / Will be to arrive where we started / And know the place for the first time'' \parencite[Eliot, 1969, ``Little Gidding'']{EliotCollected1969} names the arc of this essay: the mission returns us to the origin of will. Rilke’s admonition---``You must change your life''---arrives in another register \parencite[Snow trans., 2009]{RilkeElegies2009}. If Willard’s sentence is the mission’s epitaph, this is its coda. The...
