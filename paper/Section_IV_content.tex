\phantomsection
\section*{IV. Western Philosophy: From Enlightenment Through Idealism to Existentialism}
\addcontentsline{toc}{section}{IV. Western Philosophy}
\label{sec:iv-western-philosophy}

Modern Western philosophy reads Willard's line not as moral judgment or causal sequence but as
an experiment in self-disclosure: when desire is granted its object, what is revealed about the
freedom that desired it? From the Enlightenment's architectonics of will (Kant) through German
Idealism's dialectics of recognition (Schopenhauer, Hegel) to existentialism and phenomenology's
accounts of project, despair, and finitude (Kierkegaard, Dostoevsky, Sartre, Beauvoir, Camus,
Heidegger, Levinas), each thinker supplies a distinct optic through which ``getting what one
wants'' becomes punitive—not because satisfaction is withheld, but because it exposes a
structure (metaphysical, existential, ethical) that the will misconceived or refused. The
following twelve analyses trace that exposure across the tradition.

\phantomsection
\subsection*{Kant: Duty, Autonomy, and Why ``Success'' Proves Nothing}
\addcontentsline{toc}{subsection}{Kant: Duty, Autonomy, and Success}
\label{ssec:iii-kant}
Immanuel Kant gives the sharpest rebuke to reading fulfillment as vindication. In his moral
philosophy, the worth of an action lies not in what it achieves but in the maxim from which
it is done—the
principle the agent could will as universal law. The line ``Everyone gets everything he wants''
therefore cannot count as evidence that wanting was justified; ``for my sins I got one'' sounds
like the dawning recognition that having one's ends granted can lay bare a prior failure of
duty.

\phantomsection
\subsubsection*{The Categorical Imperative and the Humanity Formula}
\addcontentsline{toc}{subsubsection}{The Categorical Imperative}

Kant's baseline claim is that the good will is ``good \ldots\ in itself,'' not by the
advantages it produces \parencite[p.~27]{KantCPrR1996}. This relocates ethical assessment away
from effects (which are subject to luck, power, and circumstance) to the will's legislation of
its own maxim. When Willard accepts the assignment in the cool, procedural light of
\hyperref[scene:briefing]{Saigon}, the scene supplies everything success-friendly ethics
likes—clarity of ends, chain of command,
legal sanction. For Kant, none of that matters morally. The question is simple and brutal:
What maxim am I adopting, and can I will it as a law for all rational agents? If the maxim is
``Eliminate as a means any person my institution designates an obstacle,'' universalization
collapses into contradiction: it destroys the very conditions of mutual recognition that a law
for all would require. The mission can be ``successful'' and still be morally void.

This is the force of Kant's second test—the humanity constraint, expressed (across his corpus)
as treating humanity, in oneself and others, always as an end and never merely as a means.
\emph{Critique of Practical Reason} articulates the same structure when it insists that the
moral law addresses us as free and self-legislating, never as mere instruments of inclination
or authority \parencite[pp.~30--33]{KantCPrR1996}. Transpose this into the film's grammar:
a mission-form that disables reciprocity and reduces persons to objects of procedure cannot be
rescued by neat outcomes. Willard's acknowledgment that it was a ``real choice mission''
intensifies the Kantian judgment---he cannot hide behind heteronomy or claim he was merely
following orders. The will freely adopted the maxim, making the subsequent exposure of its
failure absolute. ``Everyone gets everything he wants'' becomes, under Kant, not an excuse but
an indictment: autonomous choice revealed its own wrong orientation.

\phantomsection
\subsubsection*{Autonomy vs. Heteronomy: Incentive, Not Just Compliance}
\addcontentsline{toc}{subsubsection}{Autonomy vs. Heteronomy}

Kant's distinction between legality and morality intensifies this. An act can conform to the
law outwardly (legality) while lacking the right incentive (morality). What makes an action
moral is that its determining ground is respect for the moral law, not fear, habit, or advantage
\parencite[pp.~72--76]{KantCPrR1996}. The \hyperref[scene:briefing]{dossier scene} is a study
in outward conformity: orders, signatures, the rhetoric of necessity. But the incentive that
animates ``I wanted a mission'' is not respect for law; it is a desire for orientation, relief
from aimlessness,
and ultimately institutional recognition. When the mission is granted, fulfillment exposes
the incentive: instead of being moved by duty, the will was moved by a need to still its own
drift. ``For my sins I got one'' now reads as Kantian confession: I acted from heteronomy,
and success only made that visible.

\phantomsection
\subsubsection*{Why Success Proves Nothing: Effects vs. Maxims}
\addcontentsline{toc}{subsubsection}{Why Success Proves Nothing}

Kant's moral psychology helps clarify why fulfillment can feel like punishment. Respect for
the law is an incentive that humbles self-love; it is experienced as a constraint on inclination
\parencite[pp.~70--73]{KantCPrR1996}. To the extent that the film's assignments cloak
inclination under moral language—security, order, ``surgical'' necessity—the later ``success''
functions as a de-masker: the will discovers it was not obeying a law it could legislate for
all, but rather serving a maxim it would never publicly endorse as universal. The tight,
affectless tone of Willard's narration after each ``win'' matches this discovery: the more
procedure works, the clearer it becomes that working isn't the same as willing rightly.

Kant's emphasis on autonomy sharpens the point. To be free is not to get what one wants, but
to give oneself a law that any rational agent could adopt \parencite[pp.~30--33]{KantCPrR1996}.
Measured this way, the mission-form is structurally tempting to heteronomy: it outsources
lawgiving to the institution and treats persons encountered en route as mere bearers of protocol.
Even when the mission targets someone like Kurtz—himself a violator of reciprocity—the maxim
``neutralize by assassination when the institution decrees'' cannot be a law for all, because
it erodes the very standing of rational agents as co-legislators. The fact that ``everyone gets
everything he wants'' in such a system is precisely the problem: it signals the reliable
availability of means for heteronomous ends.

Kant also insists that morality is not a ledger of effects but an orientation of maxims
sustained through adversity. This explains why the film's most chilling moments are not its
brutalities but its efficiencies: when the \hyperref[scene:sampan]{sampan search} is executed
by the book, the clean line from maxim to act to outcome throws the wrong maxim into relief.
Even if the damage were minimized, the principle—instrumentalization under orders—would still
fail the humanity
constraint. Fulfillment punishes because it removes the excuse of friction: the will must own
what it willed when everything ``worked.''

Does Kant leave any room for the line's first half—``Everyone gets everything he wants''—to
carry moral weight? Only in a highly restricted sense. If the ``want'' is already shaped by
the moral law—if the agent wants to act from a universalizable maxim out of respect for
persons—then ``getting what one wants'' is just the possibility to do one's duty. Otherwise,
success is morally insignificant at best and accusatory at worst. In context, the confession
``for my sins I got one'' catches this: the grant itself is the mirror that shows the will's
prior choice against autonomy.

Finally, Kant's idea of the highest good (happiness proportioned to virtue) underscores the
tragedy: the world does not guarantee any convergence between success and moral worth
\parencite[pp.~125--131]{KantCPrR1996}. The mission can be fully accomplished and still fail
the test of law; conversely, refusal might align with duty but bring ruin. This is why a
Kantian reading refuses consolation at the end: what matters is not that the project ended,
but whether the maxim survives scrutiny. By that light, ``Everyone gets everything he wants''
names a morally irrelevant fact about means and outcomes; ``\ldots and for my sins I got one''
names the relevant fact about the will that chose the maxim it did.

\pagebreak[2]
\phantomsection
\subsection*{Schopenhauer: Fulfillment as Disclosure of Lack}
\addcontentsline{toc}{subsection}{Schopenhauer: Fulfillment as Disclosure of Lack}
\label{ssec:iii-schopenhauer}
Arthur Schopenhauer's analysis of will offers a rigorous grammar for Willard's confession:
``Everyone gets everything he wants. I wanted a mission, and for my sins they gave me one.''
For Schopenhauer, desire is not a teleology that culminates in peace but a mechanism whose
very satisfaction resets itself. Hence the mission granted is not a gift that stills the heart;
it is the next oscillation of a structure that cannot be stilled.

\phantomsection
\subsubsection*{1) The structure of willing: lack $\rightarrow$ striving
	$\rightarrow$ relief $\rightarrow$ renewed lack}
\addcontentsline{toc}{subsubsection}{The structure of willing}

``All willing springs from lack, from deficiency, and therefore from suffering''
\parencite[p.~196]{SchopenhauerWWR1969}. The object that seems to promise rest is already
implicated in the will's unease; when attained, it ``at once makes room for a new one''
\parencite[p.~319]{SchopenhauerWWR1969}. Schopenhauer's famed image is diagnostic rather
than rhetorical: life ``swings like a pendulum, to and fro between pain and boredom''
\parencite[p.~312]{SchopenhauerWWR1969}.

Read against the film's first movements, the pattern holds precisely. In
\hyperref[scene:saigon-opening]{Saigon}, Willard's lack is staged as agitation and
intoxication; the \hyperref[scene:briefing]{dossier scene} supplies an object (the mission)
and a narrative. Relief appears as orientation—a reason to move upriver—but immediately becomes
renewed lack: each checkpoint demands the next, each ``win'' opens a further deficit.
Willard's voiceover keeps the pendulum audible: completion never completes; it only
re-initiates striving.

\phantomsection
\subsubsection*{2) Why fulfillment punishes: the object is a delusion of rest}
\addcontentsline{toc}{subsubsection}{Why fulfillment punishes}

Schopenhauer emphasizes that satisfaction exposes, rather than heals, the structure of desire.
Enjoyment ``we have longed for'' soon leaves us ``bored,'' and the will ``returns to its old
course'' \parencite[p.~319]{SchopenhauerWWR1969}. In that sense, getting what one wanted hurts
because it removes the fantasy that the object could silence the will. The hurt is cognitive:
fulfillment disenchants.

The film thematizes this in miniature. The \hyperref[scene:playboy-show]{Playboy show}
promises heightened pleasure; the immediate after-image is agitation and bargaining.
\hyperref[scene:kilgore-beach]{Kilgore's beachhead} produces tactical success, but the
spectacle (``I love the smell of napalm in the morning'') converts victory into appetite. The
\hyperref[scene:sampan]{sampan search} yields compliance, then horror; the ``completed''
procedure reveals leanness of soul. In Schopenhauer's terms, each fulfilled want punctures its own
promissory aura and re-installs the need to want.

\phantomsection
\subsubsection*{3) Representation, will, and the aesthetic \& ethical brakes that fail in war}
\addcontentsline{toc}{subsubsection}{Representation and brakes that fail}

Schopenhauer distinguishes the world as representation (ordered by the principle of sufficient
reason) from the world as will (blind striving) \parencite[pp.~3--5]{SchopenhauerWWR1969}.
Two brakes can mitigate the will's tyranny. First, aesthetic contemplation suspends willing by
fixing consciousness on the Idea—to ``lose oneself in the object''
\parencite[p.~178]{SchopenhauerWWR1969}. Second, compassion reframes the other not as instrument
but as a fellow bearer of suffering \parencite[pp.~372--374]{SchopenhauerWWR1969}.
Both brakes fail in the film's wartime economy. Wagner's ``Ride of the Valkyries'' is mobilized
as a stimulant for domination, not as will-suspending contemplation; the sampan protocol
subordinates pity to procedure. The very mechanisms that could have cooled willing are
conscripted by it.

\phantomsection
\subsubsection*{4) Boredom, repetition, and the river as pendulum}
\addcontentsline{toc}{subsubsection}{Boredom and repetition}

If pain signals unfulfilled desire, boredom signals desire's deflation after satisfaction.
Schopenhauer's claim that joy fades into ennui is not a counsel of mood but an analysis
of the will's metabolism \parencite[pp.~312--320]{SchopenhauerWWR1969}. The river sequences
enact this metabolism: periods of frantic danger (pain) alternate with slack stretches of
waiting (incipient boredom), and Willard's narration re-ignites the need for the next trial.
The \hyperref[scene:do-lung-bridge]{Do Lung Bridge} sequence literalizes the pendulum:
building by day, destruction by night; every ``achievement'' immediately generates its
contrary.

\phantomsection
\subsubsection*{5) ``For my sins I got one'': the inner necessity of punishment}
\addcontentsline{toc}{subsubsection}{The inner necessity of punishment}

Schopenhauer does not require an external punisher. The punishment is inner: to ``get what one
wants'' is to have the will's insatiability revealed to oneself. Hence the tone of Willard's
clause; the ``sin'' is not only moral guilt but attachment to the fantasy that a mission could
deliver more than recurrence. The gift is therefore a judgment.

At \hyperref[scene:kurtz-compound]{Kurtz's compound}, this inner necessity is complete. Kurtz
has arranged the conditions for maximal satisfaction (command, removal of obstacles) and finds
that mastery produces only a
clarified view of the will's void: possession does not pacify. Willard's approach—labor, danger,
deprivation—does not redeem the will, but it strips away the last illusions about what
fulfillment can do. In Schopenhauer's lexicon, the world is disclosed as will precisely when
desire succeeds.

\phantomsection
\subsubsection*{6) Objection \& counterpoint: is there any release?}
\addcontentsline{toc}{subsubsection}{Is there any release?}

One might object that Schopenhauer allows two releases. First, aesthetic states offer
``deliverance'' from the will's press \parencite[p.~178]{SchopenhauerWWR1969}. Second,
compassion grounds ethics beyond egoistic striving \parencite[pp.~372--374]{SchopenhauerWWR1969}.
The film acknowledges both in negative: music becomes a tool for domination rather than
contemplation; pity is subordinated to protocol. The point is not that release is metaphysically
impossible, but that this narrative world is structured to block it; thus, fulfillment returns
as exposure.

\phantomsection
\subsubsection*{7) Payoff for the thesis}
\addcontentsline{toc}{subsubsection}{Payoff for the thesis}

Schopenhauer thus illuminates the first half of Willard's sentence: \emph{everyone} (because
willing is universal) \emph{gets} (because objects are available) \emph{everything} (because
the will projects ``the all'' onto finite objects) he \emph{wants} (because wanting, not the
wanted, is fundamental). The second half—``for my sins I got one''—expresses the cognate of this
metaphysics: punishment is not denial but grant that unmasks the will's structure. The mission
is not a deviation from desire's grammar; it is the very form in which that grammar is made
visible.

Willard's claim that afterward he ``never wanted another'' might seem to contradict
Schopenhauer's pendulum model, which predicts satisfaction immediately generates new lack. But
Schopenhauer allows one reading: the will can be stilled through total disillusionment. Not
aesthetic contemplation or compassion, but the exhaustive exposure of every object's
emptiness. If so, ``never wanted another'' is not liberation but the death of willing
itself---a grim confirmation of Schopenhauer's diagnosis that the will's only true peace is
its negation.

\pagebreak[2]
\phantomsection
\subsection*{IV.3—Hegel: From Object-Desire to Recognition, Mastery's Emptiness, and the Truth of
	Work}
\addcontentsline{toc}{subsection}{IV.3—Hegel: Recognition and Mastery's Emptiness}
\label{ssec:iii-hegel}
G.W.F. Hegel's decisive move is to show why fulfillment through possession cannot settle
desire. In the \emph{Phenomenology of Spirit}, self-consciousness first appears as desire that
negates
otherness, but it learns that consuming things can never yield self-certainty:
``self-consciousness achieves its satisfaction only in another self-consciousness''
\parencite[\S 175]{HegelPhenomenology1977}. The thing I devour does not look back; it cannot
recognize me. If ``Everyone gets everything he wants'' is read as a promise of objects and
outcomes, Hegel's rejoinder is that objects are the wrong currency for the desire at stake.
The later clause—``\ldots for my sins I got one''—sounds like the experience of having obtained
the wrong coin.

Hegel dramatizes this transition in the struggle for recognition culminating in lordship and
bondage \parencite[\S\S 178--196]{HegelPhenomenology1977}. The combatants risk death because
only a being who risks its life shows that it is not bound to bare preservation. The so-called
Lord ``wins,'' but his victory is hollow: the Bondsman's submission is not free recognition
\parencite[\S\S 187--189]{HegelPhenomenology1977}. Mastery therefore ``gets what it
wants''—dominion—and
finds it empty of the very confirmation it sought. This emptiness is not psychological
disappointment; it is structural. Recognition that counts must be mutual between free subjects.
Where a project's logic—administrative or militarized—reduces others to functions, the more
perfectly it attains its end, the more sharply its lack of recognition appears.

The truth of self-consciousness, Hegel says, lies not with the Lord but with the Bondsman, who,
through fear, service, and formative work (\emph{Bildung}), mediates self and world
\parencite[\S 196]{HegelPhenomenology1977}. Work transforms the given without annihilating it;
it commits the self to a shared, durable world. The river journey's serial procedures—secure a
beach, clear a waterway, enforce a protocol—have the outer form of work, yet the world they
leave is not stabilized as a space of mutual recognition. The cycle at the bridge—construction
by day, erasure by night—parodies \emph{Bildung}: it produces, but it does not found.
Fulfillment here punishes by revealing the absence of the only recognition that could have
satisfied the desire that set the project in motion.

Hegel's dialectic also clarifies why the most ``efficient'' victories feel airless. Dominating
the other as instrument silences the very freedom from which recognition must come. Each success,
then, intensifies the contradiction: the more complete the procedure, the more total the other's
silencing, and the less possible the confirmation the agent craves. ``Everyone gets everything
he wants'' becomes tragic because the want was mis-specified: it sought certitude about self
through the mute success of operations. ``\ldots For my sins I got one'' is the moment mastery
confesses its own null confirmation.

\pagebreak[2]
\phantomsection
\subsection*{Nietzsche: Fulfillment as Style of Will—Affirmation, Domination, Transvaluation}
\addcontentsline{toc}{subsection}{Nietzsche: Fulfillment as Style of Will}
\label{ssec:iii-nietzsche}
Friedrich Nietzsche does not dispute Schopenhauer's observation that willing does not rest; he
revalues it. The problem is not desire's recurrence but our craving for a terminal perch that
would end
the need to will. In this light, ``Everyone gets everything he wants'' becomes a diagnostic:
fulfillment reveals whether the will has the style to affirm its own recurrence, or whether it
smuggles domination in under the names of truth and duty. ``For my sins I got one'' is the
moment the mask of those names slips.

Nietzsche's stark claim—``man would rather will nothingness than not will at all''—positions
the refusal of willing as more intolerable to life than suffering
\parencite[III.28, p.~162]{NietzscheGenealogy1994}. The danger, then, is not the intensity of
volition but its self-deception: the will valorizes itself as ``truth'' so it can command
without admitting it. ``The desire for `truth' has hitherto been the most dangerous of all
possessions'' because it disguises a need to impose \parencite[\S 34]{NietzscheBGE1990}.
The Saigon briefing's cool rationality—dossiers, maps, a narrative of ``removing an
aberration''—is exemplary of this danger: a project of command is presented as neutral cognition.
When Willard ``wants a mission,'' the wanting is not epistemic; it is a pledge of will stamped
with the authority of ``truth.'' The sentence's first clause (``Everyone gets everything he
wants'') thus records not luck but the world's capacity to deliver the objects that our
valuations already framed as necessary; the second clause (``for my sins\ldots'') signals
the after-knowledge that those valuations were life-denying.

Against such self-deception, Nietzsche sets style—the capacity to shape one's evaluations when
reality exposes them as evasions. He urges a ``revaluation of all values''
\parencite[\S\S 203--211]{NietzscheBGE1990}, and the call to ``live dangerously!''
\parencite[\S 283]{NietzscheBGE1990} names a refusal of anesthetized security rather than a
cult of risk. In this register, fulfillment is not possession of the object but self-formation:
the will confirms itself by changing its own measure. The
\hyperref[scene:upriver-journey]{upriver progression} continuously offers occasions for such
revaluation—each checkpoint turning success into a new claim on the self. When compliance with
procedure at the \hyperref[scene:sampan]{sampan} yields horror, a Nietzschean response would be
to transvalue the maxim that licensed it. Instead, the will prefers continuity of command;
it ``gets what it wants'' (control, clarity) and is punished by the disclosure that its wanting
is reactive—obedience to inherited values that present themselves as necessity.

Nietzsche's psychology of ressentiment further clarifies the moralizing energies that travel
with domination. The weak, unable to act, invert impotence into virtue by calling the strong
``evil'' and themselves ``good'' for not doing what they cannot do
\parencite[I.10--14]{NietzscheGenealogy1994}. Yet he also describes a noble pathos that wants
to expand and test itself \parencite[\S\S 260--265]{NietzscheBGE1990}. In the film's middle
movements, these vectors cross: theatrical sovereignty stages itself as exuberance
(\hyperref[scene:kilgore-beach]{``I love the smell of napalm in the morning''}), while the
bureaucratic ``we'' that dispatches the assassin wraps elimination in the moral language of
purification. Both are forms of wanting
that the sentence anatomizes: one wants spectacle of command, the other wants justification for
command—but neither shows the transvaluative courage to alter its measure when outcomes strip
the rhetoric bare.

Kurtz, often read as the one who has gone ``beyond good and evil,'' in fact illustrates
Nietzsche's worry about the will's last refuge: after unseating inherited norms, it longs for
a final verdict that would secure mastery once more. Nietzsche's ``beyond good and evil'' is
not a license for cruelty; it is lucidity about the genealogy of one's values and the refusal
to enthrone a new absolute \parencite[\S\S 259--260]{NietzscheBGE1990}. Kurtz's pronouncement
of ``the horror'' behaves like that new absolute—a metaphysical seal on judgment that would
still the will's vulnerability. If he has ``got what he wants'' (freedom to rule, pronounce,
and be obeyed), fulfillment punishes by revealing the emptiness of mastery that will not
relinquish its last metaphysical crutch.

The health-criterion, for Nietzsche, is severe and simple: does this willing increase one's
capacity to affirm life—including ambiguity and pain—or does it shrink that capacity under a
rhetoric of truth and duty? A project may be perfectly ``true'' by institutional measures and
yet sick by this criterion. When the assignment is executed to the letter, and nothing
redemptive follows—no enlargement of perspective, no transvaluation of maxims—the confession
(``\ldots for my sins I got one'') reads as recognition that fulfillment has exposed the willing
as life-denying. In Nietzsche's terms, the punishment is not the mission's cost but its clarity:
getting what one wanted shows which kind of will one is.

\pagebreak[2]
\phantomsection
\subsection*{IV.5—Kierkegaard: Despair as Absolutized Project—Why Success Thickens the
	Misrelation}
\addcontentsline{toc}{subsection}{IV.5—Kierkegaard: Despair as Absolutized Project}
\label{ssec:iii-kierkegaard}
Søren Kierkegaard treats despair not as a mood but as a structural error in how the self relates
to itself. ``The self is a relation that relates itself to itself,'' and it can be ``sick unto
death'' when that relation is grounded in the wrong power
\parencite[pp.~49--52]{KierkegaardSUD1980}.
Two principal forms of despair matter here: (1) the despair of weakness—``not to will to be
oneself,'' and (2) the despair of defiance—to will to be oneself ``in one's own strength''
\parencite[pp.~52--61, 69--73]{KierkegaardSUD1980}. Read against Willard's confession, the line
``Everyone gets everything he wants'' sketches the field on which both forms operate;
``\ldots and for my sins I got one'' is the moment the misrelation becomes clear through
success.

Kierkegaard's analysis bites hardest when a contingent project is taken as absolute. The self,
needing to be grounded ``in the power that established it,'' substitutes a finite end as its
measure and thereby misrelates itself \parencite[pp.~79--83]{KierkegaardSUD1980}. To will to be
the one who has a mission is precisely such absolutization. Before the assignment, the self
appears as lack (restless aimlessness); once the assignment is granted, the self congeals around
the project—orientation replaces drift. But in Kierkegaard's grammar this is not healing; it is
the despair of defiance: the self wills to be itself by itself through the project. The more
coherent the mission becomes, the more intense the misrelation grows, because the self is
secured by something that cannot finally ground it.

This is why, for Kierkegaard, success does not rescue but thickens despair. Success confirms
the illusion that one can be oneself by one's own project; yet every success is also a mirror,
showing that the self remains unfounded. Kierkegaard emphasizes that despair often hides beneath
``the most colossal energy'' and apparent resolve; it is ``misrelationship in a self'' that can
be ``perfectly transparent to itself'' about its project while being wrong about its ground
\parencite[pp.~72--76]{KierkegaardSUD1980}. In this light, the cool execution of procedures and
the narrowing of affect after each ``win'' signal not mastery but the tightening of defiant
despair. The line's second clause—``for my sins I got one''—reads as the moment when the grant
of the mission throws the absence of a true ground into relief.

Kierkegaard also distinguishes immediacy (living immersed in finite goods) from reflection that
can discover the self's task \parencite[pp.~84--90]{KierkegaardSUD1980}. War intensifies
immediacy by turning every face into a function and every act into a means; it encourages the
self to hide in roles. In such a milieu, the very practices that appear to deliver meaning—orders,
dossiers, operational clarity—supply the self with a surrogate infinity: a finite project that
pretends to be sufficient. But the self, for Kierkegaard, is tasked with becoming itself before
God \parencite[pp.~79--83]{KierkegaardSUD1980}. This is not a pietistic add-on; it is his way
of marking that the self's measure must transcend its own chosen ends. Whenever the measure is
reduced to the project's success, despair results—``the greater the natural capacities, the more
dangerous the despair'' \parencite[pp.~76--78]{KierkegaardSUD1980}.

Note how this frame explains the distinctive tonality of fulfillment-as-punishment. To ``get
what one wants'' is to lose the alibi that failure provides. As long as the project is ungranted,
the self can imagine that possession will establish it. Once granted, the self's emptiness
becomes undeniable. The confession ``for my sins I got one'' thus does not (primarily) express
guilt over discrete acts; it expresses recognition of a wrong willing—to be oneself by one's own
finite project. That is Kierkegaard's ``sin'' in the strict sense: not a single deed, but a
posture of self-grounding \parencite[pp.~79--83]{KierkegaardSUD1980}.

Kierkegaard's analysis also clarifies why horror does not teach the defiant self what it most
needs to learn. The self in defiance is willing to suffer anything rather than relinquish its
chosen measure; it would rather ``be itself with all the torments of hell than not be itself''
\parencite[p.~69]{KierkegaardSUD1980}. Hence the spectacle of a self that persists—relentlessly,
competently—through increasingly unredeeming outcomes. The punishment of fulfillment is that
competence becomes the instrument of despair: every efficient act confirms the sovereignty of
the project, and every confirmation deepens the misrelation.

Is there a Kierkegaardian way out? Only if the maxim of the project is converted—a re-grounding
of the self in that ``power which established it,'' which, in his lexicon, entails repentance
and a change in the measure of willing \parencite[pp.~79--83, 111--116]{KierkegaardSUD1980}.
Short of such a conversion, neither failure nor success can heal; success merely strips away
the illusion that success could heal. Thus the two halves of the sentence lock together:
``everyone gets everything he wants'' = finite ends can indeed be obtained; ``for my sins I got
one'' = obtaining them revealed the despair that absolutized them.

\pagebreak[2]
\phantomsection
\subsection*{Dostoevsky: Independent Desire, Anti-Mechanism, and Agency That Eats Itself}
\addcontentsline{toc}{subsection}{Dostoevsky: Independent Desire and Agency}
\label{ssec:iii-dostoevsky}
Fyodor Dostoevsky's \emph{Notes from Underground} is the classic anatomy of a will that prefers
independence to well-being—``man only wants independent desire, whatever that independence may
cost'' \parencite[p.~131]{DostoevskyNFU1994}. The underground man's most scandalous claim—``To
hell with two times two makes four!'' \parencite[p.~129]{DostoevskyNFU1994}—is not
anti-arithmetic; it is anti-mechanism in human affairs. He rejects any calculus in which
rational prediction, utility, or institutional procedure would close the space of spontaneous
willing. Read in this key, the line ``Everyone gets everything he wants'' marks not prosperity
but a world well-stocked with mechanisms that deliver objects on demand; ``for my sins I got
one'' acknowledges the price of willing agency itself within such machinery.

The underground man's revolt targets the dream that human conduct can be rendered scientific—that
motives can be predicted and optimized so that ``good'' outcomes follow from the right levers
\parencite[pp.~120--132]{DostoevskyNFU1994}. His point is not that people are irrational, but
that personhood includes a residual freedom that will ``assert itself'' against the system,
even destructively, simply to prove it exists \parencite[pp.~129--132]{DostoevskyNFU1994}.
This is why a ``crystal palace'' of perfect provisions would provoke sabotage; the human being,
he insists, will sometimes choose what is harmful to demonstrate authorship. The
\hyperref[scene:briefing]{dossier room's} hygienic proceduralism—clarity of ends, chain of
command, calibrated means—inhabits precisely the rational order the underground man distrusts.
Willard's wanting a mission is not a longing for certainty alone; it is the chance to act, to
break the inertia of Saigon's aimlessness, even if the action risks moral injury. The ``sin''
is that the system can supply just such occasions and call the result necessary.

Dostoevsky's dialectic also explains the peculiar pleasure the underground man takes in acting
against his own interest: ``the most advantageous advantage'' is sometimes the freedom to choose
what is not advantageous \parencite[pp.~129--131]{DostoevskyNFU1994}. Agency is not measured by
outcomes but by the felt authorship of choice. That is why, on his account, a rational program
that secures only beneficial outcomes is degrading: it would reduce a man to a ``piano key'' on
which nature (or the institution) plays \parencite[pp.~115--120]{DostoevskyNFU1994}.
In an environment of orders and protocols, the upriver insistence on continuing, whatever the
evidence or cost, becomes intelligible as a defense of non-instrumentality: continuing proves
that one is not merely a key. Fulfillment punishes because the moment the system grants the
mission, the space of ``independent desire'' shrinks into the execution of a mechanism that
now owns the storyline.

Yet Dostoevsky's insight is double-edged. The underground man's independence is real, but it
corrodes itself when it refuses any measure beyond negation. He confesses to relishing
humiliation and spite, to savoring ``the sweetly painful pleasure'' of acting against himself
\parencite[pp.~108--115]{DostoevskyNFU1994}. This is agency that proves itself by injury.
When procedure yields horror (the \hyperref[scene:sampan]{sampan}), the rational mechanism has
plainly failed; but if
the next choice is animated only by the need to keep asserting agency—``I go on because I go
on''—the will ratifies the same emptiness the underground man inhabits. ``Everyone gets
everything he wants'' then describes a trap: the institution gets its obedient executor; the
agent gets the feeling of authorship; neither gets a norm by which the act could be vindicated.

Dostoevsky also anticipates what we might call moral theater: the staging of motives after the
fact to render destructive agency palatable. The underground man is merciless about his own
self-narration—confessing how quickly the ego invents edifying reasons for what was, in truth,
caprice or spite \parencite[pp.~103--107]{DostoevskyNFU1994}. This maps onto the rhetoric that
frames the assignment as sanitary necessity. The mask of moral purification (``remove an
aberration'') converts the hunger to act into an edifying plot; fulfillment then functions as
exposure when, at the end, the narrative yields no enlargement of soul. The confession ``for my
sins I got one'' reads, in Dostoevskian terms, as the recognition that the story was a postscript
to the will to act, not its ground.

A further Dostoevskian thread concerns responsibility under conditions of determinism. The
underground man refuses to let causal explanation excuse him: even if he can trace motives, he
will not permit the explanation to replace ownership \parencite[pp.~109--113]{DostoevskyNFU1994}.
This refusal illuminates the post-fulfillment chill: once the mission is complete, the agent
cannot hide in causal chains (``orders,'' ``procedure,'' ``necessity'') without committing the
very self-abdication he despises. Fulfillment punishes by removing alibis: the deed is done;
the authorship is mine.

Finally, Dostoevsky's anti-mechanism clarifies why the film's most efficient scenes are its most
disturbing. The underground man's nightmare is not chaos but perfect order—an order so seamless
it leaves no room for non-instrumental choice. Where everything ``works,'' the human residue can
only show itself by breaking the system or by converting obedience into a performance of will.
In either case, getting what one wanted discloses a deficit: agency defended merely as
independence becomes self-consuming. The sentence's halves therefore lock: the world can indeed
deliver the occasion to act (``everyone gets\ldots''), but the one who wanted agency itself
discovers, upon receiving it, that agency without a measure is indistinguishable from compulsion
in disguise—hence ``\ldots for my sins I got one.''

\pagebreak[2]
\phantomsection
\subsection*{Sartre: Freedom as Condemnation, the Impossible Synthesis, and Fulfillment as
	Exposure}
\addcontentsline{toc}{subsection}{Sartre: Freedom as Condemnation}
\label{ssec:iii-sartre}
Jean-Paul Sartre's ontology makes ``Everyone gets everything he wants'' a trap built into
freedom. For him, human reality (\emph{pour-soi}) is a lack that projects itself toward being;
it is ``what
it is not and not what it is,'' a perpetual surpassing of itself
\parencite[pp.~100--110]{SartreBN2003}. Desire therefore aims, at bottom, at an ontological
closure it can never attain. The will does not simply seek objects; it seeks to abolish its
lack by becoming a settled being. That is the hidden horizon against which the mission takes on
its peculiar glow. ``For my sins I got one'' names the moment the project's promised closure
reveals itself as structurally impossible.

\phantomsection
\subsubsection*{Condemned to Be Free: Pour-soi and the Project}
\addcontentsline{toc}{subsubsection}{Condemned to Be Free}

The project-form is central to Sartre's account of freedom. Freedom is not a privilege but the
very structure of consciousness: we are ``condemned to be free,'' without essence to excuse or
guarantee our choices \parencite[pp.~34--36]{SartreBN2003}. Because the \emph{pour-soi} is
nothing but transcendence beyond the given (facticity), every life is a project—a coherent
orientation that confers meaning retroactively on its acts
\parencite[pp.~561--569]{SartreBN2003}. The \hyperref[scene:briefing]{Saigon acceptance scene}
reads here as the decisive orientation of a freedom in flight from its drift: a project chosen
to still contingency by giving it a vector. But, in Sartre's grammar, such orientation never
stills the source; it intensifies responsibility. Once the mission is chosen, there are no
alibis left.

\phantomsection
\subsubsection*{The Project to Be God: Impossible Ontological Synthesis}
\addcontentsline{toc}{subsubsection}{The Project to Be God}

Beneath every finite project, Sartre identifies a secret, universal temptation—the ``project to
be God'': to fuse our throwness (facticity) and our transcendence into a single, self-grounding
plenitude \parencite[pp.~586--604]{SartreBN2003}. That synthesis is impossible. The
\emph{pour-soi} can never coincide with itself as the \emph{en-soi} does; it can only nihilate
the given and project beyond it. When a mission is taken as the end that would reconcile what
we are (situated, limited) with what we intend (sovereign authorship), fulfillment must punish
because its very success exposes the misconceived telos: the project could not, even in
principle, provide what the will implicitly asked of it—ontological peace.

\phantomsection
\subsubsection*{Bad Faith, the Look, and Absolute Responsibility}
\addcontentsline{toc}{subsubsection}{Bad Faith and Responsibility}

Sartre's analysis of bad faith illuminates how institutional roles mask this impossibility.
``Bad faith'' names the flight from freedom by posing oneself as either pure thing (just
obeying orders) or pure transcendence (unconditioned author), disowning the inseparable unity
of both \parencite[pp.~86--116]{SartreBN2003}. The procedural rhetoric of dossiers, signatures,
and necessity tempts the agent to occupy the role of function—a thing among things—while
narrating himself as a lucid executor. In fact, the act is freely chosen under a maxim that the
agent owns. The \hyperref[scene:french-plantation]{French plantation} stages bad faith as
frozen ritual: the colonists maintain elaborate forms (dinners, refinement, cultural identity)
despite the empire's collapse. They achieved their project (land, dynasty) and now face that
nothing in the world compelled it or justifies keeping it, yet they refuse this recognition by
clinging to the role. The forms remain, but the alibi (imperial mandate, civilization's burden)
has evaporated. Fulfillment punishes because once the project is complete, the alibi of role
collapses: nothing in the world compelled this project as mine. The confession (``\ldots for my
sins I got one'') reads as a crack in bad faith: a recognition that the necessity was staged.

The problem deepens when we consider Sartre's account of the Look (\emph{le regard}), which
shows how others reveal our facticity while tempting us to convert them into means for our
project \parencite[pp.~252--302]{SartreBN2003}. A mission-form that objectifies faces as
obstacles or instruments produces a world of being-for-others devoid of reciprocity. Each
``efficient'' success therefore deepens alienation: it multiplies acts in which the other's
freedom is suppressed to maintain the project's clarity. The more cleanly the procedure runs,
the more legible the structure becomes: meaning has been outsourced to instrumental success,
not grounded in a shared world. Fulfillment is thus a revelation: what we wanted was not meaning
but the uninterrupted sovereignty of a plan.

Sartre's relentless claim is that responsibility remains absolute. Causal explanation never
cancels authorship. Situations ``are what they are,'' but they are what they are for a freedom
that chooses what to make of them \parencite[pp.~553--561]{SartreBN2003}. This is why the end
of a project often feels accusatory. When nothing redemptive follows a technically perfect
execution, the agent confronts the naked fact that the project's value was not in the world but
in the choice that sustained it. ``Everyone gets everything he wants'' then means: the world is
reliable at delivering objects for our projects; ``\ldots and for my sins I got one'' means:
once delivered, the project reflects my choice back at me without the cushion of failure.

One might object, Sartreanly, that lucid perseverance—owning the act without appeal—could
transfigure the mission into authenticity. But for Sartre authenticity is not stubbornness; it
is lucidity about the impossibility of completion and refusal of bad faith in either direction
(no hiding in role; no fantasy of omnipotent authorship). In a world where the project's
structure systematically instrumentalizes others, lucidity would require altering the project or
abandoning it, not merely executing it honestly. Where no such alteration occurs, fulfillment
cannot be redemptive: it is an X-ray of the willing that carried it.

Sartre thus underwrites both halves of Willard's sentence. ``Everyone gets everything he
wants'': the world provides ample situations in which freedom can adopt ends and see them
through. ``For my sins I got one'': because the end was implicitly a bid for the impossible
synthesis (completion, immunity from ambiguity), getting it reveals the project as bad-faith
flight from freedom's structure. Punishment is not failure but clarity—the clarity that
completion was never on offer.

\pagebreak[2]
\phantomsection
\subsection*{Beauvoir: Reciprocity as the Form of Authentic Freedom}
\addcontentsline{toc}{subsection}{Beauvoir: Reciprocity and Authentic Freedom}
\label{ssec:iii-beauvoir}
Simone de Beauvoir's central thesis is that freedom is relational: my freedom is authentic only as a
practice that wills the freedom of others \parencite[p.~73]{Beauvoir1976}. This follows from
her ontology of ambiguity: human existence is at once facticity and transcendence, and meaning
is co-authored in a shared world \parencite[pp.~9--14, 24--30]{Beauvoir1976}. A project that
systematically reduces others to means contradicts the very structure of freedom it claims to
exercise. In this light, ``Everyone gets everything he wants'' is ethically indeterminate until
we ask whether the wanting included the other's freedom; ``\ldots and for my sins I got one''
reads as the moment the granted project exposes that it did not.

Beauvoir distinguishes authentic from inauthentic willing: the former embraces ambiguity and
seeks ``situations'' where others can transcend; the latter flees ambiguity by freezing others
into functions \parencite[pp.~70--76, 134--145]{Beauvoir1976}. Authenticity is not benevolence
but method: to pursue ends in a way that enlarges co-agency. This supplies a criterion the film
keeps failing. The \hyperref[scene:briefing]{Saigon briefing} frames action as administrative
necessity; its language (sanitation, dossiers) predetermines an inauthentic mode of encounter in
which faces will appear as obstacles or instruments. That mode is not corrected by later
``successes''; it is confirmed by them.

Beauvoir's account of oppression makes this failure legible. Oppression is not just harm; it is
the organization of the world so that another's transcendence can appear only as a threat
\parencite[pp.~85--91, 157--161]{Beauvoir1976}. Where a project's telos presupposes such
organization, efficiency deepens guilt. The \hyperref[scene:sampan]{sampan inspection} is
exemplary: even before the fatal moment, the protocol treats persons as risk variables in a
supply chain. Beauvoir's
question is not whether force is ever permissible; it is whether the mode of action keeps open
a horizon in which the other can still be a source of meaning
\parencite[pp.~139--147, 164--173]{Beauvoir1976}. Here the very grammar of the check—its
anticipations, its allowable responses—has already closed that horizon.

Beauvoir recasts justification in terms of world-building: deeds are justified when they found
a common world, i.e., when they set up institutions or practices through which others can also
project ends \parencite[pp.~145--153]{Beauvoir1976}. Measured by that standard, the film's
repeating structures—Kilgore's spectacle of sovereignty, Do Lung Bridge rebuilt nightly by
nameless hands—show action that circulates without founding. The spectacular will and the
faceless mechanism are two faces of the same inauthenticity: each consumes the other's
transcendence for its own continuity. ``Everyone gets everything he wants'' here names only the
reliability of means; it says nothing about the world those means build.

Beauvoir insists that constraint does not absolve; it conditions responsibility. Authentic
freedom exploits cracks in necessity to remake situations toward reciprocity
\parencite[pp.~34--42]{Beauvoir1976}. Hence the ethical failure is clearest where things
``work.'' When procedures function smoothly and no revision follows—no new practice that
protects faces, no altered maxim that includes co-agency—success becomes self-indicting. This is
why the confession's sting is specifically Beauvoirian: ``\ldots for my sins I got one''
acknowledges that the mission's efficient fulfillment revealed what its end had never
embraced—the other's freedom.

Finally, her ethics reframes the film's terminal clarity. For Beauvoir, one cannot sanitize
ambiguity; every deed risks harm. But she denies the alibi of purity: the right response to risk
is vigilant reciprocity, not resignation \parencite[pp.~139--147]{Beauvoir1976}. If a project's
form cannot be made reciprocal, authenticity demands refusal or re-foundation. In a setting
where refusal is not chosen and re-foundation never occurs, the two halves of Willard's line
align perfectly with Beauvoir's verdict: the world can indeed deliver the object of desire
(``everyone gets\ldots''), and precisely that delivery discloses that what was desired was
sovereignty without co-agency (``\ldots for my sins I got one'').

\pagebreak[2]
\phantomsection
\subsection*{Camus: Absurd Lucidity, Revolt ``Without Appeal,'' and Completion as Knowledge
	(Not Meaning)}
\addcontentsline{toc}{subsection}{Camus: Absurd Lucidity and Revolt}
\label{ssec:iii-camus}
Albert Camus begins with a refusal of consolations. ``There is but one truly serious philosophical
problem, and that is suicide'' \parencite[p.~3]{CamusMyth1991}. The claim sets the tone: the
question is not whether life can be made coherent, but whether one can live honestly when it
cannot. The absurd is the name for this standoff—``born of the confrontation between the human
need and the unreasonable silence of the world'' \parencite[p.~28]{CamusMyth1991}. Read against
Willard's line, ``Everyone gets everything he wants,'' the absurd warns that the delivery of
objects and outcomes has no built-in power to answer the need that generated them. ``\ldots And
for my sins I got one'' sounds, in Camus's vocabulary, like an onset of lucidity: completion
gives knowledge, not meaning.

Camus is suspicious of what he calls philosophical suicide—any leap (religious, metaphysical,
or ideological) that smuggles meaning back in after the absurd has been recognized
\parencite[pp.~53--58]{CamusMyth1991}. To live ``without appeal'' is to refuse that leap
\parencite[p.~54]{CamusMyth1991}. Much of the film's rhetoric—dossier certainties, the mission's
hygienic narrative—functions as an appeal to an order that would dissolve ambiguity. As the
journey upriver strips those narratives away, the world retains its ``unreasonable silence,''
yet the project continues. Camus would say: the willing, deprived of its fictions, is now
exposed to the task of revolt—not overthrow, but a ``permanent confrontation'' with
meaninglessness \parencite[p.~55]{CamusMyth1991}. If the revolt does not transvalue its maxims,
completion will punish by revealing that the act was, after all, an appeal in disguise.

Camus reframes fulfillment by recoding value as lucidity, freedom, passion—the three modalities
of living the absurd \parencite[pp.~54--71]{CamusMyth1991}. Lucidity means remaining with what
the world actually grants; freedom means recognizing that, if meanings are not given, our
projects are ours without metaphysical guarantees; passion means intensifying experience rather
than seeking a terminal sanction. In this light, the film's most efficient scenes (where things
work) are its least meaningful. The \hyperref[scene:do-lung-bridge]{Do Lung Bridge}
cycle—building by day, destruction by night—reads like the myth of Sisyphus in military dress:
strenuous labor without appeal, the
task's perfection indifferent to significance. Camus's verdict on Sisyphus—``One must imagine
Sisyphus happy'' \parencite[p.~123]{CamusMyth1991}—does not romanticize toil; it claims that
honesty about the task's finitude can be a site of dignity. The catch is that such dignity
requires abandoning the promise that the task will redeem. The
\hyperref[scene:french-plantation]{French plantation} presents the inverse: where Sisyphus
never reaches the summit and thus retains the task, the colonists achieved their goal (land,
dynasty, permanence) and discovered that completion delivers no meaning. They are past the
summit and find only emptiness—a monument to the absurd truth that getting what one wants does
not answer the need that generated the wanting. Where Willard's project is still mortgaged to a
redemptive story (purge the aberration, restore sense), its successful completion must recoil
as knowledge that no redemption follows.

The figure most tempted by appeal is the one who seeks a final verdict on existence. Camus's
polemic targets precisely that longing: the desire to seal the world with an ultimate judgment
that would still the need to will \parencite[pp.~53--60]{CamusMyth1991}. In this register,
Kurtz's pronouncement—``the horror''—behaves like a metaphysical seal, an ultimate word that
would turn lucidity into law. Camus would demur: the absurd forbids the last word. To honor the
absurd is to continue acting without that word—no appeal to a transcendent rule, no enthronement
of the self as tribunal. ``Everyone gets everything he wants'' thus becomes, for Camus, a
litmus: if what one wanted was an ultimate exoneration, getting it will read as punishment—the
world remains silent.

Camus's portrait of the absurd hero helps explain the tone of the confession. The absurd hero
does not seek to solve the absurd; he keeps faith with it through measured revolt
\parencite[pp.~54--60, 121--123]{CamusMyth1991}. He does not deny limits, and he does not
pretend his acts are guaranteed meaning by a higher court. Where the mission-form equates
success with justification, Camus severs that link. After the deed, what remains is clarity:
we know what the world is (silent), what we are (beings who will without guarantee), and what
action can be (finite, accountable, unredeemed). If the project has been conducted under the
illusion that completion = meaning, then completion unveils the illusion. ``For my sins I got
one'' is exactly that unveiling—the point at which the will, faced with the absurd, loses its
alibi.

Finally, Camus's injunction—live ``without appeal''—tightens the essay's thesis. If the world
can reliably supply objects for our projects (hence ``everyone gets\ldots''), and if those
projects often carry tacit appeals (to necessity, to cleansing narratives, to final judgments),
then the punitive feel of fulfillment is simply the return of the real: the object arrives; the
appeal fails; lucidity remains. The task, if there is one, is to convert willing from a demand
for consummation into a discipline of revolt—a way of acting that neither lies about meaning
nor abdicates it. Absent that conversion, getting what one wants will continue to accuse the
will that wanted it.

\pagebreak[2]
\phantomsection
\subsection*{Heidegger: Finitude, Anticipatory Resoluteness, and Why Completion Is
	Ontologically Out of Reach}
\addcontentsline{toc}{subsection}{Heidegger: Finitude and Resoluteness}
\label{ssec:iii-heidegger}
Martin Heidegger's account of existence (Dasein) makes completion a category mistake. Dasein is
essentially being-possible—a projecting that never coincides with itself as a finished thing;
its wholeness is disclosed only in being-toward-death \parencite[pp.~279--311]{HeideggerBT1962}.
Death is not (primarily) a future event to be scheduled but the ownmost, nonrelational
possibility that individualizes Dasein now, stripping away the illusions of totalization
\parencite[pp.~294--307]{HeideggerBT1962}. Thus, any project that promises narrative wholeness—that
a mission will ``make it come together''—misreads existence. When Willard says ``Everyone gets
everything he wants,'' the Heideggerian gloss is brutal: the world may indeed supply objects and
tasks, but existence is not something an object can finish. ``\ldots And for my sins I got one''
is the moment the project's alleged telos collides with finitude.

Heidegger's analysis of everydayness and the They (\emph{das Man}) clarifies why projects so
easily wear the mask of necessity. In average everydayness, Dasein takes over possibilities ``as
one does,'' letting anonymous norms dictate what counts as urgent, clean, or right
\parencite[pp.~149--168]{HeideggerBT1962}. The \hyperref[scene:briefing]{Saigon briefing}'s
procedural tone—dossiers, signatures, the grammar of sanitation—exemplifies this absorption in
\emph{das Man}: the mission
shows up as what ``one'' does when a file reads anomalous. To take it up as such is not yet
resolute choice; it is fallenness into the ready-made interpretation. Fulfillment then
``punishes'' by disclosing that the accomplished sequence was never a route to owned wholeness;
it was a they-self rhythm all along.

The film's temporality maps onto Heidegger's account of ecstatic time. Dasein's temporality is
not a string of nows but an ``ahead-of-itself'' (future), already-in (past), and being-alongside
(present) \parencite[pp.~373--383]{HeideggerBT1962}. The Do Lung Bridge cycle—construction by
day, destruction by night—stages a caricature of inauthentic time: a serial present that never
gathers. Anticipatory resoluteness does not end such cycles; it interprets them soberly by
owning death as the limit that prevents totalization \parencite[pp.~307--311]{HeideggerBT1962}.
By this light, the climactic ``success'' cannot heal the fracture; it can only remove the alibi
that failure once provided. The felt judgment of the line is that clarity: the project is
complete and therefore unable to hide the truth that existence cannot be.

Heidegger's conscience and guilt intensify the point. Conscience ``calls'' Dasein from \emph{das
	Man} to its ownmost possibility; guilt (\emph{Schuld}) names not juridical fault but
being-the-basis of a nullity—that our thrown projection always leaves something out and cannot
guarantee innocence \parencite[\S\S 57--60, pp.~311--354]{HeideggerBT1962}. When procedures run
perfectly (the \hyperref[scene:sampan]{sampan inspection} as ``by the book'') and still yield
devastation, what is revealed is not only moral failure but ontological mismatch: the attempt
to secure existential rightness via technical closure. Anticipatory resoluteness would require
owning that mismatch, not masking it with narratives of cleansing. The confession—``for my
sins I got one''—is a resolute sentence in this sense: it drops the promise of narrative
wholeness and accepts
finitude as the horizon that renders completion impossible.

\pagebreak[2]
\phantomsection
\subsection*{Levinas: The Face's Prohibition, Asymmetrical Responsibility, and Why
	``Success'' Condemns Instrumental Projects}
\addcontentsline{toc}{subsection}{Levinas: The Face and Asymmetrical Responsibility}
\label{ssec:iii-levinas}
Emmanuel Levinas relocates first philosophy from ontology to ethics: the encounter with the face
institutes an asymmetrical demand prior to any project or knowledge. ``Desire is desire for the
absolutely other'' \parencite[p.~33]{LevinasTI1969}, and the face ``forbids us to kill''
\parencite[p.~199]{LevinasTI1969}. This is not a thesis about consequences but a command
inscribed in the presentation of the other as infinite—irreducible to roles, functions, or my
plans \parencite[pp.~194--201]{LevinasTI1969}. Measured by this standard, ``Everyone gets
everything he wants'' is ethically null until we ask whether what was wanted preserved the
other's irreducibility; ``\ldots and for my sins I got one'' reads as the moment when a granted
project reveals, by its own success, that it had bracketed that demand.

Levinas's notion of totality versus infinity names the fault-line. Totality is the regime that
reduces alterity to the Same—catalogues, protocols, categories; infinity is the breach of that
reduction in the epiphany of the face \parencite[pp.~21--24, 33--36]{LevinasTI1969}. The
mission-form—dossier, diagnosis, elimination—is quintessentially totalizing: it metabolizes
faces as data points and tasks. The sampan scene is an X-ray: even before the fatal shot, the
encounter runs on risk calculus. In Levinas's grammar, the ethical failure precedes the mistake;
the very mode of approach ``has already spoken'' by refusing the face's claim. Success cannot
redeem such refusal; it confirms it. ``Getting what one wants'' within this regime is punishment
as self-revelation: the act returns to the agent as accusation.

Levinas is explicit that the ethical relation is asymmetrical: I am responsible for the other
beyond reciprocity or contract \parencite[pp.~215--219]{LevinasTI1969}. This asymmetry is
precisely what proceduralism neutralizes, since procedures aim to distribute liability
symmetrically. Hence the peculiar chill of the film's most efficient moments: where a protocol
works, the asymmetry has been most thoroughly suppressed. The ethical demand has not been
answered; it has been absorbed—turned into a variable among others. Levinas's insistence that
the face is a ``poor one, a stranger'' \parencite[p.~213]{LevinasTI1969} gives content to the
felt wrongness of treating villagers, boat crews, and even soldiers as means for the continuity
of the project. The wrongness is not (only) that harm occurs; it is that the form of encounter
precluded responsibility before deciding what to do.

The assassination order against Kurtz does not escape this logic by turning against a tyrant.
Levinas's ``Thou shalt not kill'' is not a rule applied to friends but the structure of
encounter itself \parencite[p.~199]{LevinasTI1969}. To meet anyone—enemy included—first as a
bearer of exteriority is to be summoned to justification. There may be cases, Levinas allows,
where politics demands force; but politics is always under judgment by ethics
\parencite[pp.~21--24]{LevinasTI1969}. The film's denouement shows the inversion: politics
judges ethics, and efficiency is taken as justification. That is why the line's second half
sounds like a verdict: ``\ldots for my sins I got one'' acknowledges that the project's end
never included the first relation—the face's command—so its successful completion can only
declare that exclusion more clearly.

Levinas also explains why horror often clarifies rather than teaches in such worlds. Horror
strips away alibis and yet, without a conversion of the mode of approach, it cannot generate the
responsibility it reveals. The proper response is not a grand theory but a change in the grammar
of encounter—hospitality, attention, refusal of instrumentalization
\parencite[pp.~200--206]{LevinasTI1969}. In their absence, ``Everyone gets everything he wants''
remains the slogan of totality: the world is very good at supplying means. The punishment of
fulfillment is the renewed summons one cannot now un-hear.

\pagebreak[2]
\phantomsection
\subsection*{Koj{\`e}ve: Desire of Desire, History as Recognition, and Why Mission-Form
	Fulfillment Recurs as Lack}
\addcontentsline{toc}{subsection}{Kojève: Desire of Desire and Recognition}
\label{ssec:iii-kojeve}
Alexandre Koj{\`e}ve radicalizes Hegel's insight in anthropological terms: human desire is
``desire of another's desire''—a need to be desired/recognized by a free other
\parencite[p.~6]{KojeveIRH1980}.
The object mediates this relation, but it is not the final aim. Hence the lordship/bondage
dialectic reads, in Koj{\`e}ve's gloss, as the matrix of history: the Master obtains things
(and obedience) but not the recognition that would satisfy a human desire; the Slave, through
fearful work, transforms the world and, in so doing, becomes the bearer of truth
\parencite[pp.~27--34, 158--164]{KojeveIRH1980}. If the film's world can reliably deliver
missions and outcomes—``everyone gets everything he wants''—what it cannot deliver, by those
same means, is the desire of the other freely given.

Koj{\`e}ve's portrait of the Master maps neatly onto the mission-form that prizes clean
execution and visible effects. The Master ``gets what he wants,'' but his world is populated by
things and submissions, not by interlocutors who can confirm him
\parencite[pp.~27--34]{KojeveIRH1980}. In such a regime, success increases dependence on further
success, because each attainment fails to supply the missing confirmation. The appetite becomes
serial: new targets, new proofs, new shows of power. This is the historical engine that drives
the ``pendulum'' of operations: each completion—however perfect—returns as renewed lack, not
because the agent is psychologically thin, but because the form of fulfillment excludes the kind
of acknowledgment that could end the sequence.

By contrast, Koj{\`e}ve sees the Slave's work as the slow route to recognitive stability: work
shapes a common world in which self and other can appear to each other as free
\parencite[pp.~158--164]{KojeveIRH1980}. That is why he can speak of the ``end of history'' as
a horizon of universal recognition, not maximal accumulation
\parencite[pp.~158--164]{KojeveIRH1980}. Measured against this horizon, the upriver procedures
create no institutions of mutual address; they routinize asymmetry. Even the final act—eliminating
the figure who has refused the institution—seeks restoration of order without creating the space
in which recognition could be mutual. Fulfillment thus returns as judgment: the very evidence of
technical success is the evidence that the recognitive aim was never in view.

Koj{\`e}ve's reading also explains the peculiar tone of mastery's self-knowledge. Once the mask
of ``truth'' and ``necessity'' falls, the Master must either convert—accept that what he wanted
cannot be had by command—or double down, seeking ever more unchallengeable evidence of
sovereignty. The line ``\ldots for my sins I got one'' registers the first path as insight
without conversion: one sees that the delivery of ends cannot deliver recognition, yet one has
already acted in the Master's grammar. The punishment is temporal: the completed project does
not close history; it lengthens the sequence of unsatisfying confirmations.

In Koj{\`e}ve's terms, then, the film's world is historically stuck between mastery's emptiness
and the slow, dangerous labor that could found a recognitive order. ``Everyone gets everything
he wants'' names a high-functioning apparatus for producing things and effects; ``\ldots and for
my sins I got one'' names the self-knowledge that, within that apparatus, the human desire—desire
for the other's free acknowledgment—was never addressed. What returns is not failure but the
truth about what was really wanted.

\pagebreak[2]
\phantomsection
\subsection*{Comparative Discussion: Convergences and Tensions}
\addcontentsline{toc}{subsection}{Comparative Discussion}
\label{ssec:iv-comparative-discussion}

The preceding twelve analyses reveal not a single philosophical account of Willard's line but
a field of competing and complementary diagnoses. Some tensions are productive: they refine
the verdict by forcing precision about what kind of failure fulfillment exposes. Other
convergences are striking: across metaphysical, existential, and ethical vocabularies, the
philosophers agree that ``getting what one wants'' punishes because it reveals the will's
prior orientation or structure. What follows maps the key debates and their implications for
reading the film.

\phantomsection
\subsubsection*{Is Recurrence Curse or Opportunity? Schopenhauer, Nietzsche, Camus}
\addcontentsline{toc}{subsubsection}{Recurrence: Curse or Opportunity?}

Schopenhauer's phenomenology is precise: satisfaction ``at once makes room for a new one,'' so
life swings ``between pain and boredom'' \parencite[pp.~312, 319]{SchopenhauerWWR1969}. The
\hyperref[scene:upriver-journey]{upriver sequence} confirms this---each checkpoint delivers
relief that immediately becomes renewed lack. Nietzsche objects that such pessimism
misconstrues the task: recurrence is not a curse if the will has the courage to revalue
itself, to create new measures rather than repeat
old consumption \parencite[\S\S 34, 283]{NietzscheBGE1990}. What corrodes willing is not its
repetition but its dishonesty---domination disguised as truth.

The film tests both claims. Nietzsche is right that the mission wears the mask of cognition
(dossiers, rationality, surgical necessity), and that mask licenses command. Yet the narrative
never transvalues. After the \hyperref[scene:sampan]{sampan}, after the
\hyperref[scene:do-lung-bridge]{bridge}, no new measure emerges. Schopenhauer's
pendulum reasserts itself: fulfillment disenchants, and the will swings back to lack. Camus
cuts between them with lucidity: even a creative will must live ``without appeal,'' and no
final sanction redeems completion \parencite[pp.~28, 54, 121--123]{CamusMyth1991}. The film
sides with Camus---the world delivers objects, but what returns is knowledge, not meaning. The
convergence is grim: whether the problem is metaphysical mechanism (Schopenhauer), dishonest
transvaluation (Nietzsche), or absurdist silence (Camus), fulfillment cannot heal the will.

\phantomsection
\subsubsection*{Does Success Ever Vindicate? Kant, Nietzsche, Kierkegaard}
\addcontentsline{toc}{subsubsection}{Does Success Ever Vindicate?}

Kant denies that outcomes certify worth: the good will is ``good \ldots\ in itself,'' not
``because of what it effects'' \parencite[p.~27]{KantGroundwork1996}. The
\hyperref[scene:sampan]{sampan inspection}, executed flawlessly, still fails the humanity
constraint---persons treated merely as means cannot be rescued by efficient procedure
\parencite[pp.~36--37]{KantCPrR1996}. Nietzsche
counters that Kantian morality can itself be a will to command in disguise: the ``desire for
`truth''' becomes a tool of domination when it pretends neutrality
\parencite[\S 34]{NietzscheBGE1990}.

Both critiques land. The briefing room stages Nietzsche's suspicion---rational necessity as
rhetorical cover for institutional will. Yet Kant supplies the verdict that still condemns:
even if we unmask the rhetoric, the maxim (eliminate persons designated as obstacles) fails
universalizability. Kierkegaard adds an internal dimension: even if the maxim somehow passed,
absolutizing a finite project as the self's ground thickens despair
\parencite[pp.~69--83]{KierkegaardSUD1980}. The triple pressure is severe: success cannot
vindicate (Kant), unmasking cannot excuse (Nietzsche), and structural rightness cannot cure
misrelation (Kierkegaard). ``For my sins I got one'' thus reads as the removal of every alibi.

\phantomsection
\subsubsection*{Freedom as Burden or Condemnation? Sartre, Beauvoir, Dostoevsky}
\addcontentsline{toc}{subsubsection}{Freedom as Burden or Condemnation?}

Sartre's radical claim---we are ``condemned to be free''
\parencite[pp.~34--36]{SartreBN2003}---makes every project an authorship with absolute
responsibility. The mission cannot be blamed on
orders or necessity; it is freely chosen and owned. Beauvoir specifies the ethical constraint:
freedom is authentic only when it wills the freedom of others \parencite[p.~73]{Beauvoir1976}.
A project that systematically instrumentalizes contradicts freedom's structure. Together, they
condemn the mission on two grounds: it is bad faith (Sartre) and it violates reciprocity
(Beauvoir).

Dostoevsky complicates this by insisting that agency itself can be the disease. The
Underground Man wants ``independent desire, whatever that independence may cost''
\parencite[p.~131]{DostoevskyNFU1994}---he would rather act destructively than be a ``piano
key'' in a rational system. The mission-form threatens precisely this: absorption into
mechanism. Yet agency defended as pure negation (``I go on because I refuse the system'')
corrodes itself into spite. The film stages both traps: obedience performed as authorship
(bad faith) and continuation without measure (self-consuming agency). Fulfillment exposes that
neither path preserves genuine freedom.

\phantomsection
\subsubsection*{Completion as Ontological Error: Sartre, Heidegger}
\addcontentsline{toc}{subsubsection}{Completion as Ontological Error}

Sartre and Heidegger converge that completion is impossible, but their reasons differ. For
Sartre, the \emph{pour-soi} is perpetual transcendence; it ``is what it is not and not what
it is'' \parencite[pp.~100--110]{SartreBN2003}. The hidden ``project to be God''---to fuse
facticity and transcendence into self-grounding plenitude---cannot succeed
\parencite[pp.~586--604]{SartreBN2003}. Heidegger roots the error differently: Dasein's
wholeness is disclosed only in being-toward-death, which individualizes now and strips
totalization-fantasies \parencite[pp.~294--307]{HeideggerBT1962}. Where Sartre diagnoses a
wish for ontological closure, Heidegger diagnoses fallenness into \emph{das Man}---the fantasy
that doing ``what one does'' could yield authentic wholeness
\parencite[pp.~149--168]{HeideggerBT1962}.

The difference matters for interpretation. Sartre reads the mission's end as exposing bad
faith; Heidegger reads it as removing the alibi of average everydayness. Both see the vacuum
after clean procedures, but Sartre emphasizes free choice's responsibility, while Heidegger
emphasizes the they-self's inauthenticity. The film allows both: Willard freely chose the
project (Sartre) and absorbed it as ``what one does'' (Heidegger). Fulfillment punishes both
ways.

\phantomsection
\subsubsection*{Reciprocity or the Face's Command? Beauvoir, Levinas}
\addcontentsline{toc}{subsubsection}{Reciprocity or the Face's Command?}

Beauvoir and Levinas both condemn instrumental projects, but from different starting points.
Beauvoir builds the other into freedom's structure: authentic willing must will the other's
freedom \parencite[p.~73]{Beauvoir1976}. Projects are justified when they found situations
where others can transcend \parencite[pp.~145--153]{Beauvoir1976}. Levinas argues this comes
too late: the face's prohibition (``Thou shalt not kill'') precedes all projects and resists
assimilation into reciprocal frameworks \parencite[pp.~199, 21--24]{LevinasTI1969}. Ethics is
asymmetrical---I am responsible for the other beyond contract.

The tension is productive. Beauvoir's framework can critique the mission's failure to build a
common world; Levinas can indict the very mode of approach (dossier, protocol) as a refusal of
the face. Beauvoir worries Levinas risks ethical purity without political efficacy; Levinas
warns Beauvoir's world-building easily re-totalizes. The film confirms both critiques:
procedures flatten alterity (Levinas) and successes never found shared institutions
(Beauvoir). ``Everyone gets everything he wants'' names efficiency without co-agency; ``for my
sins I got one'' marks the double failure.

\phantomsection
\subsubsection*{Objects or Recognition? Hegel, Kojève}
\addcontentsline{toc}{subsubsection}{Objects or Recognition?}

Hegel's master-slave dialectic reveals why possessing things cannot satisfy:
``self-consciousness achieves its satisfaction only in another self-consciousness''
\parencite[\S 175]{HegelPhenomenology1977}. The Master gets obedience but finds it empty---coerced
submission is not free recognition \parencite[\S\S 187--189]{HegelPhenomenology1977}. Truth
lies with work that transforms the world into a space of mutual acknowledgment
\parencite[\S 196]{HegelPhenomenology1977}. Koj{\`e}ve radicalizes this: human desire is
``desire of another's desire'' \parencite[p.~6]{KojeveIRH1980}---we want to be desired/recognized,
not merely to possess.

Applied to the film, this explains the hollowness of each ``win.'' Willard secures objectives,
but objectives do not recognize him. The currency is wrong: dominion over things and
submissions cannot purchase what human desire seeks (free acknowledgment from an equal). The
mission-form systematically forecloses recognition by reducing others to obstacles or
instruments. Hence ``getting what one wants'' delivers everything except what was unconsciously
sought. Hegel and Koj{\`e}ve converge with Levinas's asymmetry and Beauvoir's reciprocity from
a different angle: all four insist the other's freedom cannot be bracketed without voiding
satisfaction.

\phantomsection
\subsubsection*{Conditions for Fulfillment That Satisfies}
\addcontentsline{toc}{subsubsection}{Conditions for Fulfillment That Satisfies}

The preceding analyses diagnose how fulfillment punishes. What would allow it to satisfy
instead? The debates converge on systematic requirements that the film's mission violates at
every turn:

\textbf{(1) Anti-instrumentality (Kant, Beauvoir, Levinas):} Treat persons as ends, will
others' freedom, respect the face's prohibition.

\textbf{(2) Anti-bad-faith (Sartre, Heidegger):} Own the project as freely chosen, not as
``what one does.''

\textbf{(3) Anti-totalization (Hegel, Kojève):} Seek recognition through work that founds a
common world, not mastery that silences.

\textbf{(4) Anti-absolutization (Kierkegaard, Dostoevsky):} Do not stake the self's ground on
a finite project; preserve agency's measure.

\textbf{(5) Anti-stasis (Nietzsche, Camus):} Revalue maxims when outcomes strip rhetoric; live
without appeal to final vindication.

The mission fails every test. It treats persons as means, disguises choice as necessity, seeks
dominion not recognition, absolutizes a finite end, and never transvalues. That it
``succeeds'' procedurally is precisely why it punishes existentially. The line's two halves
lock: the world delivers what systems can deliver; the will discovers that delivery was not
what it needed.

\phantomsection
\subsubsection*{Implications for the Film's Moral Thesis}
\addcontentsline{toc}{subsubsection}{Implications for the Film's Moral Thesis}

These philosophical debates constrain what Willard's line can mean. The film's structure---a
journey where every success generates new lack, where efficient means yield moral emptiness,
where completion does not redeem---confirms the convergent diagnosis across the traditions.
Whether the vocabulary is Schopenhauer's pendulum, Sartre's bad faith, Kant's heteronomy,
Levinas's totality, or Hegel's empty mastery, the pattern holds: getting what one wants
exposes the wanting's misdirection.

Yet the philosophers also preserve hope, if severely qualified. Kant's duty, Beauvoir's
reciprocity, Levinas's asymmetrical responsibility, Nietzsche's transvaluation, Camus's
lucidity without appeal---each offers a discipline for willing differently. The film refuses
this path. \hyperref[scene:assassination]{Willard's final silence} is not conversion but
paralysis. He has seen the mirror but
cannot alter what it shows. The essay thus reads the film as tragedy in the philosophers'
sense: not the defeat of a good will by external forces, but the exposure of a will whose very
structure ensured that fulfillment would punish. The line is not wisdom but epitaph.

