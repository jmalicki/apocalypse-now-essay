\phantomsection
\section*{IV. Western Philosophy: From Enlightenment Through Idealism to Existentialism}
\addcontentsline{toc}{section}{IV. Western Philosophy}
\label{sec:iv-western-philosophy}

Modern Western philosophy reads Willard's line not as moral judgment or causal sequence but as
an experiment in self-disclosure: when desire is granted its object, what is revealed about the
freedom that desired it? From the Enlightenment's architectonics of will (Kant) through German
Idealism's dialectics of recognition (Schopenhauer, Hegel) to existentialism and phenomenology's
accounts of project, despair, and finitude (Kierkegaard, Dostoevsky, Sartre, Beauvoir, Camus,
Heidegger, Levinas), each thinker supplies a distinct optic through which ``getting what one
wants'' becomes punitive—not because satisfaction is withheld, but because it exposes a
structure (metaphysical, existential, ethical) that the will misconceived or refused. The
following twelve analyses trace that exposure across the tradition.

\input{section-iv/IV_Kant_content.tex}
\pagebreak[2]
\input{section-iv/IV_Schopenhauer_content.tex}
\pagebreak[2]
\input{section-iv/IV_Hegel_content.tex}
\pagebreak[2]
\input{section-iv/IV_Nietzsche_content.tex}
\pagebreak[2]
\input{section-iv/IV_Kierkegaard_content.tex}
\pagebreak[2]
\phantomsection
\subsection*{Dostoevsky: Independent Desire, Anti-Mechanism, and Agency That Eats Itself}
\addcontentsline{toc}{subsection}{Dostoevsky: Independent Desire and Agency}
\label{ssec:iii-dostoevsky}
Fyodor Dostoevsky's \emph{Notes from Underground} is the classic anatomy of a will that prefers
independence to well-being—``man only wants independent desire, whatever that independence may
cost'' \parencite[p.~131]{DostoevskyNFU1994}. The underground man's most scandalous claim—``To
hell with two times two makes four!'' \parencite[p.~129]{DostoevskyNFU1994}—is not
anti-arithmetic; it is anti-mechanism in human affairs. He rejects any calculus in which
rational prediction, utility, or institutional procedure would close the space of spontaneous
willing. Read in this key, the line ``Everyone gets everything he wants'' marks not prosperity
but a world well-stocked with mechanisms that deliver objects on demand; ``for my sins I got
one'' acknowledges the price of willing agency itself within such machinery.

The underground man's revolt targets the dream that human conduct can be rendered scientific—that
motives can be predicted and optimized so that ``good'' outcomes follow from the right levers
\parencite[pp.~120--132]{DostoevskyNFU1994}. His point is not that people are irrational, but
that personhood includes a residual freedom that will ``assert itself'' against the system,
even destructively, simply to prove it exists \parencite[pp.~129--132]{DostoevskyNFU1994}.
This is why a ``crystal palace'' of perfect provisions would provoke sabotage; the human being,
he insists, will sometimes choose what is harmful to demonstrate authorship. The
\hyperref[scene:briefing]{dossier room's} hygienic proceduralism—clarity of ends, chain of
command, calibrated means—inhabits precisely the rational order the underground man distrusts.
Willard's wanting a mission is not a longing for certainty alone; it is the chance to act, to
break the inertia of \hyperref[scene:saigon-opening]{Saigon's aimlessness}, even if the action
risks moral injury. The ``sin''
is that the system can supply just such occasions and call the result necessary.

Dostoevsky's dialectic also explains the peculiar pleasure the underground man takes in acting
against his own interest: ``the most advantageous advantage'' is sometimes the freedom to choose
what is not advantageous \parencite[pp.~129--131]{DostoevskyNFU1994}. Agency is not measured by
outcomes but by the felt authorship of choice. That is why, on his account, a rational program
that secures only beneficial outcomes is degrading: it would reduce a man to a ``piano key'' on
which nature (or the institution) plays \parencite[pp.~115--120]{DostoevskyNFU1994}.
The \hyperref[scene:upriver-journey]{upriver insistence on continuing}, whatever the evidence or
cost, might \emph{seem} like a defense of non-instrumentality—``I choose to persist''—but
Dostoevsky's dialectic cuts both ways. If the persistence merely executes what the institution
already scripted, then the agent \emph{is} the piano key, not the player. Fulfillment punishes
because the moment the system grants the mission, the space of ``independent desire'' shrinks
into the execution of a mechanism that now owns the storyline.

Yet Dostoevsky's insight is double-edged. The underground man's independence is real, but it
corrodes itself when it refuses any measure beyond negation. He confesses to relishing
humiliation and spite, to savoring ``the sweetly painful pleasure'' of acting against himself
\parencite[pp.~108--115]{DostoevskyNFU1994}. This is agency that proves itself by injury.
When procedure yields horror (the \hyperref[scene:sampan]{sampan}), the rational mechanism has
plainly failed; but if
the next choice is animated only by the need to keep asserting agency—``I go on because I go
on''—the will ratifies the same emptiness the underground man inhabits. ``Everyone gets
everything he wants'' then describes a trap: the institution gets its obedient executor; the
agent gets the feeling of authorship; neither gets a norm by which the act could be vindicated.

Dostoevsky also anticipates what we might call moral theater: the staging of motives after the
fact to render destructive agency palatable. The underground man is merciless about his own
self-narration—confessing how quickly the ego invents edifying reasons for what was, in truth,
caprice or spite \parencite[pp.~103--107]{DostoevskyNFU1994}. This maps onto the rhetoric that
frames the assignment as sanitary necessity. The mask of moral purification (``remove an
aberration'') converts the hunger to act into an edifying plot; fulfillment then functions as
exposure when, at the end, the narrative yields no enlargement of soul. The confession ``for my
sins I got one'' reads, in Dostoevskian terms, as the recognition that the story was a postscript
to the will to act, not its ground.

A further Dostoevskian thread concerns responsibility under conditions of determinism. The
underground man refuses to let causal explanation excuse him: even if he can trace motives, he
will not permit the explanation to replace ownership \parencite[pp.~109--113]{DostoevskyNFU1994}.
This refusal illuminates the post-fulfillment chill: once the mission is complete, the agent
cannot hide in causal chains (``orders,'' ``procedure,'' ``necessity'') without committing the
very self-abdication he despises. Fulfillment punishes by removing alibis: the deed is done;
the authorship is mine.

Finally, Dostoevsky's anti-mechanism clarifies why the film's most efficient scenes are its most
disturbing. The underground man's nightmare is not chaos but perfect order—an order so seamless
it leaves no room for non-instrumental choice. Where everything ``works,'' the human residue can
only show itself by breaking the system or by converting obedience into a performance of will.
In either case, getting what one wanted discloses a deficit: agency defended merely as
independence becomes self-consuming. The sentence's halves therefore lock: the world can indeed
deliver the occasion to act (``everyone gets\ldots''), but the one who wanted agency itself
discovers, upon receiving it, that agency without a measure is indistinguishable from compulsion
in disguise—hence ``\ldots for my sins I got one.''

\pagebreak[2]
\input{section-iv/IV_Sartre_content.tex}
\pagebreak[2]
\phantomsection
\subsection*{Beauvoir: Reciprocity as the Form of Authentic Freedom}
\addcontentsline{toc}{subsection}{Beauvoir: Reciprocity and Authentic Freedom}
\label{ssec:iii-beauvoir}
Simone de Beauvoir's central thesis is that freedom is relational: my freedom is authentic only as a
practice that wills the freedom of others \parencite[p.~73]{Beauvoir1976}. This follows from
her ontology of ambiguity: human existence is at once facticity and transcendence, and meaning
is co-authored in a shared world \parencite[pp.~9--14, 24--30]{Beauvoir1976}. A project that
systematically reduces others to means contradicts the very structure of freedom it claims to
exercise. In this light, ``Everyone gets everything he wants'' is ethically indeterminate until
we ask whether the wanting included the other's freedom; ``\ldots and for my sins I got one''
reads as the moment the granted project exposes that it did not.

Beauvoir distinguishes authentic from inauthentic willing: the former embraces ambiguity and
seeks ``situations'' where others can transcend; the latter flees ambiguity by freezing others
into functions \parencite[pp.~70--76, 134--145]{Beauvoir1976}. Authenticity is not benevolence
but method: to pursue ends in a way that enlarges co-agency. This supplies a criterion the film
keeps failing. The \hyperref[scene:briefing]{Saigon briefing} frames action as administrative
necessity; its language (sanitation, dossiers) predetermines an inauthentic mode of encounter in
which faces will appear as obstacles or instruments. That mode is not corrected by later
``successes''; it is confirmed by them.

Beauvoir's account of oppression makes this failure legible. Oppression is not just harm; it is
the organization of the world so that another's transcendence can appear only as a threat
\parencite[pp.~85--91, 157--161]{Beauvoir1976}. Where a project's telos presupposes such
organization, efficiency deepens guilt. The \hyperref[scene:sampan]{sampan inspection} is
exemplary: even before the fatal moment, the protocol treats persons as risk variables in a
supply chain. Beauvoir's
question is not whether force is ever permissible; it is whether the mode of action keeps open
a horizon in which the other can still be a source of meaning
\parencite[pp.~139--147, 164--173]{Beauvoir1976}. Here the very grammar of the check—its
anticipations, its allowable responses—has already closed that horizon.

Beauvoir recasts justification in terms of world-building: deeds are justified when they found
a common world, i.e., when they set up institutions or practices through which others can also
project ends \parencite[pp.~145--153]{Beauvoir1976}. Measured by that standard, the film's
repeating structures—\hyperref[scene:kilgore-beach]{Kilgore's spectacle of sovereignty},
\hyperref[scene:do-lung-bridge]{Do Lung Bridge} rebuilt nightly by nameless hands—show action
that circulates without founding. The spectacular will and the
faceless mechanism are two faces of the same inauthenticity: each consumes the other's
transcendence for its own continuity. ``Everyone gets everything he wants'' here names only the
reliability of means; it says nothing about the world those means build.

Beauvoir insists that constraint does not absolve; it conditions responsibility. Authentic
freedom exploits cracks in necessity to remake situations toward reciprocity
\parencite[pp.~34--42]{Beauvoir1976}. Hence the ethical failure is clearest where things
``work.'' When procedures function smoothly and no revision follows—no new practice that
protects faces, no altered maxim that includes co-agency—success becomes self-indicting. This is
why the confession's sting is specifically Beauvoirian: ``\ldots for my sins I got one''
acknowledges that the mission's efficient fulfillment revealed what its end had never
embraced—the other's freedom.

Finally, her ethics reframes the film's terminal clarity. For Beauvoir, one cannot sanitize
ambiguity; every deed risks harm. But she denies the alibi of purity: the right response to risk
is vigilant reciprocity, not resignation \parencite[pp.~139--147]{Beauvoir1976}. If a project's
form cannot be made reciprocal, authenticity demands refusal or re-foundation. In a setting
where refusal is not chosen and re-foundation never occurs, the two halves of Willard's line
align perfectly with Beauvoir's verdict: the world can indeed deliver the object of desire
(``everyone gets\ldots''), and precisely that delivery discloses that what was desired was
sovereignty without co-agency (``\ldots for my sins I got one'').

\pagebreak[2]
\phantomsection
\subsection*{Camus: Absurd Lucidity, Revolt ``Without Appeal,'' and Completion as Knowledge
	(Not Meaning)}
\addcontentsline{toc}{subsection}{Camus: Absurd Lucidity and Revolt}
\label{ssec:iii-camus}
Albert Camus begins with a refusal of consolations. ``There is but one truly serious philosophical
problem, and that is suicide'' \parencite[p.~3]{CamusMyth1991}. The claim sets the tone: the
question is not whether life can be made coherent, but whether one can live honestly when it
cannot. The absurd is the name for this standoff—``born of the confrontation between the human
need and the unreasonable silence of the world'' \parencite[p.~28]{CamusMyth1991}. Read against
Willard's line, ``Everyone gets everything he wants,'' the absurd warns that the delivery of
objects and outcomes has no built-in power to answer the need that generated them. ``\ldots And
for my sins I got one'' sounds, in Camus's vocabulary, like an onset of lucidity: completion
gives knowledge, not meaning.

Camus is suspicious of what he calls philosophical suicide—any leap (religious, metaphysical,
or ideological) that smuggles meaning back in after the absurd has been recognized
\parencite[pp.~53--58]{CamusMyth1991}. To live ``without appeal'' is to refuse that leap
\parencite[p.~54]{CamusMyth1991}. Much of the film's rhetoric—dossier certainties, the mission's
hygienic narrative—functions as an appeal to an order that would dissolve ambiguity. As the
journey upriver strips those narratives away, the world retains its ``unreasonable silence,''
yet the project continues. Camus would say: the willing, deprived of its fictions, is now
exposed to the task of revolt—not overthrow, but a ``permanent confrontation'' with
meaninglessness \parencite[p.~55]{CamusMyth1991}. If the revolt does not transvalue its maxims,
completion will punish by revealing that the act was, after all, an appeal in disguise.

Camus reframes fulfillment by recoding value as lucidity, freedom, passion—the three modalities
of living the absurd \parencite[pp.~54--71]{CamusMyth1991}. Lucidity means remaining with what
the world actually grants; freedom means recognizing that, if meanings are not given, our
projects are ours without metaphysical guarantees; passion means intensifying experience rather
than seeking a terminal sanction. In this light, the film's most efficient scenes (where things
work) are its least meaningful. The Do Lung Bridge cycle—building by day, destruction by
night—reads like the myth of Sisyphus in military dress: strenuous labor without appeal, the
task's perfection indifferent to significance. Camus's verdict on Sisyphus—``One must imagine
Sisyphus happy'' \parencite[p.~123]{CamusMyth1991}—does not romanticize toil; it claims that
honesty about the task's finitude can be a site of dignity. The catch is that such dignity
requires abandoning the promise that the task will redeem. Where Willard's project is still
mortgaged to a redemptive story (purge the aberration, restore sense), its successful completion
must recoil as knowledge that no redemption follows.

The figure most tempted by appeal is the one who seeks a final verdict on existence. Camus's
polemic targets precisely that longing: the desire to seal the world with an ultimate judgment
that would still the need to will \parencite[pp.~53--60]{CamusMyth1991}. In this register,
Kurtz's pronouncement—``the horror''—behaves like a metaphysical seal, an ultimate word that
would turn lucidity into law. Camus would demur: the absurd forbids the last word. To honor the
absurd is to continue acting without that word—no appeal to a transcendent rule, no enthronement
of the self as tribunal. ``Everyone gets everything he wants'' thus becomes, for Camus, a
litmus: if what one wanted was an ultimate exoneration, getting it will read as punishment—the
world remains silent.

Camus's portrait of the absurd hero helps explain the tone of the confession. The absurd hero
does not seek to solve the absurd; he keeps faith with it through measured revolt
\parencite[pp.~54--60, 121--123]{CamusMyth1991}. He does not deny limits, and he does not
pretend his acts are guaranteed meaning by a higher court. Where the mission-form equates
success with justification, Camus severs that link. After the deed, what remains is clarity:
we know what the world is (silent), what we are (beings who will without guarantee), and what
action can be (finite, accountable, unredeemed). If the project has been conducted under the
illusion that completion = meaning, then completion unveils the illusion. ``For my sins I got
one'' is exactly that unveiling—the point at which the will, faced with the absurd, loses its
alibi.

Finally, Camus's injunction—live ``without appeal''—tightens the essay's thesis. If the world
can reliably supply objects for our projects (hence ``everyone gets\ldots''), and if those
projects often carry tacit appeals (to necessity, to cleansing narratives, to final judgments),
then the punitive feel of fulfillment is simply the return of the real: the object arrives; the
appeal fails; lucidity remains. The task, if there is one, is to convert willing from a demand
for consummation into a discipline of revolt—a way of acting that neither lies about meaning
nor abdicates it. Absent that conversion, getting what one wants will continue to accuse the
will that wanted it.

\pagebreak[2]
\phantomsection
\subsection*{IV.10—Heidegger: Finitude, Anticipatory Resoluteness, and Why Completion Is
	Ontologically Out of Reach}
\addcontentsline{toc}{subsection}{IV.10—Heidegger: Finitude and Resoluteness}
\label{ssec:iii-heidegger}
Martin Heidegger's account of existence (Dasein) makes completion a category mistake. Dasein is
essentially being-possible—a projecting that never coincides with itself as a finished thing;
its wholeness is disclosed only in being-toward-death \parencite[pp.~279--311]{HeideggerBT1962}.
Death is not (primarily) a future event to be scheduled but the ownmost, nonrelational
possibility that individualizes Dasein now, stripping away the illusions of totalization
\parencite[pp.~294--307]{HeideggerBT1962}. Thus, any project that promises narrative wholeness—that
a mission will ``make it come together''—misreads existence. When Willard says ``Everyone gets
everything he wants,'' the Heideggerian gloss is brutal: the world may indeed supply objects and
tasks, but existence is not something an object can finish. ``\ldots And for my sins I got one''
is the moment the project's alleged telos collides with finitude.

Heidegger's analysis of everydayness and the They (\emph{das Man}) clarifies why projects so
easily wear the mask of necessity. In average everydayness, Dasein takes over possibilities ``as
one does,'' letting anonymous norms dictate what counts as urgent, clean, or right
\parencite[pp.~149--168]{HeideggerBT1962}. The \hyperref[scene:briefing]{Saigon briefing}'s
procedural tone—dossiers, signatures, the grammar of sanitation—exemplifies this absorption in
\emph{das Man}: the mission
shows up as what ``one'' does when a file reads anomalous. To take it up as such is not yet
resolute choice; it is fallenness into the ready-made interpretation. Fulfillment then
``punishes'' by disclosing that the accomplished sequence was never a route to owned wholeness;
it was a they-self rhythm all along.

The film's temporality maps onto Heidegger's account of ecstatic time. Dasein's temporality is
not a string of nows but an ``ahead-of-itself'' (future), already-in (past), and being-alongside
(present) \parencite[pp.~373--383]{HeideggerBT1962}. The Do Lung Bridge cycle—construction by
day, destruction by night—stages a caricature of inauthentic time: a serial present that never
gathers. Anticipatory resoluteness does not end such cycles; it interprets them soberly by
owning death as the limit that prevents totalization \parencite[pp.~307--311]{HeideggerBT1962}.
By this light, the climactic ``success'' cannot heal the fracture; it can only remove the alibi
that failure once provided. The felt judgment of the line is that clarity: the project is
complete and therefore unable to hide the truth that existence cannot be.

Heidegger's conscience and guilt intensify the point. Conscience ``calls'' Dasein from \emph{das
	Man} to its ownmost possibility; guilt (\emph{Schuld}) names not juridical fault but
being-the-basis of a nullity—that our thrown projection always leaves something out and cannot
guarantee innocence \parencite[\S\S 57--60, pp.~311--354]{HeideggerBT1962}. When procedures run
perfectly (the sampan inspection as ``by the book'') and still yield devastation, what is
revealed is not only moral failure but ontological mismatch: the attempt to secure existential
rightness via technical closure. Anticipatory resoluteness would require owning that mismatch,
not masking it with narratives of cleansing. The confession—``for my sins I got one''—is a
resolute sentence in this sense: it drops the promise of narrative wholeness and accepts
finitude as the horizon that renders completion impossible.

\pagebreak[2]
\input{section-iv/IV_Levinas_content.tex}
\pagebreak[2]
\phantomsection
\subsection*{Koj{\`e}ve: Desire of Desire, History as Recognition, and Why Mission-Form
	Fulfillment Recurs as Lack}
\addcontentsline{toc}{subsection}{Kojève: Desire of Desire and Recognition}
\label{ssec:iii-kojeve}
Alexandre Koj{\`e}ve radicalizes Hegel's insight in anthropological terms: human desire is
``desire of another's desire''—a need to be desired/recognized by a free other
\parencite[p.~6]{KojeveIRH1980}.
The object mediates this relation, but it is not the final aim. Hence the lordship/bondage
dialectic reads, in Koj{\`e}ve's gloss, as the matrix of history: the Master obtains things
(and obedience) but not the recognition that would satisfy a human desire; the Slave, through
fearful work, transforms the world and, in so doing, becomes the bearer of truth
\parencite[pp.~27--34, 158--164]{KojeveIRH1980}. If the film's world can reliably deliver
missions and outcomes—``everyone gets everything he wants''—what it cannot deliver, by those
same means, is the desire of the other freely given.

Koj{\`e}ve's portrait of the Master maps neatly onto the mission-form that prizes clean
execution and visible effects. The Master ``gets what he wants,'' but his world is populated by
things and submissions, not by interlocutors who can confirm him
\parencite[pp.~27--34]{KojeveIRH1980}. In such a regime, success increases dependence on further
success, because each attainment fails to supply the missing confirmation. The appetite becomes
serial: new targets, new proofs, new shows of power. This is the historical engine that drives
the ``pendulum'' of operations: each completion—however perfect—returns as renewed lack, not
because the agent is psychologically thin, but because the form of fulfillment excludes the kind
of acknowledgment that could end the sequence.

By contrast, Koj{\`e}ve sees the Slave's work as the slow route to recognitive stability: work
shapes a common world in which self and other can appear to each other as free
\parencite[pp.~158--164]{KojeveIRH1980}. That is why he can speak of the ``end of history'' as
a horizon of universal recognition, not maximal accumulation
\parencite[pp.~158--164]{KojeveIRH1980}. Measured against this horizon, the upriver procedures
create no institutions of mutual address; they routinize asymmetry. Even the final act—eliminating
the figure who has refused the institution—seeks restoration of order without creating the space
in which recognition could be mutual. Fulfillment thus returns as judgment: the very evidence of
technical success is the evidence that the recognitive aim was never in view.

Koj{\`e}ve's reading also explains the peculiar tone of mastery's self-knowledge. Once the mask
of ``truth'' and ``necessity'' falls, the Master must either convert—accept that what he wanted
cannot be had by command—or double down, seeking ever more unchallengeable evidence of
sovereignty. The line ``\ldots for my sins I got one'' registers the first path as insight
without conversion: one sees that the delivery of ends cannot deliver recognition, yet one has
already acted in the Master's grammar. The punishment is temporal: the completed project does
not close history; it lengthens the sequence of unsatisfying confirmations.

In Koj{\`e}ve's terms, then, the film's world is historically stuck between mastery's emptiness
and the slow, dangerous labor that could found a recognitive order. ``Everyone gets everything
he wants'' names a high-functioning apparatus for producing things and effects; ``\ldots and for
my sins I got one'' names the self-knowledge that, within that apparatus, the human desire—desire
for the other's free acknowledgment—was never addressed. What returns is not failure but the
truth about what was really wanted.

\pagebreak[2]
\phantomsection
\subsection*{IV.13—Comparative Discussion: Convergences and Tensions}
\addcontentsline{toc}{subsection}{IV.13—Comparative Discussion}
\label{ssec:iv-comparative-discussion}

The preceding twelve analyses reveal not a single philosophical account of Willard's line but 
a field of competing and complementary diagnoses. Some tensions are productive: they refine 
the verdict by forcing precision about what kind of failure fulfillment exposes. Other 
convergences are striking: across metaphysical, existential, and ethical vocabularies, the 
philosophers agree that ``getting what one wants'' punishes because it reveals the will's 
prior orientation or structure. What follows maps the key debates and their implications for 
reading the film.

\subsubsection*{Is Recurrence Curse or Opportunity? Schopenhauer, Nietzsche, Camus}

Schopenhauer's phenomenology is precise: satisfaction ``at once makes room for a new one,'' so 
life swings ``between pain and boredom'' \parencite[pp.~312, 319]{SchopenhauerWWR1969}. The 
upriver sequence confirms this---each checkpoint delivers relief that immediately becomes 
renewed lack. Nietzsche objects that such pessimism misconstrues the task: recurrence is not a 
curse if the will has the courage to revalue itself, to create new measures rather than repeat 
old consumption \parencite[\S\S 34, 283]{NietzscheBGE1990}. What corrodes willing is not its 
repetition but its dishonesty---domination disguised as truth.

The film tests both claims. Nietzsche is right that the mission wears the mask of cognition 
(dossiers, rationality, surgical necessity), and that mask licenses command. Yet the narrative 
never transvalues. After the sampan, after the bridge, no new measure emerges. Schopenhauer's 
pendulum reasserts itself: fulfillment disenchants, and the will swings back to lack. Camus 
cuts between them with lucidity: even a creative will must live ``without appeal,'' and no 
final sanction redeems completion \parencite[pp.~28, 54, 121--123]{CamusMyth1991}. The film 
sides with Camus---the world delivers objects, but what returns is knowledge, not meaning. The 
convergence is grim: whether the problem is metaphysical mechanism (Schopenhauer), dishonest 
transvaluation (Nietzsche), or absurdist silence (Camus), fulfillment cannot heal the will.

\subsubsection*{Does Success Ever Vindicate? Kant, Nietzsche, Kierkegaard}

Kant denies that outcomes certify worth: the good will is ``good \ldots\ in itself,'' not 
``because of what it effects'' \parencite[p.~27]{KantGroundwork1996}. The sampan inspection, 
executed flawlessly, still fails the humanity constraint---persons treated merely as means 
cannot be rescued by efficient procedure \parencite[pp.~36--37]{KantCPrR1996}. Nietzsche 
counters that Kantian morality can itself be a will to command in disguise: the ``desire for 
`truth''' becomes a tool of domination when it pretends neutrality \parencite[\S 34]{NietzscheBGE1990}.

Both critiques land. The briefing room stages Nietzsche's suspicion---rational necessity as 
rhetorical cover for institutional will. Yet Kant supplies the verdict that still condemns: 
even if we unmask the rhetoric, the maxim (eliminate persons designated as obstacles) fails 
universalizability. Kierkegaard adds an internal dimension: even if the maxim somehow passed, 
absolutizing a finite project as the self's ground thickens despair 
\parencite[pp.~69--83]{KierkegaardSUD1980}. The triple pressure is severe: success cannot 
vindicate (Kant), unmasking cannot excuse (Nietzsche), and structural rightness cannot cure 
misrelation (Kierkegaard). ``For my sins I got one'' thus reads as the removal of every alibi.

\subsubsection*{Freedom as Burden or Condemnation? Sartre, Beauvoir, Dostoevsky}

Sartre's radical claim---we are ``condemned to be free'' \parencite[pp.~34--36]{SartreBN2003}---makes 
every project an authorship with absolute responsibility. The mission cannot be blamed on 
orders or necessity; it is freely chosen and owned. Beauvoir specifies the ethical constraint: 
freedom is authentic only when it wills the freedom of others \parencite[p.~73]{Beauvoir1976}. 
A project that systematically instrumentalizes contradicts freedom's structure. Together, they 
condemn the mission on two grounds: it is bad faith (Sartre) and it violates reciprocity 
(Beauvoir).

Dostoevsky complicates this by insisting that agency itself can be the disease. The 
Underground Man wants ``independent desire, whatever that independence may cost'' 
\parencite[p.~131]{DostoevskyNFU1994}---he would rather act destructively than be a ``piano 
key'' in a rational system. The mission-form threatens precisely this: absorption into 
mechanism. Yet agency defended as pure negation (``I go on because I refuse the system'') 
corrodes itself into spite. The film stages both traps: obedience performed as authorship 
(bad faith) and continuation without measure (self-consuming agency). Fulfillment exposes that 
neither path preserves genuine freedom.

\subsubsection*{Completion as Ontological Error: Sartre, Heidegger}

Sartre and Heidegger converge that completion is impossible, but their reasons differ. For 
Sartre, the \emph{pour-soi} is perpetual transcendence; it ``is what it is not and not what 
it is'' \parencite[pp.~100--110]{SartreBN2003}. The hidden ``project to be God''---to fuse 
facticity and transcendence into self-grounding plenitude---cannot succeed 
\parencite[pp.~586--604]{SartreBN2003}. Heidegger roots the error differently: Dasein's 
wholeness is disclosed only in being-toward-death, which individualizes now and strips 
totalization-fantasies \parencite[pp.~294--307]{HeideggerBT1962}. Where Sartre diagnoses a 
wish for ontological closure, Heidegger diagnoses fallenness into \emph{das Man}---the fantasy 
that doing ``what one does'' could yield authentic wholeness \parencite[pp.~149--168]{HeideggerBT1962}.

The difference matters for interpretation. Sartre reads the mission's end as exposing bad 
faith; Heidegger reads it as removing the alibi of average everydayness. Both see the vacuum 
after clean procedures, but Sartre emphasizes free choice's responsibility, while Heidegger 
emphasizes the they-self's inauthenticity. The film allows both: Willard freely chose the 
project (Sartre) and absorbed it as ``what one does'' (Heidegger). Fulfillment punishes both 
ways.

\subsubsection*{Reciprocity or the Face's Command? Beauvoir, Levinas}

Beauvoir and Levinas both condemn instrumental projects, but from different starting points. 
Beauvoir builds the other into freedom's structure: authentic willing must will the other's 
freedom \parencite[p.~73]{Beauvoir1976}. Projects are justified when they found situations 
where others can transcend \parencite[pp.~145--153]{Beauvoir1976}. Levinas argues this comes 
too late: the face's prohibition (``Thou shalt not kill'') precedes all projects and resists 
assimilation into reciprocal frameworks \parencite[pp.~199, 21--24]{LevinasTI1969}. Ethics is 
asymmetrical---I am responsible for the other beyond contract.

The tension is productive. Beauvoir's framework can critique the mission's failure to build a 
common world; Levinas can indict the very mode of approach (dossier, protocol) as a refusal of 
the face. Beauvoir worries Levinas risks ethical purity without political efficacy; Levinas 
warns Beauvoir's world-building easily re-totalizes. The film confirms both critiques: 
procedures flatten alterity (Levinas) and successes never found shared institutions 
(Beauvoir). ``Everyone gets everything he wants'' names efficiency without co-agency; ``for my 
sins I got one'' marks the double failure.

\subsubsection*{Objects or Recognition? Hegel, Kojève}

Hegel's master-slave dialectic reveals why possessing things cannot satisfy: 
``self-consciousness achieves its satisfaction only in another self-consciousness'' 
\parencite[\S 175]{HegelPhenomenology1977}. The Master gets obedience but finds it empty---coerced 
submission is not free recognition \parencite[\S\S 187--189]{HegelPhenomenology1977}. Truth 
lies with work that transforms the world into a space of mutual acknowledgment 
\parencite[\S 196]{HegelPhenomenology1977}. Koj{\`e}ve radicalizes this: human desire is 
``desire of another's desire'' \parencite[p.~6]{KojeveIRH1980}---we want to be desired/recognized, 
not merely to possess.

Applied to the film, this explains the hollowness of each ``win.'' Willard secures objectives, 
but objectives do not recognize him. The currency is wrong: dominion over things and 
submissions cannot purchase what human desire seeks (free acknowledgment from an equal). The 
mission-form systematically forecloses recognition by reducing others to obstacles or 
instruments. Hence ``getting what one wants'' delivers everything except what was unconsciously 
sought. Hegel and Koj{\`e}ve converge with Levinas's asymmetry and Beauvoir's reciprocity from 
a different angle: all four insist the other's freedom cannot be bracketed without voiding 
satisfaction.

\subsubsection*{Minimal Conditions for Non-Punitive Fulfillment}

The debates yield negative constraints that any non-punitive project must satisfy:

\textbf{(1) Anti-instrumentality (Kant, Beauvoir, Levinas):} Treat persons as ends, will 
others' freedom, respect the face's prohibition.

\textbf{(2) Anti-bad-faith (Sartre, Heidegger):} Own the project as freely chosen, not as 
``what one does.''

\textbf{(3) Anti-totalization (Hegel, Kojève):} Seek recognition through work that founds a 
common world, not mastery that silences.

\textbf{(4) Anti-absolutization (Kierkegaard, Dostoevsky):} Do not stake the self's ground on 
a finite project; preserve agency's measure.

\textbf{(5) Anti-stasis (Nietzsche, Camus):} Revalue maxims when outcomes strip rhetoric; live 
without appeal to final vindication.

The mission fails every test. It treats persons as means, disguises choice as necessity, seeks 
dominion not recognition, absolutizes a finite end, and never transvalues. That it 
``succeeds'' procedurally is precisely why it punishes existentially. The line's two halves 
lock: the world delivers what systems can deliver; the will discovers that delivery was not 
what it needed.

\subsubsection*{Implications for the Film's Moral Thesis}

These philosophical debates constrain what Willard's line can mean. The film's structure---a 
journey where every success generates new lack, where efficient means yield moral emptiness, 
where completion does not redeem---confirms the convergent diagnosis across the traditions. 
Whether the vocabulary is Schopenhauer's pendulum, Sartre's bad faith, Kant's heteronomy, 
Levinas's totality, or Hegel's empty mastery, the pattern holds: getting what one wants 
exposes the wanting's misdirection.

Yet the philosophers also preserve hope, if severely qualified. Kant's duty, Beauvoir's 
reciprocity, Levinas's asymmetrical responsibility, Nietzsche's transvaluation, Camus's 
lucidity without appeal---each offers a discipline for willing differently. The film refuses 
this path. Willard's final silence is not conversion but paralysis. He has seen the mirror but 
cannot alter what it shows. The essay thus reads the film as tragedy in the philosophers' 
sense: not the defeat of a good will by external forces, but the exposure of a will whose very 
structure ensured that fulfillment would punish. The line is not wisdom but epitaph.


