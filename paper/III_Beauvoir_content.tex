\phantomsection
\subsection*{III.8—Beauvoir: Reciprocity as the Form of Authentic Freedom}
\addcontentsline{toc}{subsection}{III.8—Beauvoir: Reciprocity and Authentic Freedom}
\label{ssec:iii-beauvoir}
Simone de Beauvoir's central thesis is that freedom is relational: my freedom is authentic only as a
practice that wills the freedom of others \parencite[p.~73]{Beauvoir1976}. This follows from
her ontology of ambiguity: human existence is at once facticity and transcendence, and meaning
is co-authored in a shared world \parencite[pp.~9--14, 24--30]{Beauvoir1976}. A project that
systematically reduces others to means contradicts the very structure of freedom it claims to
exercise. In this light, ``Everyone gets everything he wants'' is ethically indeterminate until
we ask whether the wanting included the other's freedom; ``\ldots and for my sins I got one''
reads as the moment the granted project exposes that it did not.

Beauvoir distinguishes authentic from inauthentic willing: the former embraces ambiguity and
seeks ``situations'' where others can transcend; the latter flees ambiguity by freezing others
into functions \parencite[pp.~70--76, 134--145]{Beauvoir1976}. Authenticity is not benevolence
but method: to pursue ends in a way that enlarges co-agency. This supplies a criterion the film
keeps failing. The Saigon briefing frames action as administrative necessity; its language
(sanitation, dossiers) predetermines an inauthentic mode of encounter in which faces will appear
as obstacles or instruments. That mode is not corrected by later ``successes''; it is confirmed
by them.

Beauvoir's account of oppression makes this failure legible. Oppression is not just harm; it is
the organization of the world so that another's transcendence can appear only as a threat
\parencite[pp.~85--91, 157--161]{Beauvoir1976}. Where a project's telos presupposes such
organization, efficiency deepens guilt. The sampan inspection is exemplary: even before the
fatal moment, the protocol treats persons as risk variables in a supply chain. Beauvoir's
question is not whether force is ever permissible; it is whether the mode of action keeps open
a horizon in which the other can still be a source of meaning
\parencite[pp.~139--147, 164--173]{Beauvoir1976}. Here the very grammar of the check—its
anticipations, its allowable responses—has already closed that horizon.

Beauvoir recasts justification in terms of world-building: deeds are justified when they found
a common world, i.e., when they set up institutions or practices through which others can also
project ends \parencite[pp.~145--153]{Beauvoir1976}. Measured by that standard, the film's
repeating structures—Kilgore's spectacle of sovereignty, Do Lung Bridge rebuilt nightly by
nameless hands—show action that circulates without founding. The spectacular will and the
faceless mechanism are two faces of the same inauthenticity: each consumes the other's
transcendence for its own continuity. ``Everyone gets everything he wants'' here names only the
reliability of means; it says nothing about the world those means build.

Beauvoir insists that constraint does not absolve; it conditions responsibility. Authentic
freedom exploits cracks in necessity to remake situations toward reciprocity
\parencite[pp.~34--42]{Beauvoir1976}. Hence the ethical failure is clearest where things
``work.'' When procedures function smoothly and no revision follows—no new practice that
protects faces, no altered maxim that includes co-agency—success becomes self-indicting. This is
why the confession's sting is specifically Beauvoirian: ``\ldots for my sins I got one''
acknowledges that the mission's efficient fulfillment revealed what its end had never
embraced—the other's freedom.

Finally, her ethics reframes the film's terminal clarity. For Beauvoir, one cannot sanitize
ambiguity; every deed risks harm. But she denies the alibi of purity: the right response to risk
is vigilant reciprocity, not resignation \parencite[pp.~139--147]{Beauvoir1976}. If a project's
form cannot be made reciprocal, authenticity demands refusal or re-foundation. In a setting
where refusal is not chosen and re-foundation never occurs, the two halves of Willard's line
align perfectly with Beauvoir's verdict: the world can indeed deliver the object of desire
(``everyone gets\ldots''), and precisely that delivery discloses that what was desired was
sovereignty without co-agency (``\ldots for my sins I got one'').
