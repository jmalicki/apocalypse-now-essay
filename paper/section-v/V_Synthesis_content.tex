\phantomsection
\subsection*{V.6—The Historical Cage: Complicity Without Exit}
\addcontentsline{toc}{subsection}{V.6—The Historical Cage}
\label{ssec:v-historical-cage}

Where Section IV's philosophers diagnosed the will's structure as metaphysical, Section V's
critical theorists locate that structure within history. The cage is institutional, not
ontological. This distinction matters: one cannot escape by willing differently because the
cage produces the will that would escape it. Nor can incremental reform address systems that
fragment responsibility (Arendt), normalize violence (Foucault), and manufacture the
representational worlds that justify domination (Said).

The convergence of these critiques yields a single diagnostic: ``Everyone gets everything he
wants'' under modernity means the system reliably delivers what it taught you to want. ``For
my sins I got one'' means the delivery exposes complicity—not individual moral failure but
participation in structures that cannot be redeemed from within. Willard's flat affect, his
refusal of redemptive narrative, his final non-choice all confirm this. The mission worked
perfectly as procedure, and its perfection is the indictment.

The theorists disagree on strategy—Weber offers no exit, Fanon demands revolution, Arendt
preserves plural action, Foucault seeks micro-resistances—but they agree on the structure.
Willard's journey does not reveal timeless truths about desire; it reveals what desire becomes
within imperial bureaucracy. The will got what it wanted and discovered it was never its own.
