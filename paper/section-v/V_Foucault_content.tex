\phantomsection
\subsection*{Foucault: Discipline, Biopower, and the Internalized Gaze}
\addcontentsline{toc}{subsection}{Foucault: Discipline and Biopower}
\label{ssec:v-foucault}
Michel Foucault's \textit{Discipline and Punish} (1975) shows how modern power operates not
through sovereign violence but through normalization. Classical sovereignty killed to punish;
modern discipline produces docile, productive subjects by training their bodies and managing
their perceptions \parencite{FoucaultDiscipline1995}. The prison, the school, the barracks all
share a technique: surveillance, timetables, drills, examination. The subject internalizes the
gaze---learns to monitor, correct, and optimize itself. Power becomes invisible because it is
everywhere, threaded through the routines that constitute ``normal'' life.

Willard's training, files, and surveillance are not contingent backdrops; they \emph{are} the
conditions under which a mission can be willed, received, and fulfilled. He has been drilled,
monitored, filed. The \hyperref[scene:dossier-reading]{dossier that describes Kurtz} also
positions Willard: special forces, disciplined, reliable. To ``want a mission'' is already to
have been shaped by the disciplinary apparatus into a subject who experiences assignment as
relief. Foucault would say: the system does not coerce from outside; it trains desire from
within. Willard's wanting is the product of his normalization.

Foucault's later concept of \emph{biopower} extends this analysis to population-level
management: modern states do not just punish individuals; they optimize populations through
medicine, welfare, urban planning. Life itself becomes an object of
administration. The war machine in \textit{Apocalypse Now} exhibits this logic: soldiers are
resources to be deployed, casualties are statistics, ``body counts'' measure success. The
mission treats Kurtz's rogue sovereignty as a threat to this biopolitical order---he has
claimed the power over life and death that the state reserves for itself. Eliminating him
restores the state's monopoly on biopower.

In this frame, ``everyone gets everything he wants'' becomes sinister: the system produces the
conditions under which wanting aligns with administrative goals. Want a mission? The bureaucracy
has one ready. Want action, danger, meaning? The apparatus can supply all three, pre-packaged.
The punishment is not deprivation but the discovery that one's subjectivity has been colonized.
Foucault asks: where does resistance come from, if power produces the subject who would resist?
\parencite{FoucaultDiscipline1995}. Willard's journey offers no answer; it only shows the will
turned back on itself, completing the mission it was trained to want.

The \hyperref[scene:sampan]{sampan scene} is Foucault's nightmare made visible. The protocol
is rational, the procedure is disciplined, every soldier performs his role. Yet the outcome is
horror. Foucault would note that horror here is not a breakdown of discipline but its
perfection: the system worked, bodies
were managed, the threat was neutralized. Moral revulsion has no traction within disciplinary
logic. ``Getting what one wants'' (a \hyperref[scene:sampan]{secure perimeter, a checked
	boat}) exposes the vacancy of wanting shaped entirely by procedure.
