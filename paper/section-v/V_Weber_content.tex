\phantomsection
\subsection*{V.1—Weber: The Iron Cage and Vocation Without Grace}
\addcontentsline{toc}{subsection}{V.1—Weber: The Iron Cage}
\label{ssec:v-weber}
Max Weber's \textit{The Protestant Ethic and the Spirit of Capitalism} (1905) traces how
religious anxiety becomes worldly compulsion. Protestant theology, particularly Calvinist
predestination, creates unbearable uncertainty: am I among the elect? Unable to know, believers
seek signs of grace in worldly success---hard work, discipline, accumulation
\parencite{WeberProtestant2002}. But success never settles the question; it only demands more
proof. Eventually, the religious frame falls away, leaving what Weber calls the ``iron
cage''---a system of rationalized labor and bureaucratic control that persists without
transcendent justification \parencite{WeberProtestant2002}.

In Willard's world, this process is complete. The mission arrives as vocation without
salvation: a task that promises meaning through completion, yet completion yields only the next
task. ``I wanted a mission'' expresses the hunger for orientation that Weber diagnosed; ``for
my sins they gave me one'' acknowledges that the system can supply the form (orders,
objectives, structure) while withholding the substance (grace, vindication, rest). The
\hyperref[scene:upriver-journey]{upriver journey} becomes a secularized pilgrimage---discipline
and danger replacing prayer, but no divine mercy awaiting at the end.

The film's bridge sequences literalize the iron cage: construction by day, destruction by
night, soldiers who ``build it every night'' without purpose beyond the rhythm itself. No one
is in charge; the structure persists through its own momentum. Weber's warning rings through:
``specialists without spirit, sensualists without heart; this nullity imagines that it has
attained a level of civilization never before achieved'' \parencite{WeberProtestant2002}.
Willard's clinical narration, stripped of affect, embodies this nullity. Getting what he
wanted---a structured role, clear objectives---reveals the cage, not freedom.
