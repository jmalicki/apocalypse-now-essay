\phantomsection
\subsection*{Adorno, Horkheimer \& Marcuse: Instrumental Reason and Repressive Desublimation}
\addcontentsline{toc}{subsection}{Adorno, Horkheimer \& Marcuse}
\label{ssec:v-adorno-horkheimer-marcuse}
Theodor Adorno and Max Horkheimer's \textit{Dialectic of Enlightenment} (1944) traces the
Enlightenment's fatal reversal: the project to liberate humanity from myth through reason
becomes a new form of domination. Reason, they argue, degrades into \emph{instrumental
	reason}---a logic that reduces all things to calculable means for given ends, with no capacity
to evaluate the ends themselves \parencite{AdornoHorkheimer2002}. Nature is mastered through
science, but the same logic is turned on human beings: society becomes a machine for
processing persons as inputs. The result is what they call the ``totally administered world,''
where freedom is marketed as choice among pre-determined options.

The \hyperref[scene:briefing]{briefing scene} in \textit{Apocalypse Now} epitomizes this
structure. Kurtz is presented as a technical problem requiring a calculated solution. The
language is sanitized: ``terminate with extreme prejudice.'' No one deliberates whether
assassination is the right \emph{kind} of
response; instrumental reason has already framed the question as ``how to eliminate
efficiently.'' Willard receives maps, files, objectives---all the rational infrastructure---but
no invitation to ask whether the end (restoring bureaucratic order by killing a rogue officer)
is itself defensible. The system \emph{works}; that is its only vindication. When Willard says
``for my sins they gave me one,'' he voices the recognition that he wanted precisely this: to
be absorbed into a rational structure that would relieve him of radical freedom.

Adorno and Horkheimer also analyze the \emph{culture industry}---mass entertainment that
presents itself as liberation but actually standardizes desire and dulls critical consciousness
\parencite{AdornoHorkheimer2002}. The film's \hyperref[scene:playboy-show]{Playboy USO show} is
a textbook case: soldiers crowd to consume images of pleasure, but the spectacle is
pre-scripted, militarized, a reward for compliant service. Desire is not spontaneous; it is
managed. The tragedy is that the
soldiers experience this as fulfillment---they ``get'' the show they wanted---yet the form of
the getting domesticates them further. ``Everyone gets everything he wants'' under such
conditions means: the system reliably supplies what it has already taught you to want.

Herbert Marcuse radicalizes this insight in \textit{One-Dimensional Man} (1964) through the
concept of \emph{repressive desublimation}: advanced industrial society no longer represses
desire but grants it in controlled forms that deepen domination \parencite{MarcuseOneDim1964}.
Where classical repression bred discontent and potential revolt, repressive desublimation
allows pleasure, consumption, and even transgression---but only within channels that reinforce
the system. The mission itself operates this way: Willard gets adventure, meaning, autonomy (he
chooses when and how to act), yet every choice confirms his absorption into institutional
logic. Marcuse's contribution is precise: fulfillment punishes not by denying but by granting
in forms that close the space for refusal. The tragedy is not frustrated desire but satisfied
desire that reveals itself as pre-formed by the apparatus it cannot escape.

\hyperref[scene:kilgore-beach]{Kilgore's beach assault}, staged to Wagner's ``Ride of the
Valkyries,'' fuses instrumental violence with aesthetic spectacle. Music that could arrest
willing (as Schopenhauer hoped) is weaponized as soundtrack for domination. Adorno and
Horkheimer would read this as the colonization of art by administration: culture becomes an
instrument of power, not a refuge
from it. The horror is not that beauty fails but that it succeeds---the assault is
aesthetically magnificent and morally empty. Fulfillment at this level punishes by exposing
that even the sublime has been conscripted.
