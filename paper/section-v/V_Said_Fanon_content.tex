\phantomsection
\subsection*{V.4—Said \& Fanon: Orientalism, Colonial Violence, and the Will to Purity}
\addcontentsline{toc}{subsection}{V.4—Said \& Fanon: Orientalism and Colonial Violence}
\label{ssec:v-said-fanon}

\subsubsection*{Said: Orientalism and the Construction of the Other}
Edward Said's \textit{Orientalism} (1978) reveals how the ``East'' is not discovered but
constructed---a system of representations that authorizes Western domination by presenting the
Orient as exotic, irrational, passive, and in need of management \parencite{SaidOrientalism1978}.
Orientalism is not just prejudice; it is an entire episteme: a way of knowing that produces the
objects it claims to study. Academic discourse, travel writing, colonial administration, and
military intelligence all participate in building an image of the Orient that justifies
intervention. The colonized subject appears not as a person but as a problem to be solved.

\textit{Apocalypse Now} multiplies such representations. The Vietnamese appear almost
exclusively as objects of management: villagers to be searched, terrain to be controlled,
threats to be neutralized. The film's pervasive mediation---radio ``psyops,'' briefing maps,
Willard's narration, the photojournalist's framing---stages the war as spectacle for Western
consumption. The sampan is not encountered as a boat of persons but as a risk variable in a
security protocol. Said would note that this is Orientalism in action: the other has already
been \emph{known} through systems of classification before being seen.

When Willard ``gets what he wants''---a mission that promises clarity and purpose---what he
receives is a role within this representational apparatus. The dossier on Kurtz is a perfect
Orientalist text: it presents Kurtz as simultaneously fascinating (brilliant, decorated) and
dangerous (gone native, unsound). The narrative pre-determines how Kurtz will be seen, and
Willard's journey is an enactment of the knowledge the dossier already claimed. Fulfillment
punishes because it exposes the will as complicit in a system that cannot see the other except
as object. Said's thesis is that such systems are not accidents but constitutive of colonial
modernity \parencite{SaidOrientalism1978}.

\subsubsection*{Fanon: Colonial Violence and the Struggle for Recognition}
Frantz Fanon's \textit{The Wretched of the Earth} (1961) strips away liberal illusions about
gradual reform. Colonialism, he argues, is a structure of violence: it divides the world into
settlers and natives through force and maintains that division through surveillance, curfew,
and terror \parencite{FanonWretched2004}. The colonized subject is denied recognition as a
person; he appears to the colonizer only as labor, threat, or resource. Fanon reinterprets the
Hegelian struggle for recognition in this context: the colonized can achieve recognition only
through counter-violence, because the colonial relation admits no other medium of address
\parencite{FanonWretched2004}.

Kurtz absolutizes this logic. He does not disguise domination under humanitarian rhetoric; he
makes it explicit. His ``methods''---the severed heads, the ritualized sovereignty---are
``unsound'' only because they speak the unspoken truth of the colonial mission: it is about
power, not civilization. Fanon would note that Kurtz has grasped the colonial relation's inner
violence but, instead of repudiating it, he has embraced it as purity. He wants an act that
would finally coincide with intent, unmediated by bureaucratic euphemism. To get this is to
become fully inhuman.

Willard's assignment pits one form of colonial violence (bureaucratic, dispersed, ``surgical'')
against another (personal, charismatic, explicit). Fanon's insight is that both are symptoms
of the same structure: a world organized so that the colonized other can appear only as object
of will, never as interlocutor. When Willard completes the mission, he does not escape this
structure; he ratifies it. The will wanted sovereignty (the authority to decide Kurtz's fate)
and received it, but sovereignty within a colonial frame is always already complicit.

Fanon also warns against the ``pitfalls of national consciousness'': anti-colonial movements
can reproduce the colonizer's logic if they adopt the same will to purity
\parencite{FanonWretched2004}. Kurtz's compound, with its cult of personality and absolute
command, is such a reproduction. He has escaped the U.S. military's bureaucracy only to create
a sovereignty as violent and as void. Both ``getting what one wants'' paths---institutional
order or charismatic command---end in exposure: the will discovers that its object was
domination, and domination cannot ground a human world.
