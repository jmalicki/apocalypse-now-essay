\section*{III–C. Finitude \& Responsibility: Time, Death, and the Other (Heidegger, Levinas)}

\subsection*{Heidegger: Being-toward-Death and the Failure of Completion}
Authenticity requires “anticipatory resoluteness” toward death (\parencite[p.~307]{HeideggerBT1962}); “as soon as man comes to life, he is at once old enough to die” (\parencite[p.~298]{HeideggerBT1962}). A project that promises wholeness is structurally self-deceiving; fulfillment cannot complete Dasein. The river’s time—serial “nows” without totalization—exposes this futility.

\subsection*{Levinas: The Face Forbids Me}
“Desire is desire for the absolutely other” (\parencite[p.~33]{LevinasTI1969}); the face “forbids us to kill” (\parencite[p.~199]{LevinasTI1969}). Fulfillment fails ethically because the Other’s claim interrupts every sovereign end. Late encounters (prisoners, villagers, Kurtz as victim-perpetrator) dramatize this interruption: the mission-form cannot integrate the demand without ceasing to be \emph{that} mission.

\subsection*{Synthesis}
Heidegger limits projects by \emph{time}; Levinas limits them by the \emph{Other}. Willard’s confession registers both limits: the project granted exposes finitude and summons responsibility. In either case, fulfillment is not closure but \emph{exposure}.
