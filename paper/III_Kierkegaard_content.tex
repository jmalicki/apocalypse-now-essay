\subsection*{III—Kierkegaard: Absolutized Projects and Defiant Despair}
\label{ssec:iii-kierkegaard}
Kierkegaard defines the self as “a relation that relates itself to itself,” sick when it seeks to ground itself “in one’s own strength” (\parencite[pp.~49, 69--73]{KierkegaardSUD1980}). The finite project—to be the one who \emph{has a mission}—becomes absolute, and every success is then consumed as self-grounding. On this register, “everyone gets everything he wants” intensifies rather than relieves the sickness. “For my sins I got one” is not a discrete admission but a confession of wrong willing exposed by fulfillment.
