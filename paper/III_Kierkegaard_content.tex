\phantomsection
\subsection*{III—Kierkegaard: Despair as Absolutized Project—Why Success Thickens the
	Misrelation}
\addcontentsline{toc}{subsection}{Kierkegaard: Despair as Absolutized Project}
\label{ssec:iii-kierkegaard}
Kierkegaard treats despair not as a mood but as a structural error in how the self relates to
itself. ``The self is a relation that relates itself to itself,'' and it can be ``sick unto
death'' when that relation is grounded in the wrong power
\parencite[pp.~49--52]{KierkegaardSUD1980}.
Two principal forms of despair matter here: (1) the despair of weakness—``not to will to be
oneself,'' and (2) the despair of defiance—to will to be oneself ``in one's own strength''
\parencite[pp.~52--61, 69--73]{KierkegaardSUD1980}. Read against Willard's confession, the line
``Everyone gets everything he wants'' sketches the field on which both forms operate;
``\ldots and for my sins I got one'' is the moment the misrelation becomes clear through
success.

Kierkegaard's analysis bites hardest when a contingent project is taken as absolute. The self,
needing to be grounded ``in the power that established it,'' substitutes a finite end as its
measure and thereby misrelates itself \parencite[pp.~79--83]{KierkegaardSUD1980}. To will to be
the one who has a mission is precisely such absolutization. Before the assignment, the self
appears as lack (restless aimlessness); once the assignment is granted, the self congeals around
the project—orientation replaces drift. But in Kierkegaard's grammar this is not healing; it is
the despair of defiance: the self wills to be itself by itself through the project. The more
coherent the mission becomes, the more intense the misrelation grows, because the self is
secured by something that cannot finally ground it.

This is why, for Kierkegaard, success does not rescue but thickens despair. Success confirms
the illusion that one can be oneself by one's own project; yet every success is also a mirror,
showing that the self remains unfounded. Kierkegaard emphasizes that despair often hides beneath
``the most colossal energy'' and apparent resolve; it is ``misrelationship in a self'' that can
be ``perfectly transparent to itself'' about its project while being wrong about its ground
\parencite[pp.~72--76]{KierkegaardSUD1980}. In this light, the cool execution of procedures and
the narrowing of affect after each ``win'' signal not mastery but the tightening of defiant
despair. The line's second clause—``for my sins I got one''—reads as the moment when the grant
of the mission throws the absence of a true ground into relief.

Kierkegaard also distinguishes immediacy (living immersed in finite goods) from reflection that
can discover the self's task \parencite[pp.~84--90]{KierkegaardSUD1980}. War intensifies
immediacy by turning every face into a function and every act into a means; it encourages the
self to hide in roles. In such a milieu, the very practices that appear to deliver meaning—orders,
dossiers, operational clarity—supply the self with a surrogate infinity: a finite project that
pretends to be sufficient. But the self, for Kierkegaard, is tasked with becoming itself before
God \parencite[pp.~79--83]{KierkegaardSUD1980}. This is not a pietistic add-on; it is his way
of marking that the self's measure must transcend its own chosen ends. Whenever the measure is
reduced to the project's success, despair results—``the greater the natural capacities, the more
dangerous the despair'' \parencite[pp.~76--78]{KierkegaardSUD1980}.

Note how this frame explains the distinctive tonality of fulfillment-as-punishment. To ``get
what one wants'' is to lose the alibi that failure provides. As long as the project is ungranted,
the self can imagine that possession will establish it. Once granted, the self's emptiness
becomes undeniable. The confession ``for my sins I got one'' thus does not (primarily) express
guilt over discrete acts; it expresses recognition of a wrong willing—to be oneself by one's own
finite project. That is Kierkegaard's ``sin'' in the strict sense: not a single deed, but a
posture of self-grounding \parencite[pp.~79--83]{KierkegaardSUD1980}.

Kierkegaard's analysis also clarifies why horror does not teach the defiant self what it most
needs to learn. The self in defiance is willing to suffer anything rather than relinquish its
chosen measure; it would rather ``be itself with all the torments of hell than not be itself''
\parencite[p.~69]{KierkegaardSUD1980}. Hence the spectacle of a self that persists—relentlessly,
competently—through increasingly unredeeming outcomes. The punishment of fulfillment is that
competence becomes the instrument of despair: every efficient act confirms the sovereignty of
the project, and every confirmation deepens the misrelation.

Is there a Kierkegaardian way out? Only if the maxim of the project is converted—a re-grounding
of the self in that ``power which established it,'' which, in his lexicon, entails repentance
and a change in the measure of willing \parencite[pp.~79--83, 111--116]{KierkegaardSUD1980}.
Short of such a conversion, neither failure nor success can heal; success merely strips away
the illusion that success could heal. Thus the two halves of the sentence lock together:
``everyone gets everything he wants'' = finite ends can indeed be obtained; ``for my sins I got
one'' = obtaining them revealed the despair that absolutized them.
