
\phantomsection
\section*{V. Colonial Modernity and the Critique of Domination}
\addcontentsline{toc}{section}{V. Colonial Modernity and Critique of Domination}
\label{sec:v-colonial-modernity}

Having traced theological, philosophical, and psychological interpretations of Willard's line,
we now turn to critical theory's historicization of the will's structure. Conrad's novella and
Coppola's film locate desire within imperial modernity—a world where fulfillment is
bureaucratized, representation circulates power, and the will internalizes discipline. The
twentieth century's thinkers show how ``getting what one wants'' exposes not only metaphysical
structures but historical complicity.

\subsection*{1. The Spirit of Domination: Work, Discipline, and Representation}
\label{ssec:1-the-spirit-of-domination-work-discipline-and-representation}
Weber’s thesis on the Protestant ethic connects transcendent assurance to immanent compulsion: the anxiety of salvation is displaced into the worldly signs of vocation, productivity, and discipline \parencite{WeberProtestant2002}. In Willard’s world, vocation has lost its soteriological frame; the residue is compulsion alone. The river journey is a pilgrimage without grace: a labor of proof that can never culminate. Foucault’s analysis of modern power makes the continuity plain: discipline produces subjects by normalizing their bodies and perceptions \parencite{FoucaultDiscipline1995}. Willard’s training, files, and surveillance are not contingent backdrops; they \emph{are} the conditions under which a ``mission'' can be willed, received, and fulfilled. The will internalizes the gaze.

If domination requires a world to dominate, representation supplies it. Said shows how the Orient is constructed as an object of knowledge that authenticates Western authority \parencite{SaidOrientalism1978}. \textit{Apocalypse Now} multiplies such representations: the radio’s ``psyops'' patter, military briefings, newspaper clippings, and narration. The hilltop massacre under flares is not only an event but an image of an event; it exists to be seen. Benjamin’s ``Angel of History'' looks back not upon progress but upon ``a single catastrophe which keeps piling wreckage upon wreckage'' \parencite{BenjaminTheses1969}. The film literalizes this gaze: moving forward upriver is moving back into debris. Fulfillment of the mission produces a tableau of ruins through which the angel is blown.

\subsection*{2. Recognition, Violence, and the Will to Purity}
\label{ssec:2-recognition-violence-and-the-will-to-purity}
Fanon reinterprets Hegelian recognition within colonial relation: the colonized subject meets the colonizer’s will to purity as violence; the only available agency appears as counter-violence \parencite{FanonWretched2004}. Kurtz absolutizes this logic. His ``methods'' are ``pure'' because they purge ambivalence. He wants an end to contradiction; he wants an act that would finally coincide with intent. To get this is to erase the human. Willard confronts not only a man but the fantasy of unmediated will. Here the line between Section III’s existentialism and Section IV’s historicity thins: the metaphysics of will finds its historical instrument in colonial modernity. What the will wants (sovereignty) appears in the world as the right to decide life and death.

\subsection*{3. Fulfillment as Exposure: Modernity's Mirror}
\label{ssec:3-fulfillment-as-exposure-modernity-s-mirror}
Read in this frame, Willard’s opening line is not merely mordant wit but a summary of the century’s critique. ``Everyone gets everything he wants''---because modern institutions exist to circulate wants as functions, and because representation manufactures the worlds those wants require. ``For my sins I got one''---because the system returns desire as assignment, and the assignment reveals desire’s complicity with domination. Conrad supplies the form (a journey into the center where fulfillment collapses into horror). Critical theory supplies the terms (reason as domination, vocation as compulsion, representation as power). Coppola supplies the image: the fulfilled mission as an illuminated ruin.

What Willard learns upriver is what Conrad, Benjamin, and Fanon teach in theory: fulfillment is not closure but exposure. The prize of the modern will is to see itself in the world it has made.
