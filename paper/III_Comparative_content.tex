
\phantomsection
\subsection*{Comparative Discussion}
\addcontentsline{toc}{subsection}{Comparative Discussion}
\label{ssec:iii-comparative-discussion}

What follows sets the line against competing accounts of desire, normativity, selfhood, time,
ethics, and intersubjectivity, comparing and contrasting along these conceptual bearings to see
what ``getting'' and ``punishment'' amount to in each.

\subsubsection*{Insatiability or Transvaluation?}
\label{sssec:insatiability-or-transvaluation}

Schopenhauer hears in willing a mechanism that cancels its own promise---satisfaction ``at once
makes room for a new one,'' so life swings ``between pain and boredom'' \parencite[pp.~312,
319]{SchopenhauerWWR1969}. The upriver chain of ``wins'' looks like his pendulum: each success opens
a new deficit. Nietzsche objects that such fatalism misconstrues the task: recurrence is not a
curse if the will revalues itself from consumption to creation; what corrodes willing is the mask
of ``truth'' that licenses domination \parencite[\S 34]{NietzscheBGE1990}. The Saigon dossier's
hygienic tone favors Nietzsche's suspicion: command in the costume of cognition. Yet when the
narrative never transvalues after the sampan or the bridge, Schopenhauer's phenomenology
reasserts itself: fulfillment disenchants. Camus cuts between them: even a creative will must live
``without appeal''---no final sanction comes with completion \parencite[pp.~28, 54]{CamusMyth1991}.
The film sides with Camus at the end: the world delivers objects; what returns is lucidity, not
meaning \parencite[pp.~121--123]{CamusMyth1991}.

\subsubsection*{Success as Evidence---or as Indictment?}
\label{sssec:success-as-evidence-or-indictment}

Kant denies that outcomes ever certify worth: the good will is ``good \ldots\ in itself,'' not
``because of what it effects'' \parencite[p.~27]{KantGroundwork1996}. The sampan inspection, polished as
procedure, fails at the maxim level: persons used merely as means violates the humanity constraint
\parencite[pp.~30--33, 72--76]{KantGroundwork1996}. Nietzsche warns, however, that Kantian talk of law
can smuggle in a will to command---``the desire for `truth''' as a tool of domination
\parencite[\S 34]{NietzscheBGE1990}. The briefing room shows why both are needed: Nietzsche
unmasks rhetoric; Kant supplies the tribunal that still condemns the maxim after unmasking.
Kierkegaard adds an internal critique: even if the maxim passed, absolutizing a finite project
thickens despair \parencite[pp.~69--83]{KierkegaardSUD1980}. So success cannot vindicate (Kant),
unmasking cannot excuse (Nietzsche), and even ``right'' structure cannot cure misrelation
(Kierkegaard)---a triple pressure that makes ``\ldots for my sins I got one'' sound like the
removal of alibis.

\subsubsection*{Absolutized Projects and Agency Without Measure}
\label{sssec:absolutized-projects-and-agency-without-measure}

Kierkegaard and Dostoevsky agree that the self can destroy itself through success, but they
quarrel over the disease. For Kierkegaard, despair is a misrelation: to will to be oneself ``in
one's own strength'' by way of a project \parencite[pp.~69--73]{KierkegaardSUD1980}. Every
tactical win upriver further tightens this wrong grounding. Dostoevsky emphasizes anti-mechanism:
``man only wants independent desire,'' even against interest; the human refuses to be a ``piano
key'' \parencite[pp.~115, 129--131]{DostoevskyNFU1994}. The dossier machine offers exactly what he
fears: a rational program that absorbs agency into function. Yet the Underground Man's
medicine---negation for its own sake---corrodes too; agency defended as pure independence collapses
into self-harm. The film dramatizes both errors: obedience performed as authorship (Dostoevsky's
nightmare), and identity collapsed into the project (Kierkegaard's despair). Fulfillment exposes
both at once.

\subsubsection*{The Impossible Telos of Completion}
\label{sssec:the-impossible-telos-of-completion}

Sartre and Heidegger converge that completion is a category mistake, but for different
reasons---and their difference matters. For Sartre, the \emph{pour-soi} is ``what it is not and
not what it is'' \parencite[pp.~100--110]{SartreBN2003}; the tacit ``project to be God''
\parencite[pp.~586--604]{SartreBN2003} seeks a synthesis that cannot exist. The mission's end
punishes by revealing that impossibility. Heidegger roots the error in finitude: Dasein's wholeness
is disclosed only in being-toward-death, which individualizes now and forbids narrative
totalization \parencite[pp.~294--307]{HeideggerBT1962}. Where Sartre indicts a wish for
ontological closure, Heidegger indicts the \emph{they}-authorized fantasy that a right sequence of
``what one does'' could yield wholeness \parencite[pp.~149--168]{HeideggerBT1962}. The felt vacuum
after clean procedures speaks both languages: Sartrean exposure of bad faith and Heideggerian
removal of the \emph{das Man} alibi.

\subsubsection*{Reciprocity vs. Instrumentality}
\label{sssec:reciprocity-vs-instrumentality}

Beauvoir and Levinas both condemn instrumental projects but argue from different first
principles---and their friction refines the verdict. Beauvoir builds the other into freedom's form:
``To will oneself free is also to will others free'' \parencite[p.~73]{Beauvoir1976}. Authentic
projects open situations where others can transcend; efficient means that close horizon indict
themselves \parencite[pp.~134--147, 157--161, 164--173]{Beauvoir1976}. Levinas says even that
framework comes too late: the face's ``Thou shalt not kill'' precedes projects and resists
assimilation to any totality \parencite[pp.~21--24, 33, 199]{LevinasTI1969}. The dossier world is
already a betrayal because it sees a case, not a face. Beauvoir answers that without reciprocal
world-building, Levinasian command risks impotence; Levinas counters that ``world-building'' easily
re-totalizes. The film lets both charges land: procedures flatten alterity (Levinas), and successes
never found a shared world (Beauvoir). ``Everyone gets\ldots'' thus names means without co-agency;
``\ldots for my sins\ldots'' marks the ethical cost either way.

\subsubsection*{Asymmetry, the Face, and the First Relation}
\label{sssec:asymmetry-the-face-and-the-first-relation}

Levinas re-situates judgment before ontology: the face confronts me with an asymmetrical
demand---``forbids us to kill''---that resists absorption into my plans \parencite[pp.~33,
199]{LevinasTI1969}. The dossier/protocol world belongs to totality, which reduces alterity to the
Same \parencite[pp.~21--24, 33--36]{LevinasTI1969}. Within that mode, success itself confirms
refusal of the first relation; punishment is the deed returning as accusation. Politics may require
force, but politics is always under ethical judgment \parencite[pp.~21--24,
215--219]{LevinasTI1969}. The confession is thus not merely ontological disappointment or Kantian
heteronomy; it is acknowledgment of a primary ethical breach structured into the wanting.

\subsubsection*{Possession or Recognition?}
\label{sssec:possession-or-recognition}

Hegel and Koj{\`e}ve insist the desired satisfaction is not of things but of recognition:
``self-consciousness achieves its satisfaction only in another self-consciousness''
\parencite[\S 175]{HegelPhenomenology1977}. Lordship gets obedience and finds it void---submission
is not free acknowledgment \parencite[\S\S 187--189]{HegelPhenomenology1977}; the ``truth'' lies
with formative work that builds a common world \parencite[\S 196]{HegelPhenomenology1977}. Levinas
worries this reciprocity reinscribes alterity into system; Hegel replies that without mediation
there is no world in which freedom can appear. The film's mission-form flunks both sides: it
routinizes asymmetry, so recognition cannot stabilize (Hegel/Koj{\`e}ve), and it approaches others
as material, so the first ethical relation is refused \parencite[pp.~21--24, 199]{LevinasTI1969}.
Hence the peculiar hollowness of ``getting what one wants'': the currency (objects, effects,
dominion) cannot purchase what a human desire seeks (free acknowledgment), and the very apparatus
that ensures delivery ensures refusal.

\subsubsection*{Minimal Norms for ``Non-Punitive'' Fulfillment}
\label{sssec:minimal-norms-for-non-punitive-fulfillment}

Gathering the threads yields three negative tests and one positive norm:

\begin{enumerate}
\item \textbf{Anti-mastery (Hegel/Koj{\`e}ve):} Fulfillment that reduces the other to instrument
secures only empty recognition; it will punish in exposure.

\item \textbf{Anti-violation (Levinas):} Fulfillment that ignores the face's prohibition is
ethically null; the punishment is accusation within the self.

\item \textbf{Anti-bad-faith (Sartre):} Fulfillment that disavows its own freedom is flight;
exposure returns as nausea, not peace.

\item \textbf{Pro-reciprocity (Beauvoir):} Only projects that will others free transmute
fulfillment from possession into co-realization.
\end{enumerate}

These norms do not reconcile the traditions; they articulate a practical sieve for projects. Under
this sieve, the sentence ``Everyone gets everything he wants'' ceases to be fatalism and becomes
a diagnostic: in getting the project, one learns whether the project's structure could have yielded
anything but punishment.

\subsubsection*{Implications for Willard's Utterance}
\label{sssec:implications-for-willard-s-utterance}

Taken together, the debates constrain what the sentence can mean in this narrative world. The
film's reliable delivery of objects and outcomes accords with Schopenhauer's phenomenology of
post-fulfillment deflation and with Nietzsche's warning about a will that refuses to revalue
itself; yet the very efficiency of means proves normatively empty on Kant's measure. Success
thickens, rather than cures, the self's misrelation in Kierkegaard's sense and invites the kind of
agency without measure that Dostoevsky anatomizes; it cannot, in Sartre's ontology, grant the
closure it promises. Where Beauvoir and Levinas relocate judgment to the form of willing,
instrumental success becomes self-indicting, independent of results. Hegel and Koj{\`e}ve finally
explain the peculiar hollowness of the achieved objective: objects and dominion are the wrong
currency for a recognitive desire. In this light, the first clause---``everyone gets everything he
wants''---is descriptively true only because a totalizing order is good at furnishing means; the
second---``\ldots and for my sins I got one''---registers the moment at which that same success
exposes a misdirected wanting: one that asked from fulfillment what only lawful maxims, reciprocal
freedom, or mutual recognition could provide. The line's force is thus diagnostic rather than
aphoristic; it names not just a fact about acquisition but a judgment about the will that sought
it.

