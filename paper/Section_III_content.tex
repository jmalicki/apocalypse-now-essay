\phantomsection
\section*{III. Biblical Justice and Buddhist Causality: Fulfillment as Punishment}
\addcontentsline{toc}{section}{III. Biblical Justice and Buddhist Causality}
\label{sec:iii-biblical-justice-and-buddhist-causality}

Captain Willard's aphorism---``Everyone gets everything he wants. I wanted a mission, and for
my sins they gave me one''---condenses a structure shared across religious philosophies:
\emph{fulfillment discloses the truth of desire}. In the Biblical tradition (Hebrew Scripture
and Christian New Testament), that disclosure is moral and teleological: a will shows itself to
be
rightly or wrongly ordered toward God and neighbor. In early Buddhist analysis, the disclosure
is causal and phenomenological: craving (\emph{taṇhā}) reproduces the conditions of suffering
(\emph{dukkha}). Biblical thought reads fulfillment as a test of love's orientation; Buddhist
thought reads it as a link in a causal chain. In both, the ``gift'' of what one wants becomes
judgment---not because an external agent inflicts pain, but because the will's orientation or
the mind's grasping makes the pain intrinsic to fulfillment itself.

\phantomsection
\subsection*{III.1—Hebrew Scripture: The Gift That Judges}
\addcontentsline{toc}{subsection}{III.1—Hebrew Scripture}
\label{ssec:iii-hebrew-scripture}

The Hebrew Bible establishes a pattern where divine grant can function as judgment. The
paradigmatic text is Psalm 106:15: ``And he gave them their request; but sent leanness into
their soul'' (KJV). The verse refers to Israel's demand for meat in the wilderness: God
provides quail in abundance, yet the fulfillment brings plague. The structure is not arbitrary
punishment but disclosure---the desire for provision as if ultimate, rather than trust in
covenant faithfulness, reveals and punishes itself through its own satisfaction. The dual
movement---outward abundance, inward thinning---becomes paradigmatic for how Scripture reads
desire fulfilled.

Ecclesiastes extends this diagnostic to all earthly fulfillment: ``I have seen all the works
that are done under the sun; and, behold, all is vanity and vexation of spirit'' (Eccl.
1:14). The Preacher catalogs achievements---building, planting, accumulating---and finds that
``there is no profit under the sun'' (Eccl. 2:11). Notably, he does not lack capacity to
fulfill desire: ``whatsoever mine eyes desired I kept not from them'' (Eccl. 2:10). The
judgment comes \emph{through} the getting, not despite it. The will's misdirection toward
finite goods as if infinite produces the ``vanity.'' A person may receive ``riches, wealth,
and honour, so that he wanteth nothing for his soul,'' yet it remains ``vanity and vexation
of spirit'' (Eccl. 6:2). The gift is given; the gift exposes.

This pattern threads through Hebrew Scripture: fulfillment can reveal the disorder of the
wanting. Later interpretive traditions will systematize this into a theology of desire's
orientation, but the scriptural logic is already present: getting what one wants, when the
wanting is misdirected, does not bless but judges.

\pagebreak[2]
\phantomsection
\subsection*{III.2—Christian Scripture: Handed Over to Desire}
\addcontentsline{toc}{subsection}{III.2—Christian Scripture}
\label{ssec:iii-christian-scripture}

The Christian New Testament inherits and intensifies the Hebrew pattern. Romans 1:18--32
provides
the clearest articulation: divine ``wrath'' is revealed not as external intervention but as
permissive abandonment. Three times Paul writes, ``God gave them over'' (Rom. 1:24, 26, 28):
first to the lusts of their hearts, then to dishonorable passions, finally to a debased mind.
The grammar is juridical but the mechanism is organic: suppressed truth about God leads to
idolatry, which distorts desire, and the distorted desire is then permitted its own object.
The punishment is getting what was wanted under the conditions that made the wanting corrupt.

Joseph Fitzmyer notes that the phrase ``God gave them over'' (\emph{paredōken autous ho theos})
echoes covenant-curses in Deuteronomy and uses legal terminology for handing a prisoner to
executioners---yet here the ``executioner'' is the unchecked desire itself
\parencite{FitzmyerRomans1993}. The structure matches Psalm 106:15: outward grant, inward
leanness. Notably, Paul does not claim desire is extinguished or thwarted; rather, it achieves
its aims and the achievement exposes the will's disorder. The ``wrath'' is not retributive
bolt but revelatory hand-over.

Jesus' teaching in the Synoptic Gospels intensifies this with eschatological urgency. In Mark
8:36, he asks, ``For what shall it profit a man, if he shall gain the whole world, and lose
his own soul?'' The conditional is not hypothetical but diagnostic: maximal fulfillment
(``gaining the world'') can coincide with maximal loss. The getting does not prevent the
judgment; it \emph{is} the judgment.

The Johannine literature adds a mystical dimension. In John 12:25, Jesus declares, ``He that
loveth his life shall lose it; and he that hateth his life in this world shall keep it unto
life eternal.'' The paradox is structural: possessive attachment (``loveth his life'')
achieves its object and loses it in the keeping; self-dispossession opens eschatological
retention. First John 2:15--17 frames worldly desire---lust of the flesh, lust of the eyes,
pride of life---as inherently passing: ``he that doeth the will of God abideth for ever.''
The contrast is not between frustrated and satisfied desire but between desire-structures:
one that grasps
finitude and perishes with it, another that loves God and participates in eternity.

Within Christian canon, then, the pattern is consistent but variously emphasized. Synoptic
tradition warns eschatologically (gain the world, lose the soul). Pauline tradition diagnoses
juridically (handed over to the lusts). Johannine tradition frames mystically (loving life
loses it). All agree: fulfillment can be judgment when desire is disordered, and the disorder
is exposed not by denial but by grant.

\pagebreak[2]
\phantomsection
\subsection*{III.3—Patristic and Medieval Synthesis: Disorder, Privation, Telos}
\addcontentsline{toc}{subsection}{III.3—Patristic and Medieval Synthesis}
\label{ssec:iii-patristic-medieval}

The Church Fathers and medieval scholastics systematize the scriptural pattern into a
psychology and metaphysics of desire. Their contribution is to distinguish not \emph{that} we
desire but \emph{how} we desire---the orientation, object, and telos of willing.

Augustine of Hippo interiorizes the diagnosis in the \emph{Confessions}. His famous line,
``Every disordered affection is its own punishment'' (\emph{omnis inordinatus animus sibi ipsi
	supplicium est}), appears early in the work (2.2.2) as he reflects on adolescent theft
\parencite[p.~47]{AugustineConfessions1998}. The point is not that theft was punished by
external penalty but that the act of willing against order produced internal fragmentation.
Later, Augustine develops the distinction between \emph{cupiditas} (possessive desire, ``use''
of what should be enjoyed, orientation toward finite goods as ultimate) and \emph{caritas}
(rightly ordered love, ``enjoyment'' of God and ``use'' of creatures toward that end). In
\emph{De doctrina christiana}, he writes: ``To enjoy something is to cling to it with love for
its own sake. To use something, however, is to employ it in obtaining that which you love''
\parencite{AugustineDeDoctrina1958}. Fulfillment of \emph{cupiditas} punishes because the
finite cannot bear the weight of ultimacy; only God, as infinite Good, can satisfy without
remainder.

In \emph{City of God}, Augustine extends this to social and political order. The ``earthly
city'' organizes itself around self-love (\emph{amor sui}), seeking dominion and glory. It
succeeds---Rome conquers the Mediterranean---and the success exposes the emptiness: endless
wars, internal factionalism, the anxiety of preserving what was won
\parencite{AugustineCity2003}. The city ``gets everything it wants'' (imperium, tribute,
fame) and finds that the getting binds rather than frees. By contrast, the ``heavenly city''
orients love toward God (\emph{amor Dei}), which alone reorders desire so that fulfillment
participates in rather than exhausts the Good. Crucially, for Augustine, this reordering
requires grace: the will cannot fix itself by willing harder but must be healed by the One
toward whom it should tend.

Thomas Aquinas formalizes the metaphysics in the \emph{Summa Theologiae}. Every act of will,
he argues, aims at some good, real or apparent (ST I--II, q.~19, a.~1). Sin occurs not when
the will targets evil \emph{as such} (nothing wills its own negation) but when it turns away
from the immutable Good toward a mutable one as if it were ultimate---a movement Aquinas calls
\emph{aversio a Deo} (aversion from God) coincident with \emph{conversio ad creaturam}
(turning toward the creature) (ST I--II, q.~19, a.~9; \parencite{AquinasST1947}). The
problem is not the creature---finite goods are genuinely good---but the absolutizing.
Fulfillment of this misdirected will is privative: the good achieved cannot deliver what was
implicitly demanded (ultimacy), and the gap between expectation and reality is punishment.
Aquinas adds that this structure applies even to virtuous acts done for vainglory: the act
succeeds, recognition is gained, and the recognition reveals that honor sought \emph{for its
	own sake} cannot satisfy the rational appetite for the Absolute (ST II--II, q.~132).

Gregory of Nyssa complicates the Augustinian synthesis with his doctrine of \emph{epektasis}:
the soul's infinite ascent toward the infinite God. In \emph{Life of Moses}, Gregory interprets
Moses' request to see God's glory and the divine reply (``Thou canst not see my face'') as a
paradox: the vision is granted by being perpetually deferred, so that desire intensifies
rather than satiates \parencite{GregoryMoses1978}. Later, in the \emph{Homilies on the
	Beatitudes}, he argues that the blessed, having fulfilled one level of virtue, discover a
higher one, ad infinitum \parencite[p.~31]{GregoryBeatitudes1954}. This might seem to
contradict the ``fulfillment punishes'' structure, but Gregory's point is that
\emph{possessive} fulfillment (desire that grasps and exhausts) fails, while participatory
desire (asymptotic approach to an inexhaustible Good) continually fulfills without depleting.
Punishment still attaches to the former, not the latter.

The patristic-medieval consensus, then, is teleological: desire must be ordered toward an
appropriate end. When finite goods are taken as ultimate, their attainment exposes the
disjunction between what they are (genuinely but limitedly good) and what was demanded
(ultimate satisfaction). Augustine names this ``disordered affection''; Aquinas calls it
``aversion from God.'' Both agree: the disorder is its own punishment, and grace must
intervene to reorient the will toward its proper end.

\pagebreak[2]
\phantomsection
\subsection*{III.4—Buddhist Causal Analysis: Craving Fulfilled, Suffering Renewed}
\addcontentsline{toc}{subsection}{III.4—Buddhist Causal Analysis}
\label{ssec:iii-buddhist-causal-analysis}

Early Buddhist teaching locates suffering not in moral disorder but in the causal structure of
craving itself. The \emph{Dhammapada} states the principle: ``From craving arises grief, from
craving arises fear; for one who is free of craving there is no grief or fear''
\parencite{BuddharakkhitaDhp1993}. The logic is causal, not punitive. Craving (\emph{taṇhā})
seeks to secure what is inherently unstable, and fulfillment does not resolve this---it
triggers the next cycle.

The teaching of dependent origination (\emph{paṭiccasamuppāda}) formalizes this:
craving conditions clinging, clinging conditions becoming, becoming conditions suffering
\parencite{BodhiSN2000}. The key insight is that craving, once gratified, does not terminate
but perpetuates the chain. Satisfaction is structurally impossible because the craving is not
for any particular thing but for permanence in what is impermanent. Fulfillment feeds the
chain rather than breaking it.

The Fire Sermon makes this visceral: ``The eye is burning \ldots\ burning with the fire of
lust, with the fire of hate, with the fire of delusion'' \parencite{BodhiSN2000}. Fire
consumes its fuel. Craving, granted its object, does not rest but continues to burn because
the burning \emph{is} the craving. Fulfillment cannot cool what is intrinsically aflame.

The remedy differs fundamentally from Biblical reordering. Rather than redirecting desire
toward a highest good, the path aims at cessation (\emph{nirodha})---extinguishing the fires
through disenchantment, dispassion, and liberation \parencite{Rahula1959}. There is no
reorientation toward an eternal Good but a cooling, an unbinding (\emph{nibbāna}).

Coppola's river journey maps this phenomenology without endorsing its metaphysics. Do Lung
Bridge is rebuilt daily, destroyed nightly---samsaric repetition without progress. The Playboy
show promises gratification, amplifies agitation, and leaves the crew more restless: craving
fed, suffering renewed. The mission itself, completed, does not bring rest but disillusionment.
In Buddhist terms, each ``success'' is another link in the chain, and the chain's logic is
exposure: what you wanted cannot satisfy, because wanting is the problem.

\pagebreak[2]
\phantomsection
\subsection*{Comparative Synthesis: Binding, Punishment, Healing}
\addcontentsline{toc}{subsection}{Comparative Synthesis}
\label{ssec:iii-synthesis}

Hebrew Scripture, Christian New Testament, patristic-medieval theology, and early Buddhism
converge on the paradox that fulfillment can be judgment---yet they diverge profoundly in
ontology, diagnosis, and remedy.

\textbf{What binds?} Biblical tradition names the bondage as \emph{sin}: love disordered
toward finite goods as if ultimate. Psalm 106 shows Israel getting what it wants and
discovering the want itself was misdirected. Augustine systematizes this in the
\emph{Confessions} as \emph{cupiditas} (possessive desire) versus \emph{caritas} (rightly
ordered love) \parencite[p.~47]{AugustineConfessions1998}. Aquinas locates the disorder in
\emph{aversio a Deo}: turning from the infinite Good toward creatures absolutized (ST I--II,
q.~19, a.~9; \parencite{AquinasST1947}). The problem is not desire \emph{per se} but its
target and manner.

Buddhist tradition names the bondage as \emph{taṇhā-upādāna}: craving and clinging that arise
from ignorance of impermanence and non-self. The \emph{Dhammapada} (Dhp. 216) and
\emph{Saṃyutta Nikāya} (SN 12.2) diagnose suffering as endogenous to grasping
\parencite{BuddharakkhitaDhp1993}; \parencite[p.~536]{BodhiSN2000}. The problem is not
misdirection but \emph{direction-as-such}---any attachment to conditioned phenomena
perpetuates becoming.

\textbf{What punishes?} In Biblical-patristic idiom, the will is ``handed over'' to its object
(Romans 1:24), or fulfillment brings ``leanness of soul'' (Psalm 106:15). Augustine writes
that disordered affection is its own punishment; Aquinas that turning from God entails
privation. The punishment is not external retribution but the intrinsic failure of finite goods
to bear infinite demand. The mechanism is teleological: a will aimed wrongly cannot reach its
true end, and the mismatch is experienced as emptiness, fragmentation, or despair.

In Buddhist idiom, gratification feeds the next link in dependent origination. Craving
conditions clinging, clinging conditions becoming, becoming conditions birth, birth conditions
aging-and-death (SN 12.2). The punishment is causal and impersonal: no judge assigns penalty,
but the structure of conditioned arising guarantees that satisfied craving generates new
suffering. The Fire Sermon's imagery---sense faculties ``burning'' with lust, hate,
delusion---shows that the fire is not fuel-dependent; it \emph{is} the craving itself, so
``feeding'' it intensifies rather than extinguishes it.

Both traditions understand Coppola's \hyperref[scene:sampan]{sampan episode} identically in
form but differently in ground: the mission ``succeeds,'' the unarmed civilians are killed,
the crew's affect goes hollow. Biblical interpretation reads this as disordered desire
(securing the mission as ultimate) exposed through its own success. Buddhist interpretation
reads it as craving (security, mission-completion) perpetuating suffering through another link
in the chain. The
phenomenology is the same; the metaphysics differ.

\textbf{What heals?} Biblical-patristic tradition proposes grace re-ordering love. Augustine
insists in the \emph{City of God} that the will, enslaved to \emph{cupiditas}, cannot free
itself but must be healed by participation in divine love (DCD XIV.28;
\parencite{AugustineCity2003}). Aquinas agrees: only when desire is rightly ordered toward God
as ultimate end can finite goods be enjoyed without privation (ST I--II, q.~19;
\parencite{AquinasST1947}). Gregory of Nyssa's \emph{epektasis} in the \emph{Life of Moses}
suggests that infinite desire directed toward the infinite God becomes non-possessive ascent
rather than grasping \parencite[pp.~113--114]{GregoryMoses1978}. The remedy is
\emph{reorientation}: not ceasing to desire but desiring rightly.

Buddhist tradition proposes \emph{cessation}. The Noble Eightfold Path cultivates
disenchantment with conditioned phenomena, leading to the cooling of craving and liberation
from the cycle \parencite[pp.~45--50]{Rahula1959}. There is no reorientation toward a higher
Good but a progressive unbinding from all objects of attachment.

The two paths are not easily reconciled. Biblical-patristic thought insists that the soul is
made for God and only infinite Good can satisfy; desire rightly ordered ascends eternally
without exhaustion. Early Buddhism insists that any conditioned object, even ``God,'' remains
within the cycle and thus subject to dukkha; only cessation of craving brings peace. One
infinitizes desire toward the Absolute; the other extinguishes grasping altogether.

Yet both agree on the diagnostic: finite fulfillment, sought possessively, punishes. The
administrative mission that Willard receives promises clarity, purpose, even redemption. It
delivers exposure: of complicity (Biblical reading) or of samsaric repetition (Buddhist
reading). ``Everyone gets everything he wants''---the structure is universal. ``For my sins I
got one''---the judgment is intrinsic. The film offers no remedy, only the vision of
fulfillment as wound. Whether healing requires reordered love or cessation of craving remains
theologically contested. What the \hyperref[scene:upriver-journey]{river journey} confirms is
that getting what one wants, when the wanting is disordered or grasping, does not liberate. It
binds---and the binding is the punishment.

