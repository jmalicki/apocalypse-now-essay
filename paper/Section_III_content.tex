\phantomsection
\section*{III. Western Philosophy: From Enlightenment Through Idealism to Existentialism}
\addcontentsline{toc}{section}{III. Western Philosophy}
\label{sec:iii-western-philosophy}

Modern Western philosophy reads Willard's line not as moral judgment or causal sequence but as
an experiment in self-disclosure: when desire is granted its object, what is revealed about the
freedom that desired it? From the Enlightenment's architectonics of will (Kant) through German
Idealism's dialectics of recognition (Schopenhauer, Hegel) to existentialism and phenomenology's
accounts of project, despair, and finitude (Kierkegaard, Dostoevsky, Sartre, Beauvoir, Camus,
Heidegger, Levinas), each thinker supplies a distinct optic through which ``getting what one
wants'' becomes punitive—not because satisfaction is withheld, but because it exposes a
structure (metaphysical, existential, ethical) that the will misconceived or refused. The
following twelve analyses trace that exposure across the tradition.

\subsection*{III—Schopenhauer: Fulfillment as Disclosure of Lack}
\label{ssec:iii-schopenhauer}
Schopenhauer's analysis of will offers a rigorous grammar for Willard's confession:
``Everyone gets everything he wants. I wanted a mission, and for my sins they gave me one.''
For Schopenhauer, desire is not a teleology that culminates in peace but a mechanism whose
very satisfaction resets itself. Hence the mission granted is not a gift that stills the heart;
it is the next oscillation of a structure that cannot be stilled.

\subsubsection*{1) The structure of willing: lack $\rightarrow$ striving
	$\rightarrow$ relief $\rightarrow$ renewed lack}

``All willing springs from lack, from deficiency, and therefore from suffering''
\parencite[p.~196]{SchopenhauerWWR1969}. The object that seems to promise rest is already
implicated in the will's unease; when attained, it ``at once makes room for a new one''
\parencite[p.~319]{SchopenhauerWWR1969}. Schopenhauer's famed image is diagnostic rather
than rhetorical: life ``swings like a pendulum, to and fro between pain and boredom''
\parencite[p.~312]{SchopenhauerWWR1969}.

Read against the film's first movements, the pattern holds precisely. In Saigon, Willard's
lack is staged as agitation and intoxication; the dossier scene supplies an object (the mission)
and a narrative. Relief appears as orientation—a reason to move upriver—but immediately becomes
renewed lack: each checkpoint demands the next, each ``win'' opens a further deficit.
Willard's voiceover keeps the pendulum audible: completion never completes; it only
re-initiates striving.

\subsubsection*{2) Why fulfillment punishes: the object is a delusion of rest}

Schopenhauer emphasizes that satisfaction exposes, rather than heals, the structure of desire.
Enjoyment ``we have longed for'' soon leaves us ``bored,'' and the will ``returns to its old
course'' \parencite[p.~319]{SchopenhauerWWR1969}. In that sense, getting what one wanted hurts
because it removes the fantasy that the object could silence the will. The hurt is cognitive:
fulfillment disenchants.

The film thematizes this in miniature. The Playboy show promises heightened pleasure; the
immediate after-image is agitation and bargaining. Kilgore's beachhead produces tactical
success, but the spectacle (``I love the smell of napalm in the morning'') converts victory
into appetite. The sampan search yields compliance, then horror; the ``completed'' procedure
reveals leanness of soul. In Schopenhauer's terms, each fulfilled want punctures its own
promissory aura and re-installs the need to want.

\subsubsection*{3) Representation, will, and the aesthetic \& ethical brakes that fail in war}

Schopenhauer distinguishes the world as representation (ordered by the principle of sufficient
reason) from the world as will (blind striving) \parencite[pp.~3--5]{SchopenhauerWWR1969}.
Two brakes can mitigate the will's tyranny. First, aesthetic contemplation suspends willing by
fixing consciousness on the Idea—to ``lose oneself in the object''
\parencite[p.~178]{SchopenhauerWWR1969}. Second, compassion reframes the other not as instrument
but as a fellow bearer of suffering \parencite[pp.~372--374]{SchopenhauerWWR1969}.
Both brakes fail in the film's wartime economy. Wagner's ``Ride of the Valkyries'' is mobilized
as a stimulant for domination, not as will-suspending contemplation; the sampan protocol
subordinates pity to procedure. The very mechanisms that could have cooled willing are
conscripted by it.

\subsubsection*{4) Boredom, repetition, and the river as pendulum}

If pain signals unfulfilled desire, boredom signals desire's deflation after satisfaction.
Schopenhauer's claim that joy fades into ennui is not a counsel of mood but an analysis
of the will's metabolism \parencite[pp.~312--320]{SchopenhauerWWR1969}. The river sequences
enact this metabolism: periods of frantic danger (pain) alternate with slack stretches of
waiting (incipient boredom), and Willard's narration re-ignites the need for the next trial.
The Do Lung Bridge sequence literalizes the pendulum: building by day, destruction by night;
every ``achievement'' immediately generates its contrary.

\subsubsection*{5) ``For my sins I got one'': the inner necessity of punishment}

Schopenhauer does not require an external punisher. The punishment is inner: to ``get what one
wants'' is to have the will's insatiability revealed to oneself. Hence the tone of Willard's
clause; the ``sin'' is not only moral guilt but attachment to the fantasy that a mission could
deliver more than recurrence. The gift is therefore a judgment.

At Kurtz's compound, this inner necessity is complete. Kurtz has arranged the conditions for
maximal satisfaction (command, removal of obstacles) and finds that mastery produces only a
clarified view of the will's void: possession does not pacify. Willard's approach—labor, danger,
deprivation—does not redeem the will, but it strips away the last illusions about what
fulfillment can do. In Schopenhauer's lexicon, the world is disclosed as will precisely when
desire succeeds.

\subsubsection*{6) Objection \& counterpoint: is there any release?}

One might object that Schopenhauer allows two releases. First, aesthetic states offer
``deliverance'' from the will's press \parencite[p.~178]{SchopenhauerWWR1969}. Second,
compassion grounds ethics beyond egoistic striving \parencite[pp.~372--374]{SchopenhauerWWR1969}.
The film acknowledges both in negative: music becomes a tool for domination rather than
contemplation; pity is subordinated to protocol. The point is not that release is metaphysically
impossible, but that this narrative world is structured to block it; thus, fulfillment returns
as exposure.

\subsubsection*{7) Payoff for the thesis}

Schopenhauer thus illuminates the first half of Willard's sentence: \emph{everyone} (because
willing is universal) \emph{gets} (because objects are available) \emph{everything} (because
the will projects ``the all'' onto finite objects) he \emph{wants} (because wanting, not the
wanted, is fundamental). The second half—``for my sins I got one''—expresses the cognate of this
metaphysics: punishment is not denial but grant that unmasks the will's structure. The mission
is not a deviation from desire's grammar; it is the very form in which that grammar is made
visible.

\newpage
\subsection*{III—Nietzsche: Transvaluation versus the Mask of “Truth”}
\label{ssec:iii-nietzsche}
Nietzsche contests Schopenhauer’s resignation while preserving recurrence: what corrodes willing is not recurrence itself but a \emph{valuation} that reduces willing to consumption. The task is transvaluation—recasting recurrence as creative affirmation. His warning that the “desire for ‘truth’ ” can operate as a disguised will to command (\parencite[\S34]{NietzscheBGE1990}) is dramatized by the Saigon dossier’s sanitary rhetoric: cognition as a mask for domination. In that light, “everyone gets everything he wants” reads as the triumph of a will to order; because no transvaluation occurs upriver, recurrence returns as nausea, not joy. “…for my sins I got one” marks the recognition that correct outcomes without revalued ends deepen rather than cure the malaise.

\newpage
\subsection*{III—Kant: Maxims, Humanity, and the Irrelevance of Outcomes}
\label{ssec:iii-kant}
For Kant, no accumulation of successes can certify moral worth: the good will is “good … in itself,” not “because of what it effects” (\parencite[p.~27]{KantCPrR1996}). The decisive question is the maxim—can it be willed as universal law, and does it honor humanity “always as an end and never merely as a means” (\parencite[pp.~27--33]{KantGroundwork1996})? The sampan inspection is instructive: meticulous adherence to procedure cannot redeem a maxim that objectifies persons (\parencite[pp.~30--33, 72--76]{KantCPrR1996}). The first clause is morally mute (efficacy, not legitimacy); the second names exposure of heteronomy in the moment of success.

\newpage
\phantomsection
\subsection*{III—Kierkegaard: Despair as Absolutized Project—Why Success Thickens the
	Misrelation}
\addcontentsline{toc}{subsection}{Kierkegaard: Despair as Absolutized Project}
\label{ssec:iii-kierkegaard}
Søren Kierkegaard treats despair not as a mood but as a structural error in how the self relates
to itself. ``The self is a relation that relates itself to itself,'' and it can be ``sick unto
death'' when that relation is grounded in the wrong power
\parencite[pp.~49--52]{KierkegaardSUD1980}.
Two principal forms of despair matter here: (1) the despair of weakness—``not to will to be
oneself,'' and (2) the despair of defiance—to will to be oneself ``in one's own strength''
\parencite[pp.~52--61, 69--73]{KierkegaardSUD1980}. Read against Willard's confession, the line
``Everyone gets everything he wants'' sketches the field on which both forms operate;
``\ldots and for my sins I got one'' is the moment the misrelation becomes clear through
success.

Kierkegaard's analysis bites hardest when a contingent project is taken as absolute. The self,
needing to be grounded ``in the power that established it,'' substitutes a finite end as its
measure and thereby misrelates itself \parencite[pp.~79--83]{KierkegaardSUD1980}. To will to be
the one who has a mission is precisely such absolutization. Before the assignment, the self
appears as lack (restless aimlessness); once the assignment is granted, the self congeals around
the project—orientation replaces drift. But in Kierkegaard's grammar this is not healing; it is
the despair of defiance: the self wills to be itself by itself through the project. The more
coherent the mission becomes, the more intense the misrelation grows, because the self is
secured by something that cannot finally ground it.

This is why, for Kierkegaard, success does not rescue but thickens despair. Success confirms
the illusion that one can be oneself by one's own project; yet every success is also a mirror,
showing that the self remains unfounded. Kierkegaard emphasizes that despair often hides beneath
``the most colossal energy'' and apparent resolve; it is ``misrelationship in a self'' that can
be ``perfectly transparent to itself'' about its project while being wrong about its ground
\parencite[pp.~72--76]{KierkegaardSUD1980}. In this light, the cool execution of procedures and
the narrowing of affect after each ``win'' signal not mastery but the tightening of defiant
despair. The line's second clause—``for my sins I got one''—reads as the moment when the grant
of the mission throws the absence of a true ground into relief.

Kierkegaard also distinguishes immediacy (living immersed in finite goods) from reflection that
can discover the self's task \parencite[pp.~84--90]{KierkegaardSUD1980}. War intensifies
immediacy by turning every face into a function and every act into a means; it encourages the
self to hide in roles. In such a milieu, the very practices that appear to deliver meaning—orders,
dossiers, operational clarity—supply the self with a surrogate infinity: a finite project that
pretends to be sufficient. But the self, for Kierkegaard, is tasked with becoming itself before
God \parencite[pp.~79--83]{KierkegaardSUD1980}. This is not a pietistic add-on; it is his way
of marking that the self's measure must transcend its own chosen ends. Whenever the measure is
reduced to the project's success, despair results—``the greater the natural capacities, the more
dangerous the despair'' \parencite[pp.~76--78]{KierkegaardSUD1980}.

Note how this frame explains the distinctive tonality of fulfillment-as-punishment. To ``get
what one wants'' is to lose the alibi that failure provides. As long as the project is ungranted,
the self can imagine that possession will establish it. Once granted, the self's emptiness
becomes undeniable. The confession ``for my sins I got one'' thus does not (primarily) express
guilt over discrete acts; it expresses recognition of a wrong willing—to be oneself by one's own
finite project. That is Kierkegaard's ``sin'' in the strict sense: not a single deed, but a
posture of self-grounding \parencite[pp.~79--83]{KierkegaardSUD1980}.

Kierkegaard's analysis also clarifies why horror does not teach the defiant self what it most
needs to learn. The self in defiance is willing to suffer anything rather than relinquish its
chosen measure; it would rather ``be itself with all the torments of hell than not be itself''
\parencite[p.~69]{KierkegaardSUD1980}. Hence the spectacle of a self that persists—relentlessly,
competently—through increasingly unredeeming outcomes. The punishment of fulfillment is that
competence becomes the instrument of despair: every efficient act confirms the sovereignty of
the project, and every confirmation deepens the misrelation.

Is there a Kierkegaardian way out? Only if the maxim of the project is converted—a re-grounding
of the self in that ``power which established it,'' which, in his lexicon, entails repentance
and a change in the measure of willing \parencite[pp.~79--83, 111--116]{KierkegaardSUD1980}.
Short of such a conversion, neither failure nor success can heal; success merely strips away
the illusion that success could heal. Thus the two halves of the sentence lock together:
``everyone gets everything he wants'' = finite ends can indeed be obtained; ``for my sins I got
one'' = obtaining them revealed the despair that absolutized them.

\newpage
\subsection*{III—Dostoevsky: Independent Desire and Anti-Mechanism}
\label{ssec:iii-dostoevsky}
The Underground Man insists “man only wants independent desire,” refusing to be a “piano key” in a rational program (\parencite[pp.~115, 129--131]{DostoevskyNFU1994}). The dossier-machine is the nightmare: an apparatus that \emph{delivers} missions and interprets obedience as authorship. The line’s two halves catch the trap: the system supplies occasions to act (“everyone gets …”), and the agent, having sought agency \emph{as such}, discovers in success only the mirror of compulsion. Mission-success reveals heteronomy in the form of efficacy.

\newpage
\phantomsection
\subsection*{III—Sartre: Freedom as Condemnation, the Impossible Synthesis, and Fulfillment as
	Exposure}
\addcontentsline{toc}{subsection}{Sartre: Freedom as Condemnation}
\label{ssec:iii-sartre}
Sartre's ontology makes ``Everyone gets everything he wants'' a trap built into freedom. For
him, human reality (\emph{pour-soi}) is a lack that projects itself toward being; it is ``what
it is not and not what it is,'' a perpetual surpassing of itself
\parencite[pp.~100--110]{SartreBN2003}. Desire therefore aims, at bottom, at an ontological
closure it can never attain. The will does not simply seek objects; it seeks to abolish its
lack by becoming a settled being. That is the hidden horizon against which the mission takes on
its peculiar glow. ``For my sins I got one'' names the moment the project's promised closure
reveals itself as structurally impossible.

The project-form is central to Sartre's account of freedom. Freedom is not a privilege but the
very structure of consciousness: we are ``condemned to be free,'' without essence to excuse or
guarantee our choices \parencite[pp.~34--36]{SartreBN2003}. Because the \emph{pour-soi} is
nothing but transcendence beyond the given (facticity), every life is a project—a coherent
orientation that confers meaning retroactively on its acts
\parencite[pp.~561--569]{SartreBN2003}. The Saigon acceptance scene reads here as the decisive
orientation of a freedom in flight from its drift: a project chosen to still contingency by
giving it a vector. But, in Sartre's grammar, such orientation never stills the source; it
intensifies responsibility. Once the mission is chosen, there are no alibis left.

Beneath every finite project, Sartre identifies a secret, universal temptation—the ``project to
be God'': to fuse our throwness (facticity) and our transcendence into a single, self-grounding
plenitude \parencite[pp.~586--604]{SartreBN2003}. That synthesis is impossible. The
\emph{pour-soi} can never coincide with itself as the \emph{en-soi} does; it can only nihilate
the given and project beyond it. When a mission is taken as the end that would reconcile what
we are (situated, limited) with what we intend (sovereign authorship), fulfillment must punish
because its very success exposes the misconceived telos: the project could not, even in
principle, provide what the will implicitly asked of it—ontological peace.

Sartre's analysis of bad faith illuminates how institutional roles mask this impossibility.
``Bad faith'' names the flight from freedom by posing oneself as either pure thing (just
obeying orders) or pure transcendence (unconditioned author), disowning the inseparable unity
of both \parencite[pp.~86--116]{SartreBN2003}. The procedural rhetoric of dossiers, signatures,
and necessity tempts the agent to occupy the role of function—a thing among things—while
narrating himself as a lucid executor. In fact, the act is freely chosen under a maxim that the
agent owns. Fulfillment punishes because once the project is complete, the alibi of role
collapses: nothing in the world compelled this project as mine. The confession (``\ldots for my
sins I got one'') reads as a crack in bad faith: a recognition that the necessity was staged.

The problem deepens when we consider Sartre's account of the Look (\emph{le regard}), which
shows how others reveal our facticity while tempting us to convert them into means for our
project \parencite[pp.~252--302]{SartreBN2003}. A mission-form that objectifies faces as
obstacles or instruments produces a world of being-for-others devoid of reciprocity. Each
``efficient'' success therefore deepens alienation: it multiplies acts in which the other's
freedom is suppressed to maintain the project's clarity. The more cleanly the procedure runs,
the more legible the structure becomes: meaning has been outsourced to instrumental success,
not grounded in a shared world. Fulfillment is thus a revelation: what we wanted was not meaning
but the uninterrupted sovereignty of a plan.

Sartre's relentless claim is that responsibility remains absolute. Causal explanation never
cancels authorship. Situations ``are what they are,'' but they are what they are for a freedom
that chooses what to make of them \parencite[pp.~553--561]{SartreBN2003}. This is why the end
of a project often feels accusatory. When nothing redemptive follows a technically perfect
execution, the agent confronts the naked fact that the project's value was not in the world but
in the choice that sustained it. ``Everyone gets everything he wants'' then means: the world is
reliable at delivering objects for our projects; ``\ldots and for my sins I got one'' means:
once delivered, the project reflects my choice back at me without the cushion of failure.

One might object, Sartreanly, that lucid perseverance—owning the act without appeal—could
transfigure the mission into authenticity. But for Sartre authenticity is not stubbornness; it
is lucidity about the impossibility of completion and refusal of bad faith in either direction
(no hiding in role; no fantasy of omnipotent authorship). In a world where the project's
structure systematically instrumentalizes others, lucidity would require altering the project or
abandoning it, not merely executing it honestly. Where no such alteration occurs, fulfillment
cannot be redemptive: it is an X-ray of the willing that carried it.

Sartre thus underwrites both halves of Willard's sentence. ``Everyone gets everything he
wants'': the world provides ample situations in which freedom can adopt ends and see them
through. ``For my sins I got one'': because the end was implicitly a bid for the impossible
synthesis (completion, immunity from ambiguity), getting it reveals the project as bad-faith
flight from freedom's structure. Punishment is not failure but clarity—the clarity that
completion was never on offer.

\newpage
\subsection*{III—Beauvoir: Reciprocity as Freedom’s Form}
\label{ssec:iii-beauvoir}
Beauvoir internalizes ethics to the structure of freedom: “To will oneself free is also to will others free” (\parencite[p.~73]{Beauvoir1976}). Authentic projects \emph{open} situations in which others can transcend; efficient means that \emph{close} horizons convict themselves by their very success (\parencite[pp.~134--147, 157--161, 164--173]{Beauvoir1976}). The first clause reports reliable means; the second is the ethical verdict that reciprocity was excluded from the end.

\newpage
\subsection*{III—Camus: Absurd Lucidity and Action without Appeal}
\label{ssec:iii-camus}
The absurd is “born of the confrontation between the human need and the unreasonable silence of the world” (\parencite[p.~28]{CamusSisyphus1991}). To live “without appeal” (\parencite[p.~54]{CamusSisyphus1991}) is to abandon the hope that completion provides a final court of justification. The Do Lung Bridge cycle—building and erasure—reads like a Sisyphean figure; completion yields knowledge, not meaning (\parencite[pp.~121--123]{CamusSisyphus1991}). Thus the sentence’s first clause can be true; the second names fulfillment as the world’s silence.

\newpage
\subsection*{III—Heidegger: Finitude and the Category Error of Wholeness}
\label{ssec:iii-heidegger}
Heidegger’s claim is ontological: Dasein’s “wholeness” is disclosed only in being-toward-death; narrative consummation is a category mistake for finite existence (\parencite[pp.~294--307]{HeideggerBT1962}). Average everydayness (\emph{das Man}) supplies the rhetoric of necessity that authorizes such fantasies (\parencite[pp.~149--168]{HeideggerBT1962}). When the sequence “works,” the alibi disappears. The first clause names a world of operable equipment; the second registers, in the key of finitude, that such operability cannot yield existential completion.

\newpage
\phantomsection
\subsection*{IV.11—Levinas: The Face's Prohibition, Asymmetrical Responsibility, and Why
	``Success'' Condemns Instrumental Projects}
\addcontentsline{toc}{subsection}{IV.11—Levinas: The Face and Asymmetrical Responsibility}
\label{ssec:iii-levinas}
Emmanuel Levinas relocates first philosophy from ontology to ethics: the encounter with the face
institutes an asymmetrical demand prior to any project or knowledge. ``Desire is desire for the
absolutely other'' \parencite[p.~33]{LevinasTI1969}, and the face ``forbids us to kill''
\parencite[p.~199]{LevinasTI1969}. This is not a thesis about consequences but a command
inscribed in the presentation of the other as infinite—irreducible to roles, functions, or my
plans \parencite[pp.~194--201]{LevinasTI1969}. Measured by this standard, ``Everyone gets
everything he wants'' is ethically null until we ask whether what was wanted preserved the
other's irreducibility; ``\ldots and for my sins I got one'' reads as the moment when a granted
project reveals, by its own success, that it had bracketed that demand.

Levinas's notion of totality versus infinity names the fault-line. Totality is the regime that
reduces alterity to the Same—catalogues, protocols, categories; infinity is the breach of that
reduction in the epiphany of the face \parencite[pp.~21--24, 33--36]{LevinasTI1969}. The
mission-form—dossier, diagnosis, elimination—is quintessentially totalizing: it metabolizes
faces as data points and tasks. The sampan scene is an X-ray: even before the fatal shot, the
encounter runs on risk calculus. In Levinas's grammar, the ethical failure precedes the mistake;
the very mode of approach ``has already spoken'' by refusing the face's claim. Success cannot
redeem such refusal; it confirms it. ``Getting what one wants'' within this regime is punishment
as self-revelation: the act returns to the agent as accusation.

Levinas is explicit that the ethical relation is asymmetrical: I am responsible for the other
beyond reciprocity or contract \parencite[pp.~215--219]{LevinasTI1969}. This asymmetry is
precisely what proceduralism neutralizes, since procedures aim to distribute liability
symmetrically. Hence the peculiar chill of the film's most efficient moments: where a protocol
works, the asymmetry has been most thoroughly suppressed. The ethical demand has not been
answered; it has been absorbed—turned into a variable among others. Levinas's insistence that
the face is a ``poor one, a stranger'' \parencite[p.~213]{LevinasTI1969} gives content to the
felt wrongness of treating villagers, boat crews, and even soldiers as means for the continuity
of the project. The wrongness is not (only) that harm occurs; it is that the form of encounter
precluded responsibility before deciding what to do.

The assassination order against Kurtz does not escape this logic by turning against a tyrant.
Levinas's ``Thou shalt not kill'' is not a rule applied to friends but the structure of
encounter itself \parencite[p.~199]{LevinasTI1969}. To meet anyone—enemy included—first as a
bearer of exteriority is to be summoned to justification. There may be cases, Levinas allows,
where politics demands force; but politics is always under judgment by ethics
\parencite[pp.~21--24]{LevinasTI1969}. The film's denouement shows the inversion: politics
judges ethics, and efficiency is taken as justification. That is why the line's second half
sounds like a verdict: ``\ldots for my sins I got one'' acknowledges that the project's end
never included the first relation—the face's command—so its successful completion can only
declare that exclusion more clearly.

Levinas also explains why horror often clarifies rather than teaches in such worlds. Horror
strips away alibis and yet, without a conversion of the mode of approach, it cannot generate the
responsibility it reveals. The proper response is not a grand theory but a change in the grammar
of encounter—hospitality, attention, refusal of instrumentalization
\parencite[pp.~200--206]{LevinasTI1969}. In their absence, ``Everyone gets everything he wants''
remains the slogan of totality: the world is very good at supplying means. The punishment of
fulfillment is the renewed summons one cannot now un-hear.

\newpage
\subsection*{III—Hegel: Recognition Against Possession}
\label{ssec:iii-hegel}
Desire is not finally of things but of recognition: “self-consciousness achieves its satisfaction only in another self-consciousness” (\parencite[\S175]{HegelPhenomenology1977}). Lordship “gets” obedience and finds it void—submission is not free acknowledgment (\parencite[\S\S187--189]{HegelPhenomenology1977}); the “truth” lies with formative work that builds a common world (\parencite[\S196]{HegelPhenomenology1977}). The first clause describes accumulation in the wrong currency; the second is the recognitive bill coming due.

\newpage
\subsection*{III—Koj\`eve: Desire of Desire and Historical Stakes}
\label{ssec:iii-kojeve}
Human desire is the desire of another’s desire; satisfaction requires being recognized as free by a free other, not merely possessing objects (\parencite[pp.~6, 27--34]{KojeveIRH1980}). Hence the master’s “victory” is empty: coerced recognition is not recognition (\parencite[pp.~158--164]{KojeveIRH1980}). “Everyone gets everything he wants” marks an apparatus expert at distributing missions; “…and for my sins I got one” is the discovery that such distribution cannot secure the recognitive relation the human wants.

