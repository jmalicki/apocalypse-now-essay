\phantomsection
\section*{III. Western Philosophy: From Enlightenment Through Idealism to Existentialism}
\addcontentsline{toc}{section}{III. Western Philosophy}
\label{sec:iii-western-philosophy}

Modern Western philosophy reads Willard's line not as moral judgment or causal sequence but as
an experiment in self-disclosure: when desire is granted its object, what is revealed about the
freedom that desired it? From the Enlightenment's architectonics of will (Kant) through German
Idealism's dialectics of recognition (Schopenhauer, Hegel) to existentialism and phenomenology's
accounts of project, despair, and finitude (Kierkegaard, Dostoevsky, Sartre, Beauvoir, Camus,
Heidegger, Levinas), each thinker supplies a distinct optic through which ``getting what one
wants'' becomes punitive—not because satisfaction is withheld, but because it exposes a
structure (metaphysical, existential, ethical) that the will misconceived or refused. The
following twelve analyses trace that exposure across the tradition.

\phantomsection
\subsection*{IV.1—Kant: Duty, Autonomy, and Why ``Success'' Proves Nothing}
\addcontentsline{toc}{subsection}{IV.1—Kant: Duty, Autonomy, and Success}
\label{ssec:iii-kant}
Immanuel Kant gives the sharpest rebuke to reading fulfillment as vindication. In his moral
philosophy, the worth of an action lies not in what it achieves but in the maxim from which
it is done—the
principle the agent could will as universal law. The line ``Everyone gets everything he wants''
therefore cannot count as evidence that wanting was justified; ``for my sins I got one'' sounds
like the dawning recognition that having one's ends granted can lay bare a prior failure of
duty.

Kant's baseline claim is that the good will is ``good \ldots\ in itself,'' not by the
advantages it produces \parencite[p.~27]{KantCPrR1996}. This relocates ethical assessment away
from effects (which are subject to luck, power, and circumstance) to the will's legislation of
its own maxim. When Willard accepts the assignment in the cool, procedural light of Saigon,
the scene supplies everything success-friendly ethics likes—clarity of ends, chain of command,
legal sanction. For Kant, none of that matters morally. The question is simple and brutal:
What maxim am I adopting, and can I will it as a law for all rational agents? If the maxim is
``Eliminate as a means any person my institution designates an obstacle,'' universalization
collapses into contradiction: it destroys the very conditions of mutual recognition that a law
for all would require. The mission can be ``successful'' and still be morally void.

This is the force of Kant's second test—the humanity constraint, expressed (across his corpus)
as treating humanity, in oneself and others, always as an end and never merely as a means.
\emph{Critique of Practical Reason} articulates the same structure when it insists that the
moral law addresses us as free and self-legislating, never as mere instruments of inclination
or authority \parencite[pp.~30--33]{KantCPrR1996}. Transpose this into the film's grammar:
a mission-form that disables reciprocity and reduces persons to objects of procedure cannot be
rescued by neat outcomes. Willard's acknowledgment that it was a ``real choice mission''
intensifies the Kantian judgment---he cannot hide behind heteronomy or claim he was merely
following orders. The will freely adopted the maxim, making the subsequent exposure of its
failure absolute. ``Everyone gets everything he wants'' becomes, under Kant, not an excuse but
an indictment: autonomous choice revealed its own wrong orientation.

Kant's distinction between legality and morality intensifies this. An act can conform to the
law outwardly (legality) while lacking the right incentive (morality). What makes an action
moral is that its determining ground is respect for the moral law, not fear, habit, or advantage
\parencite[pp.~72--76]{KantCPrR1996}. The dossier scene is a study in outward conformity:
orders, signatures, the rhetoric of necessity. But the incentive that animates ``I wanted a
mission'' is not respect for law; it is a desire for orientation, relief from aimlessness,
and ultimately institutional recognition. When the mission is granted, fulfillment exposes
the incentive: instead of being moved by duty, the will was moved by a need to still its own
drift. ``For my sins I got one'' now reads as Kantian confession: I acted from heteronomy,
and success only made that visible.

Kant's moral psychology helps clarify why fulfillment can feel like punishment. Respect for
the law is an incentive that humbles self-love; it is experienced as a constraint on inclination
\parencite[pp.~70--73]{KantCPrR1996}. To the extent that the film's assignments cloak
inclination under moral language—security, order, ``surgical'' necessity—the later ``success''
functions as a de-masker: the will discovers it was not obeying a law it could legislate for
all, but rather serving a maxim it would never publicly endorse as universal. The tight,
affectless tone of Willard's narration after each ``win'' matches this discovery: the more
procedure works, the clearer it becomes that working isn't the same as willing rightly.

Kant's emphasis on autonomy sharpens the point. To be free is not to get what one wants, but
to give oneself a law that any rational agent could adopt \parencite[pp.~30--33]{KantCPrR1996}.
Measured this way, the mission-form is structurally tempting to heteronomy: it outsources
lawgiving to the institution and treats persons encountered en route as mere bearers of protocol.
Even when the mission targets someone like Kurtz—himself a violator of reciprocity—the maxim
``neutralize by assassination when the institution decrees'' cannot be a law for all, because
it erodes the very standing of rational agents as co-legislators. The fact that ``everyone gets
everything he wants'' in such a system is precisely the problem: it signals the reliable
availability of means for heteronomous ends.

Kant also insists that morality is not a ledger of effects but an orientation of maxims
sustained through adversity. This explains why the film's most chilling moments are not its
brutalities but its efficiencies: when the sampan search is executed by the book, the clean
line from maxim to act to outcome throws the wrong maxim into relief. Even if the damage were
minimized, the principle—instrumentalization under orders—would still fail the humanity
constraint. Fulfillment punishes because it removes the excuse of friction: the will must own
what it willed when everything ``worked.''

Does Kant leave any room for the line's first half—``Everyone gets everything he wants''—to
carry moral weight? Only in a highly restricted sense. If the ``want'' is already shaped by
the moral law—if the agent wants to act from a universalizable maxim out of respect for
persons—then ``getting what one wants'' is just the possibility to do one's duty. Otherwise,
success is morally insignificant at best and accusatory at worst. In context, the confession
``for my sins I got one'' catches this: the grant itself is the mirror that shows the will's
prior choice against autonomy.

Finally, Kant's idea of the highest good (happiness proportioned to virtue) underscores the
tragedy: the world does not guarantee any convergence between success and moral worth
\parencite[pp.~125--131]{KantCPrR1996}. The mission can be fully accomplished and still fail
the test of law; conversely, refusal might align with duty but bring ruin. This is why a
Kantian reading refuses consolation at the end: what matters is not that the project ended,
but whether the maxim survives scrutiny. By that light, ``Everyone gets everything he wants''
names a morally irrelevant fact about means and outcomes; ``\ldots and for my sins I got one''
names the relevant fact about the will that chose the maxim it did.

\newpage
\phantomsection
\subsection*{III.2—Schopenhauer: Fulfillment as Disclosure of Lack}
\addcontentsline{toc}{subsection}{III.2—Schopenhauer: Fulfillment as Disclosure of Lack}
\label{ssec:iii-schopenhauer}
Arthur Schopenhauer's analysis of will offers a rigorous grammar for Willard's confession:
``Everyone gets everything he wants. I wanted a mission, and for my sins they gave me one.''
For Schopenhauer, desire is not a teleology that culminates in peace but a mechanism whose
very satisfaction resets itself. Hence the mission granted is not a gift that stills the heart;
it is the next oscillation of a structure that cannot be stilled.

\subsubsection*{1) The structure of willing: lack $\rightarrow$ striving
	$\rightarrow$ relief $\rightarrow$ renewed lack}

``All willing springs from lack, from deficiency, and therefore from suffering''
\parencite[p.~196]{SchopenhauerWWR1969}. The object that seems to promise rest is already
implicated in the will's unease; when attained, it ``at once makes room for a new one''
\parencite[p.~319]{SchopenhauerWWR1969}. Schopenhauer's famed image is diagnostic rather
than rhetorical: life ``swings like a pendulum, to and fro between pain and boredom''
\parencite[p.~312]{SchopenhauerWWR1969}.

Read against the film's first movements, the pattern holds precisely. In Saigon, Willard's
lack is staged as agitation and intoxication; the dossier scene supplies an object (the mission)
and a narrative. Relief appears as orientation—a reason to move upriver—but immediately becomes
renewed lack: each checkpoint demands the next, each ``win'' opens a further deficit.
Willard's voiceover keeps the pendulum audible: completion never completes; it only
re-initiates striving.

\subsubsection*{2) Why fulfillment punishes: the object is a delusion of rest}

Schopenhauer emphasizes that satisfaction exposes, rather than heals, the structure of desire.
Enjoyment ``we have longed for'' soon leaves us ``bored,'' and the will ``returns to its old
course'' \parencite[p.~319]{SchopenhauerWWR1969}. In that sense, getting what one wanted hurts
because it removes the fantasy that the object could silence the will. The hurt is cognitive:
fulfillment disenchants.

The film thematizes this in miniature. The Playboy show promises heightened pleasure; the
immediate after-image is agitation and bargaining. Kilgore's beachhead produces tactical
success, but the spectacle (``I love the smell of napalm in the morning'') converts victory
into appetite. The sampan search yields compliance, then horror; the ``completed'' procedure
reveals leanness of soul. In Schopenhauer's terms, each fulfilled want punctures its own
promissory aura and re-installs the need to want.

\subsubsection*{3) Representation, will, and the aesthetic \& ethical brakes that fail in war}

Schopenhauer distinguishes the world as representation (ordered by the principle of sufficient
reason) from the world as will (blind striving) \parencite[pp.~3--5]{SchopenhauerWWR1969}.
Two brakes can mitigate the will's tyranny. First, aesthetic contemplation suspends willing by
fixing consciousness on the Idea—to ``lose oneself in the object''
\parencite[p.~178]{SchopenhauerWWR1969}. Second, compassion reframes the other not as instrument
but as a fellow bearer of suffering \parencite[pp.~372--374]{SchopenhauerWWR1969}.
Both brakes fail in the film's wartime economy. Wagner's ``Ride of the Valkyries'' is mobilized
as a stimulant for domination, not as will-suspending contemplation; the sampan protocol
subordinates pity to procedure. The very mechanisms that could have cooled willing are
conscripted by it.

\subsubsection*{4) Boredom, repetition, and the river as pendulum}

If pain signals unfulfilled desire, boredom signals desire's deflation after satisfaction.
Schopenhauer's claim that joy fades into ennui is not a counsel of mood but an analysis
of the will's metabolism \parencite[pp.~312--320]{SchopenhauerWWR1969}. The river sequences
enact this metabolism: periods of frantic danger (pain) alternate with slack stretches of
waiting (incipient boredom), and Willard's narration re-ignites the need for the next trial.
The Do Lung Bridge sequence literalizes the pendulum: building by day, destruction by night;
every ``achievement'' immediately generates its contrary.

\subsubsection*{5) ``For my sins I got one'': the inner necessity of punishment}

Schopenhauer does not require an external punisher. The punishment is inner: to ``get what one
wants'' is to have the will's insatiability revealed to oneself. Hence the tone of Willard's
clause; the ``sin'' is not only moral guilt but attachment to the fantasy that a mission could
deliver more than recurrence. The gift is therefore a judgment.

At Kurtz's compound, this inner necessity is complete. Kurtz has arranged the conditions for
maximal satisfaction (command, removal of obstacles) and finds that mastery produces only a
clarified view of the will's void: possession does not pacify. Willard's approach—labor, danger,
deprivation—does not redeem the will, but it strips away the last illusions about what
fulfillment can do. In Schopenhauer's lexicon, the world is disclosed as will precisely when
desire succeeds.

\subsubsection*{6) Objection \& counterpoint: is there any release?}

One might object that Schopenhauer allows two releases. First, aesthetic states offer
``deliverance'' from the will's press \parencite[p.~178]{SchopenhauerWWR1969}. Second,
compassion grounds ethics beyond egoistic striving \parencite[pp.~372--374]{SchopenhauerWWR1969}.
The film acknowledges both in negative: music becomes a tool for domination rather than
contemplation; pity is subordinated to protocol. The point is not that release is metaphysically
impossible, but that this narrative world is structured to block it; thus, fulfillment returns
as exposure.

\subsubsection*{7) Payoff for the thesis}

Schopenhauer thus illuminates the first half of Willard's sentence: \emph{everyone} (because
willing is universal) \emph{gets} (because objects are available) \emph{everything} (because
the will projects ``the all'' onto finite objects) he \emph{wants} (because wanting, not the
wanted, is fundamental). The second half—``for my sins I got one''—expresses the cognate of this
metaphysics: punishment is not denial but grant that unmasks the will's structure. The mission
is not a deviation from desire's grammar; it is the very form in which that grammar is made
visible.

\newpage
\subsection*{III—Hegel: From Object-Desire to Recognition, Mastery's Emptiness, and the Truth of
	Work}
\label{ssec:iii-hegel}
Hegel's decisive move is to show why fulfillment through possession cannot settle desire. In the
\emph{Phenomenology of Spirit}, self-consciousness first appears as desire that negates
otherness, but it learns that consuming things can never yield self-certainty:
``self-consciousness achieves its satisfaction only in another self-consciousness''
\parencite[\S 175]{HegelPhenomenology1977}. The thing I devour does not look back; it cannot
recognize me. If ``Everyone gets everything he wants'' is read as a promise of objects and
outcomes, Hegel's rejoinder is that objects are the wrong currency for the desire at stake.
The later clause—``\ldots for my sins I got one''—sounds like the experience of having obtained
the wrong coin.

Hegel dramatizes this transition in the struggle for recognition culminating in lordship and
bondage \parencite[\S\S 178--196]{HegelPhenomenology1977}. The combatants risk death because
only a being who risks its life shows that it is not bound to bare preservation. The so-called
Lord ``wins,'' but his victory is hollow: the Bondsman's submission is not free recognition
\parencite[\S\S 187--189]{HegelPhenomenology1977}. Mastery therefore ``gets what it
wants''—dominion—and
finds it empty of the very confirmation it sought. This emptiness is not psychological
disappointment; it is structural. Recognition that counts must be mutual between free subjects.
Where a project's logic—administrative or militarized—reduces others to functions, the more
perfectly it attains its end, the more sharply its lack of recognition appears.

The truth of self-consciousness, Hegel says, lies not with the Lord but with the Bondsman, who,
through fear, service, and formative work (\emph{Bildung}), mediates self and world
\parencite[\S 196]{HegelPhenomenology1977}. Work transforms the given without annihilating it;
it commits the self to a shared, durable world. The river journey's serial procedures—secure a
beach, clear a waterway, enforce a protocol—have the outer form of work, yet the world they
leave is not stabilized as a space of mutual recognition. The cycle at the bridge—construction
by day, erasure by night—parodies \emph{Bildung}: it produces, but it does not found.
Fulfillment here punishes by revealing the absence of the only recognition that could have
satisfied the desire that set the project in motion.

Hegel's dialectic also clarifies why the most ``efficient'' victories feel airless. Dominating
the other as instrument silences the very freedom from which recognition must come. Each success,
then, intensifies the contradiction: the more complete the procedure, the more total the other's
silencing, and the less possible the confirmation the agent craves. ``Everyone gets everything
he wants'' becomes tragic because the want was mis-specified: it sought certitude about self
through the mute success of operations. ``\ldots For my sins I got one'' is the moment mastery
confesses its own null confirmation.

\newpage
\phantomsection
\subsection*{III—Nietzsche: Fulfillment as Style of Will—Affirmation, Domination, Transvaluation}
\addcontentsline{toc}{subsection}{Nietzsche: Fulfillment as Style of Will}
\label{ssec:iii-nietzsche}
Nietzsche does not dispute Schopenhauer's observation that willing does not rest; he revalues
it. The problem is not desire's recurrence but our craving for a terminal perch that would end
the need to will. In this light, ``Everyone gets everything he wants'' becomes a diagnostic:
fulfillment reveals whether the will has the style to affirm its own recurrence, or whether it
smuggles domination in under the names of truth and duty. ``For my sins I got one'' is the
moment the mask of those names slips.

Nietzsche's stark claim—``man would rather will nothingness than not will at all''—positions
the refusal of willing as more intolerable to life than suffering
\gbparencite[III.28, p.~162]{NietzscheGenealogy1994}. The danger, then, is not the intensity of
volition but its self-deception: the will valorizes itself as ``truth'' so it can command
without admitting it. ``The desire for `truth' has hitherto been the most dangerous of all
possessions'' because it disguises a need to impose \parencite[\S 34]{NietzscheBGE1990}.
The Saigon briefing's cool rationality—dossiers, maps, a narrative of ``removing an
aberration''—is exemplary of this danger: a project of command is presented as neutral cognition.
When Willard ``wants a mission,'' the wanting is not epistemic; it is a pledge of will stamped
with the authority of ``truth.'' The sentence's first clause (``Everyone gets everything he
wants'') thus records not luck but the world's capacity to deliver the objects that our
valuations already framed as necessary; the second clause (``for my sins\ldots'') signals
the after-knowledge that those valuations were life-denying.

Against such self-deception, Nietzsche sets style—the capacity to shape one's evaluations when
reality exposes them as evasions. He urges a ``revaluation of all values''
\parencite[\S\S 203--211]{NietzscheBGE1990}, and the call to ``live dangerously!''
\parencite[\S 283]{NietzscheBGE1990} names a refusal of anesthetized security rather than a
cult of risk. In this register, fulfillment is not possession of the object but self-formation:
the will confirms itself by changing its own measure. The upriver progression continuously
offers occasions for such revaluation—each checkpoint turning success into a new claim on the
self. When compliance with procedure at the sampan yields horror, a Nietzschean response would
be to transvalue the maxim that licensed it. Instead, the will prefers continuity of command;
it ``gets what it wants'' (control, clarity) and is punished by the disclosure that its wanting
is reactive—obedience to inherited values that present themselves as necessity.

Nietzsche's psychology of ressentiment further clarifies the moralizing energies that travel
with domination. The weak, unable to act, invert impotence into virtue by calling the strong
``evil'' and themselves ``good'' for not doing what they cannot do
\gbparencite[I.10--14]{NietzscheGenealogy1994}. Yet he also describes a noble pathos that wants
to expand and test itself \parencite[\S\S 260--265]{NietzscheBGE1990}. In the film's middle
movements, these vectors cross: theatrical sovereignty stages itself as exuberance
(``I love the smell of napalm in the morning''), while the bureaucratic ``we'' that dispatches
the assassin wraps elimination in the moral language of purification. Both are forms of wanting
that the sentence anatomizes: one wants spectacle of command, the other wants justification for
command—but neither shows the transvaluative courage to alter its measure when outcomes strip
the rhetoric bare.

Kurtz, often read as the one who has gone ``beyond good and evil,'' in fact illustrates
Nietzsche's worry about the will's last refuge: after unseating inherited norms, it longs for
a final verdict that would secure mastery once more. Nietzsche's ``beyond good and evil'' is
not a license for cruelty; it is lucidity about the genealogy of one's values and the refusal
to enthrone a new absolute \parencite[\S\S 259--260]{NietzscheBGE1990}. Kurtz's pronouncement
of ``the horror'' behaves like that new absolute—a metaphysical seal on judgment that would
still the will's vulnerability. If he has ``got what he wants'' (freedom to rule, pronounce,
and be obeyed), fulfillment punishes by revealing the emptiness of mastery that will not
relinquish its last metaphysical crutch.

The health-criterion, for Nietzsche, is severe and simple: does this willing increase one's
capacity to affirm life—including ambiguity and pain—or does it shrink that capacity under a
rhetoric of truth and duty? A project may be perfectly ``true'' by institutional measures and
yet sick by this criterion. When the assignment is executed to the letter, and nothing
redemptive follows—no enlargement of perspective, no transvaluation of maxims—the confession
(``\ldots for my sins I got one'') reads as recognition that fulfillment has exposed the willing
as life-denying. In Nietzsche's terms, the punishment is not the mission's cost but its clarity:
getting what one wanted shows which kind of will one is.

\newpage
\subsection*{III—Kierkegaard: Despair as Absolutized Project—Why Success Thickens the
	Misrelation}
\label{ssec:iii-kierkegaard}
Kierkegaard treats despair not as a mood but as a structural error in how the self relates to
itself. ``The self is a relation that relates itself to itself,'' and it can be ``sick unto
death'' when that relation is grounded in the wrong power
\parencite[pp.~49--52]{KierkegaardSUD1980}.
Two principal forms of despair matter here: (1) the despair of weakness—``not to will to be
oneself,'' and (2) the despair of defiance—to will to be oneself ``in one's own strength''
\parencite[pp.~52--61, 69--73]{KierkegaardSUD1980}. Read against Willard's confession, the line
``Everyone gets everything he wants'' sketches the field on which both forms operate;
``\ldots and for my sins I got one'' is the moment the misrelation becomes clear through
success.

Kierkegaard's analysis bites hardest when a contingent project is taken as absolute. The self,
needing to be grounded ``in the power that established it,'' substitutes a finite end as its
measure and thereby misrelates itself \parencite[pp.~79--83]{KierkegaardSUD1980}. To will to be
the one who has a mission is precisely such absolutization. Before the assignment, the self
appears as lack (restless aimlessness); once the assignment is granted, the self congeals around
the project—orientation replaces drift. But in Kierkegaard's grammar this is not healing; it is
the despair of defiance: the self wills to be itself by itself through the project. The more
coherent the mission becomes, the more intense the misrelation grows, because the self is
secured by something that cannot finally ground it.

This is why, for Kierkegaard, success does not rescue but thickens despair. Success confirms
the illusion that one can be oneself by one's own project; yet every success is also a mirror,
showing that the self remains unfounded. Kierkegaard emphasizes that despair often hides beneath
``the most colossal energy'' and apparent resolve; it is ``misrelationship in a self'' that can
be ``perfectly transparent to itself'' about its project while being wrong about its ground
\parencite[pp.~72--76]{KierkegaardSUD1980}. In this light, the cool execution of procedures and
the narrowing of affect after each ``win'' signal not mastery but the tightening of defiant
despair. The line's second clause—``for my sins I got one''—reads as the moment when the grant
of the mission throws the absence of a true ground into relief.

Kierkegaard also distinguishes immediacy (living immersed in finite goods) from reflection that
can discover the self's task \parencite[pp.~84--90]{KierkegaardSUD1980}. War intensifies
immediacy by turning every face into a function and every act into a means; it encourages the
self to hide in roles. In such a milieu, the very practices that appear to deliver meaning—orders,
dossiers, operational clarity—supply the self with a surrogate infinity: a finite project that
pretends to be sufficient. But the self, for Kierkegaard, is tasked with becoming itself before
God \parencite[pp.~79--83]{KierkegaardSUD1980}. This is not a pietistic add-on; it is his way
of marking that the self's measure must transcend its own chosen ends. Whenever the measure is
reduced to the project's success, despair results—``the greater the natural capacities, the more
dangerous the despair'' \parencite[pp.~76--78]{KierkegaardSUD1980}.

Note how this frame explains the distinctive tonality of fulfillment-as-punishment. To ``get
what one wants'' is to lose the alibi that failure provides. As long as the project is ungranted,
the self can imagine that possession will establish it. Once granted, the self's emptiness
becomes undeniable. The confession ``for my sins I got one'' thus does not (primarily) express
guilt over discrete acts; it expresses recognition of a wrong willing—to be oneself by one's own
finite project. That is Kierkegaard's ``sin'' in the strict sense: not a single deed, but a
posture of self-grounding \parencite[pp.~79--83]{KierkegaardSUD1980}.

Kierkegaard's analysis also clarifies why horror does not teach the defiant self what it most
needs to learn. The self in defiance is willing to suffer anything rather than relinquish its
chosen measure; it would rather ``be itself with all the torments of hell than not be itself''
\parencite[p.~69]{KierkegaardSUD1980}. Hence the spectacle of a self that persists—relentlessly,
competently—through increasingly unredeeming outcomes. The punishment of fulfillment is that
competence becomes the instrument of despair: every efficient act confirms the sovereignty of
the project, and every confirmation deepens the misrelation.

Is there a Kierkegaardian way out? Only if the maxim of the project is converted—a re-grounding
of the self in that ``power which established it,'' which, in his lexicon, entails repentance
and a change in the measure of willing \parencite[pp.~79--83, 111--116]{KierkegaardSUD1980}.
Short of such a conversion, neither failure nor success can heal; success merely strips away
the illusion that success could heal. Thus the two halves of the sentence lock together:
``everyone gets everything he wants'' = finite ends can indeed be obtained; ``for my sins I got
one'' = obtaining them revealed the despair that absolutized them.

\newpage
\phantomsection
\subsection*{III.6—Dostoevsky: Independent Desire, Anti-Mechanism, and Agency That Eats Itself}
\addcontentsline{toc}{subsection}{III.6—Dostoevsky: Independent Desire and Agency}
\label{ssec:iii-dostoevsky}
Fyodor Dostoevsky's \emph{Notes from Underground} is the classic anatomy of a will that prefers
independence to well-being—``man only wants independent desire, whatever that independence may
cost'' \parencite[p.~131]{DostoevskyNFU1994}. The underground man's most scandalous claim—``To
hell with two times two makes four!'' \parencite[p.~129]{DostoevskyNFU1994}—is not
anti-arithmetic; it is anti-mechanism in human affairs. He rejects any calculus in which
rational prediction, utility, or institutional procedure would close the space of spontaneous
willing. Read in this key, the line ``Everyone gets everything he wants'' marks not prosperity
but a world well-stocked with mechanisms that deliver objects on demand; ``for my sins I got
one'' acknowledges the price of willing agency itself within such machinery.

The underground man's revolt targets the dream that human conduct can be rendered scientific—that
motives can be predicted and optimized so that ``good'' outcomes follow from the right levers
\parencite[pp.~120--132]{DostoevskyNFU1994}. His point is not that people are irrational, but
that personhood includes a residual freedom that will ``assert itself'' against the system,
even destructively, simply to prove it exists \parencite[pp.~129--132]{DostoevskyNFU1994}.
This is why a ``crystal palace'' of perfect provisions would provoke sabotage; the human being,
he insists, will sometimes choose what is harmful to demonstrate authorship. The dossier room's
hygienic proceduralism—clarity of ends, chain of command, calibrated means—inhabits precisely
the rational order the underground man distrusts. Willard's wanting a mission is not a longing
for certainty alone; it is the chance to act, to break the inertia of Saigon's aimlessness,
even if the action risks moral injury. The ``sin'' is that the system can supply just such
occasions and call the result necessary.

Dostoevsky's dialectic also explains the peculiar pleasure the underground man takes in acting
against his own interest: ``the most advantageous advantage'' is sometimes the freedom to choose
what is not advantageous \parencite[pp.~129--131]{DostoevskyNFU1994}. Agency is not measured by
outcomes but by the felt authorship of choice. That is why, on his account, a rational program
that secures only beneficial outcomes is degrading: it would reduce a man to a ``piano key'' on
which nature (or the institution) plays \parencite[pp.~115--120]{DostoevskyNFU1994}.
In an environment of orders and protocols, the upriver insistence on continuing, whatever the
evidence or cost, becomes intelligible as a defense of non-instrumentality: continuing proves
that one is not merely a key. Fulfillment punishes because the moment the system grants the
mission, the space of ``independent desire'' shrinks into the execution of a mechanism that
now owns the storyline.

Yet Dostoevsky's insight is double-edged. The underground man's independence is real, but it
corrodes itself when it refuses any measure beyond negation. He confesses to relishing
humiliation and spite, to savoring ``the sweetly painful pleasure'' of acting against himself
\parencite[pp.~108--115]{DostoevskyNFU1994}. This is agency that proves itself by injury.
When procedure yields horror (the sampan), the rational mechanism has plainly failed; but if
the next choice is animated only by the need to keep asserting agency—``I go on because I go
on''—the will ratifies the same emptiness the underground man inhabits. ``Everyone gets
everything he wants'' then describes a trap: the institution gets its obedient executor; the
agent gets the feeling of authorship; neither gets a norm by which the act could be vindicated.

Dostoevsky also anticipates what we might call moral theater: the staging of motives after the
fact to render destructive agency palatable. The underground man is merciless about his own
self-narration—confessing how quickly the ego invents edifying reasons for what was, in truth,
caprice or spite \parencite[pp.~103--107]{DostoevskyNFU1994}. This maps onto the rhetoric that
frames the assignment as sanitary necessity. The mask of moral purification (``remove an
aberration'') converts the hunger to act into an edifying plot; fulfillment then functions as
exposure when, at the end, the narrative yields no enlargement of soul. The confession ``for my
sins I got one'' reads, in Dostoevskian terms, as the recognition that the story was a postscript
to the will to act, not its ground.

A further Dostoevskian thread concerns responsibility under conditions of determinism. The
underground man refuses to let causal explanation excuse him: even if he can trace motives, he
will not permit the explanation to replace ownership \parencite[pp.~109--113]{DostoevskyNFU1994}.
This refusal illuminates the post-fulfillment chill: once the mission is complete, the agent
cannot hide in causal chains (``orders,'' ``procedure,'' ``necessity'') without committing the
very self-abdication he despises. Fulfillment punishes by removing alibis: the deed is done;
the authorship is mine.

Finally, Dostoevsky's anti-mechanism clarifies why the film's most efficient scenes are its most
disturbing. The underground man's nightmare is not chaos but perfect order—an order so seamless
it leaves no room for non-instrumental choice. Where everything ``works,'' the human residue can
only show itself by breaking the system or by converting obedience into a performance of will.
In either case, getting what one wanted discloses a deficit: agency defended merely as
independence becomes self-consuming. The sentence's halves therefore lock: the world can indeed
deliver the occasion to act (``everyone gets\ldots''), but the one who wanted agency itself
discovers, upon receiving it, that agency without a measure is indistinguishable from compulsion
in disguise—hence ``\ldots for my sins I got one.''

\newpage
\subsection*{III—Sartre: Freedom, Bad Faith, and the Impossible Completion}
\label{ssec:iii-sartre}
For Sartre, human reality is “what it is not and not what it is” (\parencite[pp.~100--110]{SartreBN2003}); projects tacitly aim at an impossible synthesis—the “project to be God” (\parencite[pp.~586--604]{SartreBN2003}). Completion cannot grant ontological closure; it exposes bad faith, whether as pure function (“I am my orders”) or as sovereign exception (“I am exempt”). The mission’s end therefore punishes by lucidity: the sequence “worked,” yet the lack constitutive of the \emph{pour-soi} remains.

\newpage
\subsection*{III—Beauvoir: Reciprocity as Freedom’s Form}
\label{ssec:iii-beauvoir}
Beauvoir internalizes ethics to the structure of freedom: “To will oneself free is also to will others free” (\parencite[p.~73]{Beauvoir1976}). Authentic projects \emph{open} situations in which others can transcend; efficient means that \emph{close} horizons convict themselves by their very success (\parencite[pp.~134--147, 157--161, 164--173]{Beauvoir1976}). The first clause reports reliable means; the second is the ethical verdict that reciprocity was excluded from the end.

\newpage
\subsection*{III—Camus: Absurd Lucidity and Action without Appeal}
\label{ssec:iii-camus}
The absurd is “born of the confrontation between the human need and the unreasonable silence of the world” (\parencite[p.~28]{CamusSisyphus1991}). To live “without appeal” (\parencite[p.~54]{CamusSisyphus1991}) is to abandon the hope that completion provides a final court of justification. The Do Lung Bridge cycle—building and erasure—reads like a Sisyphean figure; completion yields knowledge, not meaning (\parencite[pp.~121--123]{CamusSisyphus1991}). Thus the sentence’s first clause can be true; the second names fulfillment as the world’s silence.

\newpage
\phantomsection
\subsection*{III.10—Heidegger: Finitude, Anticipatory Resoluteness, and Why Completion Is
	Ontologically Out of Reach}
\addcontentsline{toc}{subsection}{III.10—Heidegger: Finitude and Resoluteness}
\label{ssec:iii-heidegger}
Martin Heidegger's account of existence (Dasein) makes completion a category mistake. Dasein is
essentially being-possible—a projecting that never coincides with itself as a finished thing;
its wholeness is disclosed only in being-toward-death \parencite[pp.~279--311]{HeideggerBT1962}.
Death is not (primarily) a future event to be scheduled but the ownmost, nonrelational
possibility that individualizes Dasein now, stripping away the illusions of totalization
\parencite[pp.~294--307]{HeideggerBT1962}. Thus, any project that promises narrative wholeness—that
a mission will ``make it come together''—misreads existence. When Willard says ``Everyone gets
everything he wants,'' the Heideggerian gloss is brutal: the world may indeed supply objects and
tasks, but existence is not something an object can finish. ``\ldots And for my sins I got one''
is the moment the project's alleged telos collides with finitude.

Heidegger's analysis of everydayness and the They (\emph{das Man}) clarifies why projects so
easily wear the mask of necessity. In average everydayness, Dasein takes over possibilities ``as
one does,'' letting anonymous norms dictate what counts as urgent, clean, or right
\parencite[pp.~149--168]{HeideggerBT1962}. The Saigon briefing's procedural tone—dossiers,
signatures, the grammar of sanitation—exemplifies this absorption in \emph{das Man}: the mission
shows up as what ``one'' does when a file reads anomalous. To take it up as such is not yet
resolute choice; it is fallenness into the ready-made interpretation. Fulfillment then
``punishes'' by disclosing that the accomplished sequence was never a route to owned wholeness;
it was a they-self rhythm all along.

The film's temporality maps onto Heidegger's account of ecstatic time. Dasein's temporality is
not a string of nows but an ``ahead-of-itself'' (future), already-in (past), and being-alongside
(present) \parencite[pp.~373--383]{HeideggerBT1962}. The Do Lung Bridge cycle—construction by
day, destruction by night—stages a caricature of inauthentic time: a serial present that never
gathers. Anticipatory resoluteness does not end such cycles; it interprets them soberly by
owning death as the limit that prevents totalization \parencite[pp.~307--311]{HeideggerBT1962}.
By this light, the climactic ``success'' cannot heal the fracture; it can only remove the alibi
that failure once provided. The felt judgment of the line is that clarity: the project is
complete and therefore unable to hide the truth that existence cannot be.

Heidegger's conscience and guilt intensify the point. Conscience ``calls'' Dasein from \emph{das
	Man} to its ownmost possibility; guilt (\emph{Schuld}) names not juridical fault but
being-the-basis of a nullity—that our thrown projection always leaves something out and cannot
guarantee innocence \parencite[\S\S 57--60, pp.~311--354]{HeideggerBT1962}. When procedures run
perfectly (the sampan inspection as ``by the book'') and still yield devastation, what is
revealed is not only moral failure but ontological mismatch: the attempt to secure existential
rightness via technical closure. Anticipatory resoluteness would require owning that mismatch,
not masking it with narratives of cleansing. The confession—``for my sins I got one''—is a
resolute sentence in this sense: it drops the promise of narrative wholeness and accepts
finitude as the horizon that renders completion impossible.

\newpage
\phantomsection
\subsection*{III.11—Levinas: The Face's Prohibition, Asymmetrical Responsibility, and Why
	``Success'' Condemns Instrumental Projects}
\addcontentsline{toc}{subsection}{III.11—Levinas: The Face and Asymmetrical Responsibility}
\label{ssec:iii-levinas}
Emmanuel Levinas relocates first philosophy from ontology to ethics: the encounter with the face
institutes an asymmetrical demand prior to any project or knowledge. ``Desire is desire for the
absolutely other'' \parencite[p.~33]{LevinasTI1969}, and the face ``forbids us to kill''
\parencite[p.~199]{LevinasTI1969}. This is not a thesis about consequences but a command
inscribed in the presentation of the other as infinite—irreducible to roles, functions, or my
plans \parencite[pp.~194--201]{LevinasTI1969}. Measured by this standard, ``Everyone gets
everything he wants'' is ethically null until we ask whether what was wanted preserved the
other's irreducibility; ``\ldots and for my sins I got one'' reads as the moment when a granted
project reveals, by its own success, that it had bracketed that demand.

Levinas's notion of totality versus infinity names the fault-line. Totality is the regime that
reduces alterity to the Same—catalogues, protocols, categories; infinity is the breach of that
reduction in the epiphany of the face \parencite[pp.~21--24, 33--36]{LevinasTI1969}. The
mission-form—dossier, diagnosis, elimination—is quintessentially totalizing: it metabolizes
faces as data points and tasks. The sampan scene is an X-ray: even before the fatal shot, the
encounter runs on risk calculus. In Levinas's grammar, the ethical failure precedes the mistake;
the very mode of approach ``has already spoken'' by refusing the face's claim. Success cannot
redeem such refusal; it confirms it. ``Getting what one wants'' within this regime is punishment
as self-revelation: the act returns to the agent as accusation.

Levinas is explicit that the ethical relation is asymmetrical: I am responsible for the other
beyond reciprocity or contract \parencite[pp.~215--219]{LevinasTI1969}. This asymmetry is
precisely what proceduralism neutralizes, since procedures aim to distribute liability
symmetrically. Hence the peculiar chill of the film's most efficient moments: where a protocol
works, the asymmetry has been most thoroughly suppressed. The ethical demand has not been
answered; it has been absorbed—turned into a variable among others. Levinas's insistence that
the face is a ``poor one, a stranger'' \parencite[p.~213]{LevinasTI1969} gives content to the
felt wrongness of treating villagers, boat crews, and even soldiers as means for the continuity
of the project. The wrongness is not (only) that harm occurs; it is that the form of encounter
precluded responsibility before deciding what to do.

The assassination order against Kurtz does not escape this logic by turning against a tyrant.
Levinas's ``Thou shalt not kill'' is not a rule applied to friends but the structure of
encounter itself \parencite[p.~199]{LevinasTI1969}. To meet anyone—enemy included—first as a
bearer of exteriority is to be summoned to justification. There may be cases, Levinas allows,
where politics demands force; but politics is always under judgment by ethics
\parencite[pp.~21--24]{LevinasTI1969}. The film's denouement shows the inversion: politics
judges ethics, and efficiency is taken as justification. That is why the line's second half
sounds like a verdict: ``\ldots for my sins I got one'' acknowledges that the project's end
never included the first relation—the face's command—so its successful completion can only
declare that exclusion more clearly.

Levinas also explains why horror often clarifies rather than teaches in such worlds. Horror
strips away alibis and yet, without a conversion of the mode of approach, it cannot generate the
responsibility it reveals. The proper response is not a grand theory but a change in the grammar
of encounter—hospitality, attention, refusal of instrumentalization
\parencite[pp.~200--206]{LevinasTI1969}. In their absence, ``Everyone gets everything he wants''
remains the slogan of totality: the world is very good at supplying means. The punishment of
fulfillment is the renewed summons one cannot now un-hear.

\newpage
\subsection*{III—Koj\`eve: Desire of Desire and Historical Stakes}
\label{ssec:iii-kojeve}
Human desire is the desire of another’s desire; satisfaction requires being recognized as free by a free other, not merely possessing objects (\parencite[pp.~6, 27--34]{KojeveIRH1980}). Hence the master’s “victory” is empty: coerced recognition is not recognition (\parencite[pp.~158--164]{KojeveIRH1980}). “Everyone gets everything he wants” marks an apparatus expert at distributing missions; “…and for my sins I got one” is the discovery that such distribution cannot secure the recognitive relation the human wants.

\newpage

\phantomsection
\subsection*{Comparative Discussion}
\addcontentsline{toc}{subsection}{Comparative Discussion}
\label{ssec:iii-comparative-discussion}

What follows sets the line against competing accounts of desire, normativity, selfhood, time,
ethics, and intersubjectivity, comparing and contrasting along these conceptual bearings to see
what ``getting'' and ``punishment'' amount to in each.

\subsubsection*{Insatiability or Transvaluation?}
\label{sssec:insatiability-or-transvaluation}

Schopenhauer hears in willing a mechanism that cancels its own promise---satisfaction ``at once
makes room for a new one,'' so life swings ``between pain and boredom'' \parencite[pp.~312,
319]{Schopenhauer1969}. The upriver chain of ``wins'' looks like his pendulum: each success opens
a new deficit. Nietzsche objects that such fatalism misconstrues the task: recurrence is not a
curse if the will revalues itself from consumption to creation; what corrodes willing is the mask
of ``truth'' that licenses domination \parencite[\S 34]{NietzscheBGE1990}. The Saigon dossier's
hygienic tone favors Nietzsche's suspicion: command in the costume of cognition. Yet when the
narrative never transvalues after the sampan or the bridge, Schopenhauer's phenomenology
reasserts itself: fulfillment disenchants. Camus cuts between them: even a creative will must live
``without appeal''---no final sanction comes with completion \parencite[pp.~28, 54]{CamusMyth1991}.
The film sides with Camus at the end: the world delivers objects; what returns is lucidity, not
meaning \parencite[pp.~121--123]{CamusMyth1991}.

\subsubsection*{Success as Evidence---or as Indictment?}
\label{sssec:success-as-evidence-or-indictment}

Kant denies that outcomes ever certify worth: the good will is ``good \ldots\ in itself,'' not
``because of what it effects'' \parencite[p.~27]{KantGMM1997}. The sampan inspection, polished as
procedure, fails at the maxim level: persons used merely as means violates the humanity constraint
\parencite[pp.~30--33, 72--76]{KantGMM1997}. Nietzsche warns, however, that Kantian talk of law
can smuggle in a will to command---``the desire for `truth''' as a tool of domination
\parencite[\S 34]{NietzscheBGE1990}. The briefing room shows why both are needed: Nietzsche
unmasks rhetoric; Kant supplies the tribunal that still condemns the maxim after unmasking.
Kierkegaard adds an internal critique: even if the maxim passed, absolutizing a finite project
thickens despair \parencite[pp.~69--83]{KierkegaardSUD1980}. So success cannot vindicate (Kant),
unmasking cannot excuse (Nietzsche), and even ``right'' structure cannot cure misrelation
(Kierkegaard)---a triple pressure that makes ``\ldots for my sins I got one'' sound like the
removal of alibis.

\subsubsection*{Absolutized Projects and Agency Without Measure}
\label{sssec:absolutized-projects-and-agency-without-measure}

Kierkegaard and Dostoevsky agree that the self can destroy itself through success, but they
quarrel over the disease. For Kierkegaard, despair is a misrelation: to will to be oneself ``in
one's own strength'' by way of a project \parencite[pp.~69--73]{KierkegaardSUD1980}. Every
tactical win upriver further tightens this wrong grounding. Dostoevsky emphasizes anti-mechanism:
``man only wants independent desire,'' even against interest; the human refuses to be a ``piano
key'' \parencite[pp.~115, 129--131]{DostoevskyNFU1994}. The dossier machine offers exactly what he
fears: a rational program that absorbs agency into function. Yet the Underground Man's
medicine---negation for its own sake---corrodes too; agency defended as pure independence collapses
into self-harm. The film dramatizes both errors: obedience performed as authorship (Dostoevsky's
nightmare), and identity collapsed into the project (Kierkegaard's despair). Fulfillment exposes
both at once.

\subsubsection*{The Impossible Telos of Completion}
\label{sssec:the-impossible-telos-of-completion}

Sartre and Heidegger converge that completion is a category mistake, but for different
reasons---and their difference matters. For Sartre, the \emph{pour-soi} is ``what it is not and
not what it is'' \parencite[pp.~100--110]{SartreBN2003}; the tacit ``project to be God''
\parencite[pp.~586--604]{SartreBN2003} seeks a synthesis that cannot exist. The mission's end
punishes by revealing that impossibility. Heidegger roots the error in finitude: Dasein's wholeness
is disclosed only in being-toward-death, which individualizes now and forbids narrative
totalization \parencite[pp.~294--307]{HeideggerBT1962}. Where Sartre indicts a wish for
ontological closure, Heidegger indicts the \emph{they}-authorized fantasy that a right sequence of
``what one does'' could yield wholeness \parencite[pp.~149--168]{HeideggerBT1962}. The felt vacuum
after clean procedures speaks both languages: Sartrean exposure of bad faith and Heideggerian
removal of the \emph{das Man} alibi.

\subsubsection*{Reciprocity vs. Instrumentality}
\label{sssec:reciprocity-vs-instrumentality}

Beauvoir and Levinas both condemn instrumental projects but argue from different first
principles---and their friction refines the verdict. Beauvoir builds the other into freedom's form:
``To will oneself free is also to will others free'' \parencite[p.~73]{Beauvoir1976}. Authentic
projects open situations where others can transcend; efficient means that close horizon indict
themselves \parencite[pp.~134--147, 157--161, 164--173]{Beauvoir1976}. Levinas says even that
framework comes too late: the face's ``Thou shalt not kill'' precedes projects and resists
assimilation to any totality \parencite[pp.~21--24, 33, 199]{LevinasTI1969}. The dossier world is
already a betrayal because it sees a case, not a face. Beauvoir answers that without reciprocal
world-building, Levinasian command risks impotence; Levinas counters that ``world-building'' easily
re-totalizes. The film lets both charges land: procedures flatten alterity (Levinas), and successes
never found a shared world (Beauvoir). ``Everyone gets\ldots'' thus names means without co-agency;
``\ldots for my sins\ldots'' marks the ethical cost either way.

\subsubsection*{Asymmetry, the Face, and the First Relation}
\label{sssec:asymmetry-the-face-and-the-first-relation}

Levinas re-situates judgment before ontology: the face confronts me with an asymmetrical
demand---``forbids us to kill''---that resists absorption into my plans \parencite[pp.~33,
199]{LevinasTI1969}. The dossier/protocol world belongs to totality, which reduces alterity to the
Same \parencite[pp.~21--24, 33--36]{LevinasTI1969}. Within that mode, success itself confirms
refusal of the first relation; punishment is the deed returning as accusation. Politics may require
force, but politics is always under ethical judgment \parencite[pp.~21--24,
215--219]{LevinasTI1969}. The confession is thus not merely ontological disappointment or Kantian
heteronomy; it is acknowledgment of a primary ethical breach structured into the wanting.

\subsubsection*{Possession or Recognition?}
\label{sssec:possession-or-recognition}

Hegel and Koj{\`e}ve insist the desired satisfaction is not of things but of recognition:
``self-consciousness achieves its satisfaction only in another self-consciousness''
\parencite[\S 175]{HegelPhenomenology1977}. Lordship gets obedience and finds it void---submission
is not free acknowledgment \parencite[\S\S 187--189]{HegelPhenomenology1977}; the ``truth'' lies
with formative work that builds a common world \parencite[\S 196]{HegelPhenomenology1977}. Levinas
worries this reciprocity reinscribes alterity into system; Hegel replies that without mediation
there is no world in which freedom can appear. The film's mission-form flunks both sides: it
routinizes asymmetry, so recognition cannot stabilize (Hegel/Koj{\`e}ve), and it approaches others
as material, so the first ethical relation is refused \parencite[pp.~21--24, 199]{LevinasTI1969}.
Hence the peculiar hollowness of ``getting what one wants'': the currency (objects, effects,
dominion) cannot purchase what a human desire seeks (free acknowledgment), and the very apparatus
that ensures delivery ensures refusal.

\subsubsection*{Minimal Norms for ``Non-Punitive'' Fulfillment}
\label{sssec:minimal-norms-for-non-punitive-fulfillment}

Gathering the threads yields three negative tests and one positive norm:

\begin{enumerate}
\item \textbf{Anti-mastery (Hegel/Koj{\`e}ve):} Fulfillment that reduces the other to instrument
secures only empty recognition; it will punish in exposure.

\item \textbf{Anti-violation (Levinas):} Fulfillment that ignores the face's prohibition is
ethically null; the punishment is accusation within the self.

\item \textbf{Anti-bad-faith (Sartre):} Fulfillment that disavows its own freedom is flight;
exposure returns as nausea, not peace.

\item \textbf{Pro-reciprocity (Beauvoir):} Only projects that will others free transmute
fulfillment from possession into co-realization.
\end{enumerate}

These norms do not reconcile the traditions; they articulate a practical sieve for projects. Under
this sieve, the sentence ``Everyone gets everything he wants'' ceases to be fatalism and becomes
a diagnostic: in getting the project, one learns whether the project's structure could have yielded
anything but punishment.

\subsubsection*{Implications for Willard's Utterance}
\label{sssec:implications-for-willard-s-utterance}

Taken together, the debates constrain what the sentence can mean in this narrative world. The
film's reliable delivery of objects and outcomes accords with Schopenhauer's phenomenology of
post-fulfillment deflation and with Nietzsche's warning about a will that refuses to revalue
itself; yet the very efficiency of means proves normatively empty on Kant's measure. Success
thickens, rather than cures, the self's misrelation in Kierkegaard's sense and invites the kind of
agency without measure that Dostoevsky anatomizes; it cannot, in Sartre's ontology, grant the
closure it promises. Where Beauvoir and Levinas relocate judgment to the form of willing,
instrumental success becomes self-indicting, independent of results. Hegel and Koj{\`e}ve finally
explain the peculiar hollowness of the achieved objective: objects and dominion are the wrong
currency for a recognitive desire. In this light, the first clause---``everyone gets everything he
wants''---is descriptively true only because a totalizing order is good at furnishing means; the
second---``\ldots and for my sins I got one''---registers the moment at which that same success
exposes a misdirected wanting: one that asked from fulfillment what only lawful maxims, reciprocal
freedom, or mutual recognition could provide. The line's force is thus diagnostic rather than
aphoristic; it names not just a fact about acquisition but a judgment about the will that sought
it.


