\phantomsection
\subsection*{IV.1—Kant: Duty, Autonomy, and Why ``Success'' Proves Nothing}
\addcontentsline{toc}{subsection}{IV.1—Kant: Duty, Autonomy, and Success}
\label{ssec:iii-kant}
Immanuel Kant gives the sharpest rebuke to reading fulfillment as vindication. In his moral
philosophy, the worth of an action lies not in what it achieves but in the maxim from which
it is done—the
principle the agent could will as universal law. The line ``Everyone gets everything he wants''
therefore cannot count as evidence that wanting was justified; ``for my sins I got one'' sounds
like the dawning recognition that having one's ends granted can lay bare a prior failure of
duty.

Kant's baseline claim is that the good will is ``good \ldots\ in itself,'' not by the
advantages it produces \parencite[p.~27]{KantCPrR1996}. This relocates ethical assessment away
from effects (which are subject to luck, power, and circumstance) to the will's legislation of
its own maxim. When Willard accepts the assignment in the cool, procedural light of Saigon,
the scene supplies everything success-friendly ethics likes—clarity of ends, chain of command,
legal sanction. For Kant, none of that matters morally. The question is simple and brutal:
What maxim am I adopting, and can I will it as a law for all rational agents? If the maxim is
``Eliminate as a means any person my institution designates an obstacle,'' universalization
collapses into contradiction: it destroys the very conditions of mutual recognition that a law
for all would require. The mission can be ``successful'' and still be morally void.

This is the force of Kant's second test—the humanity constraint, expressed (across his corpus)
as treating humanity, in oneself and others, always as an end and never merely as a means.
\emph{Critique of Practical Reason} articulates the same structure when it insists that the
moral law addresses us as free and self-legislating, never as mere instruments of inclination
or authority \parencite[pp.~30--33]{KantCPrR1996}. Transpose this into the film's grammar:
a mission-form that disables reciprocity and reduces persons to objects of procedure cannot be
rescued by neat outcomes. Willard's acknowledgment that it was a ``real choice mission''
intensifies the Kantian judgment---he cannot hide behind heteronomy or claim he was merely
following orders. The will freely adopted the maxim, making the subsequent exposure of its
failure absolute. ``Everyone gets everything he wants'' becomes, under Kant, not an excuse but
an indictment: autonomous choice revealed its own wrong orientation.

Kant's distinction between legality and morality intensifies this. An act can conform to the
law outwardly (legality) while lacking the right incentive (morality). What makes an action
moral is that its determining ground is respect for the moral law, not fear, habit, or advantage
\parencite[pp.~72--76]{KantCPrR1996}. The dossier scene is a study in outward conformity:
orders, signatures, the rhetoric of necessity. But the incentive that animates ``I wanted a
mission'' is not respect for law; it is a desire for orientation, relief from aimlessness,
and ultimately institutional recognition. When the mission is granted, fulfillment exposes
the incentive: instead of being moved by duty, the will was moved by a need to still its own
drift. ``For my sins I got one'' now reads as Kantian confession: I acted from heteronomy,
and success only made that visible.

Kant's moral psychology helps clarify why fulfillment can feel like punishment. Respect for
the law is an incentive that humbles self-love; it is experienced as a constraint on inclination
\parencite[pp.~70--73]{KantCPrR1996}. To the extent that the film's assignments cloak
inclination under moral language—security, order, ``surgical'' necessity—the later ``success''
functions as a de-masker: the will discovers it was not obeying a law it could legislate for
all, but rather serving a maxim it would never publicly endorse as universal. The tight,
affectless tone of Willard's narration after each ``win'' matches this discovery: the more
procedure works, the clearer it becomes that working isn't the same as willing rightly.

Kant's emphasis on autonomy sharpens the point. To be free is not to get what one wants, but
to give oneself a law that any rational agent could adopt \parencite[pp.~30--33]{KantCPrR1996}.
Measured this way, the mission-form is structurally tempting to heteronomy: it outsources
lawgiving to the institution and treats persons encountered en route as mere bearers of protocol.
Even when the mission targets someone like Kurtz—himself a violator of reciprocity—the maxim
``neutralize by assassination when the institution decrees'' cannot be a law for all, because
it erodes the very standing of rational agents as co-legislators. The fact that ``everyone gets
everything he wants'' in such a system is precisely the problem: it signals the reliable
availability of means for heteronomous ends.

Kant also insists that morality is not a ledger of effects but an orientation of maxims
sustained through adversity. This explains why the film's most chilling moments are not its
brutalities but its efficiencies: when the sampan search is executed by the book, the clean
line from maxim to act to outcome throws the wrong maxim into relief. Even if the damage were
minimized, the principle—instrumentalization under orders—would still fail the humanity
constraint. Fulfillment punishes because it removes the excuse of friction: the will must own
what it willed when everything ``worked.''

Does Kant leave any room for the line's first half—``Everyone gets everything he wants''—to
carry moral weight? Only in a highly restricted sense. If the ``want'' is already shaped by
the moral law—if the agent wants to act from a universalizable maxim out of respect for
persons—then ``getting what one wants'' is just the possibility to do one's duty. Otherwise,
success is morally insignificant at best and accusatory at worst. In context, the confession
``for my sins I got one'' catches this: the grant itself is the mirror that shows the will's
prior choice against autonomy.

Finally, Kant's idea of the highest good (happiness proportioned to virtue) underscores the
tragedy: the world does not guarantee any convergence between success and moral worth
\parencite[pp.~125--131]{KantCPrR1996}. The mission can be fully accomplished and still fail
the test of law; conversely, refusal might align with duty but bring ruin. This is why a
Kantian reading refuses consolation at the end: what matters is not that the project ended,
but whether the maxim survives scrutiny. By that light, ``Everyone gets everything he wants''
names a morally irrelevant fact about means and outcomes; ``\ldots and for my sins I got one''
names the relevant fact about the will that chose the maxim it did.
