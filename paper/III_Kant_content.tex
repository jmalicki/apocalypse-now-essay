\subsection*{III—Kant: Maxims, Humanity, and the Irrelevance of Outcomes}
\label{ssec:iii-kant}
For Kant, no accumulation of successes can certify moral worth: the good will is “good … in itself,” not “because of what it effects” (\parencite[p.~27]{KantCPrR1996}). The decisive question is the maxim—can it be willed as universal law, and does it honor humanity “always as an end and never merely as a means” (\parencite[pp.~27--33]{KantGroundwork1996})? The sampan inspection is instructive: meticulous adherence to procedure cannot redeem a maxim that objectifies persons (\parencite[pp.~30--33, 72--76]{KantCPrR1996}). The first clause is morally mute (efficacy, not legitimacy); the second names exposure of heteronomy in the moment of success.
