\phantomsection
\subsection*{IV.11—Levinas: The Face's Prohibition, Asymmetrical Responsibility, and Why
	``Success'' Condemns Instrumental Projects}
\addcontentsline{toc}{subsection}{IV.11—Levinas: The Face and Asymmetrical Responsibility}
\label{ssec:iii-levinas}
Emmanuel Levinas relocates first philosophy from ontology to ethics: the encounter with the face
institutes an asymmetrical demand prior to any project or knowledge. ``Desire is desire for the
absolutely other'' \parencite[p.~33]{LevinasTI1969}, and the face ``forbids us to kill''
\parencite[p.~199]{LevinasTI1969}. This is not a thesis about consequences but a command
inscribed in the presentation of the other as infinite—irreducible to roles, functions, or my
plans \parencite[pp.~194--201]{LevinasTI1969}. Measured by this standard, ``Everyone gets
everything he wants'' is ethically null until we ask whether what was wanted preserved the
other's irreducibility; ``\ldots and for my sins I got one'' reads as the moment when a granted
project reveals, by its own success, that it had bracketed that demand.

Levinas's notion of totality versus infinity names the fault-line. Totality is the regime that
reduces alterity to the Same—catalogues, protocols, categories; infinity is the breach of that
reduction in the epiphany of the face \parencite[pp.~21--24, 33--36]{LevinasTI1969}. The
mission-form—dossier, diagnosis, elimination—is quintessentially totalizing: it metabolizes
faces as data points and tasks. The sampan scene is an X-ray: even before the fatal shot, the
encounter runs on risk calculus. In Levinas's grammar, the ethical failure precedes the mistake;
the very mode of approach ``has already spoken'' by refusing the face's claim. Success cannot
redeem such refusal; it confirms it. ``Getting what one wants'' within this regime is punishment
as self-revelation: the act returns to the agent as accusation.

Levinas is explicit that the ethical relation is asymmetrical: I am responsible for the other
beyond reciprocity or contract \parencite[pp.~215--219]{LevinasTI1969}. This asymmetry is
precisely what proceduralism neutralizes, since procedures aim to distribute liability
symmetrically. Hence the peculiar chill of the film's most efficient moments: where a protocol
works, the asymmetry has been most thoroughly suppressed. The ethical demand has not been
answered; it has been absorbed—turned into a variable among others. Levinas's insistence that
the face is a ``poor one, a stranger'' \parencite[p.~213]{LevinasTI1969} gives content to the
felt wrongness of treating villagers, boat crews, and even soldiers as means for the continuity
of the project. The wrongness is not (only) that harm occurs; it is that the form of encounter
precluded responsibility before deciding what to do.

The assassination order against Kurtz does not escape this logic by turning against a tyrant.
Levinas's ``Thou shalt not kill'' is not a rule applied to friends but the structure of
encounter itself \parencite[p.~199]{LevinasTI1969}. To meet anyone—enemy included—first as a
bearer of exteriority is to be summoned to justification. There may be cases, Levinas allows,
where politics demands force; but politics is always under judgment by ethics
\parencite[pp.~21--24]{LevinasTI1969}. The film's denouement shows the inversion: politics
judges ethics, and efficiency is taken as justification. That is why the line's second half
sounds like a verdict: ``\ldots for my sins I got one'' acknowledges that the project's end
never included the first relation—the face's command—so its successful completion can only
declare that exclusion more clearly.

Levinas also explains why horror often clarifies rather than teaches in such worlds. Horror
strips away alibis and yet, without a conversion of the mode of approach, it cannot generate the
responsibility it reveals. The proper response is not a grand theory but a change in the grammar
of encounter—hospitality, attention, refusal of instrumentalization
\parencite[pp.~200--206]{LevinasTI1969}. In their absence, ``Everyone gets everything he wants''
remains the slogan of totality: the world is very good at supplying means. The punishment of
fulfillment is the renewed summons one cannot now un-hear.
