\phantomsection
\subsection*{Dostoevsky: Independent Desire, Anti-Mechanism, and Agency That Eats Itself}
\addcontentsline{toc}{subsection}{Dostoevsky: Independent Desire and Agency}
\label{ssec:iii-dostoevsky}
Fyodor Dostoevsky's \emph{Notes from Underground} is the classic anatomy of a will that prefers
independence to well-being—``man only wants independent desire, whatever that independence may
cost'' \parencite[p.~131]{DostoevskyNFU1994}. The underground man's most scandalous claim—``To
hell with two times two makes four!'' \parencite[p.~129]{DostoevskyNFU1994}—is not
anti-arithmetic; it is anti-mechanism in human affairs. He rejects any calculus in which
rational prediction, utility, or institutional procedure would close the space of spontaneous
willing. Read in this key, the line ``Everyone gets everything he wants'' marks not prosperity
but a world well-stocked with mechanisms that deliver objects on demand; ``for my sins I got
one'' acknowledges the price of willing agency itself within such machinery.

The underground man's revolt targets the dream that human conduct can be rendered scientific—that
motives can be predicted and optimized so that ``good'' outcomes follow from the right levers
\parencite[pp.~120--132]{DostoevskyNFU1994}. His point is not that people are irrational, but
that personhood includes a residual freedom that will ``assert itself'' against the system,
even destructively, simply to prove it exists \parencite[pp.~129--132]{DostoevskyNFU1994}.
This is why a ``crystal palace'' of perfect provisions would provoke sabotage; the human being,
he insists, will sometimes choose what is harmful to demonstrate authorship. The
\hyperref[scene:briefing]{dossier room's} hygienic proceduralism—clarity of ends, chain of
command, calibrated means—inhabits precisely the rational order the underground man distrusts.
Willard's wanting a mission is not a longing for certainty alone; it is the chance to act, to
break the inertia of \hyperref[scene:saigon-opening]{Saigon's aimlessness}, even if the action
risks moral injury. The ``sin''
is that the system can supply just such occasions and call the result necessary.

Dostoevsky's dialectic also explains the peculiar pleasure the underground man takes in acting
against his own interest: ``the most advantageous advantage'' is sometimes the freedom to choose
what is not advantageous \parencite[pp.~129--131]{DostoevskyNFU1994}. Agency is not measured by
outcomes but by the felt authorship of choice. That is why, on his account, a rational program
that secures only beneficial outcomes is degrading: it would reduce a man to a ``piano key'' on
which nature (or the institution) plays \parencite[pp.~115--120]{DostoevskyNFU1994}.
The \hyperref[scene:upriver-journey]{upriver insistence on continuing}, whatever the evidence or
cost, might \emph{seem} like a defense of non-instrumentality—``I choose to persist''—but
Dostoevsky's dialectic cuts both ways. If the persistence merely executes what the institution
already scripted, then the agent \emph{is} the piano key, not the player. Fulfillment punishes
because the moment the system grants the mission, the space of ``independent desire'' shrinks
into the execution of a mechanism that now owns the storyline.

Yet Dostoevsky's insight is double-edged. The underground man's independence is real, but it
corrodes itself when it refuses any measure beyond negation. He confesses to relishing
humiliation and spite, to savoring ``the sweetly painful pleasure'' of acting against himself
\parencite[pp.~108--115]{DostoevskyNFU1994}. This is agency that proves itself by injury.
When procedure yields horror (the \hyperref[scene:sampan]{sampan}), the rational mechanism has
plainly failed; but if
the next choice is animated only by the need to keep asserting agency—``I go on because I go
on''—the will ratifies the same emptiness the underground man inhabits. ``Everyone gets
everything he wants'' then describes a trap: the institution gets its obedient executor; the
agent gets the feeling of authorship; neither gets a norm by which the act could be vindicated.

Dostoevsky also anticipates what we might call moral theater: the staging of motives after the
fact to render destructive agency palatable. The underground man is merciless about his own
self-narration—confessing how quickly the ego invents edifying reasons for what was, in truth,
caprice or spite \parencite[pp.~103--107]{DostoevskyNFU1994}. This maps onto the rhetoric that
frames the assignment as sanitary necessity. The mask of moral purification (``remove an
aberration'') converts the hunger to act into an edifying plot; fulfillment then functions as
exposure when, at the end, the narrative yields no enlargement of soul. The confession ``for my
sins I got one'' reads, in Dostoevskian terms, as the recognition that the story was a postscript
to the will to act, not its ground.

A further Dostoevskian thread concerns responsibility under conditions of determinism. The
underground man refuses to let causal explanation excuse him: even if he can trace motives, he
will not permit the explanation to replace ownership \parencite[pp.~109--113]{DostoevskyNFU1994}.
This refusal illuminates the post-fulfillment chill: once the mission is complete, the agent
cannot hide in causal chains (``orders,'' ``procedure,'' ``necessity'') without committing the
very self-abdication he despises. Fulfillment punishes by removing alibis: the deed is done;
the authorship is mine.

Finally, Dostoevsky's anti-mechanism clarifies why the film's most efficient scenes are its most
disturbing. The underground man's nightmare is not chaos but perfect order—an order so seamless
it leaves no room for non-instrumental choice. Where everything ``works,'' the human residue can
only show itself by breaking the system or by converting obedience into a performance of will.
In either case, getting what one wanted discloses a deficit: agency defended merely as
independence becomes self-consuming. The sentence's halves therefore lock: the world can indeed
deliver the occasion to act (``everyone gets\ldots''), but the one who wanted agency itself
discovers, upon receiving it, that agency without a measure is indistinguishable from compulsion
in disguise—hence ``\ldots for my sins I got one.''
