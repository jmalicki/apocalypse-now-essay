\phantomsection
\subsection*{Schopenhauer: Fulfillment as Disclosure of Lack}
\addcontentsline{toc}{subsection}{Schopenhauer: Fulfillment as Disclosure of Lack}
\label{ssec:iii-schopenhauer}
Arthur Schopenhauer's analysis of will offers a rigorous grammar for Willard's confession:
``Everyone gets everything he wants. I wanted a mission, and for my sins they gave me one.''
For Schopenhauer, desire is not a teleology that culminates in peace but a mechanism whose
very satisfaction resets itself. Hence the mission granted is not a gift that stills the heart;
it is the next oscillation of a structure that cannot be stilled.

\phantomsection
\subsubsection*{1) The structure of willing: lack $\rightarrow$ striving
	$\rightarrow$ relief $\rightarrow$ renewed lack}
\addcontentsline{toc}{subsubsection}{The structure of willing}

``All willing springs from lack, from deficiency, and therefore from suffering''
\parencite[p.~196]{SchopenhauerWWR1969}. The object that seems to promise rest is already
implicated in the will's unease; when attained, it ``at once makes room for a new one''
\parencite[p.~319]{SchopenhauerWWR1969}. Schopenhauer's famed image is diagnostic rather
than rhetorical: life ``swings like a pendulum, to and fro between pain and boredom''
\parencite[p.~312]{SchopenhauerWWR1969}.

Read against the film's first movements, the pattern holds precisely. In
\hyperref[scene:saigon-opening]{Saigon}, Willard's lack is staged as agitation and
intoxication; the \hyperref[scene:briefing]{dossier scene} supplies an object (the mission)
and a narrative. Relief appears as orientation—a reason to move upriver—but immediately becomes
renewed lack: each \hyperref[scene:upriver-journey]{checkpoint} demands the next, each ``win''
opens a further deficit.
Willard's voiceover keeps the pendulum audible: completion never completes; it only
re-initiates striving.

\phantomsection
\subsubsection*{2) Why fulfillment punishes: the object is a delusion of rest}
\addcontentsline{toc}{subsubsection}{Why fulfillment punishes}

Schopenhauer emphasizes that satisfaction exposes, rather than heals, the structure of desire.
Enjoyment ``we have longed for'' soon leaves us ``bored,'' and the will ``returns to its old
course'' \parencite[p.~319]{SchopenhauerWWR1969}. In that sense, getting what one wanted hurts
because it removes the fantasy that the object could silence the will. The hurt is cognitive:
fulfillment disenchants.

The film thematizes this in miniature. The \hyperref[scene:playboy-show]{Playboy show}
promises heightened pleasure; the immediate after-image is agitation and bargaining.
\hyperref[scene:kilgore-beach]{Kilgore's beachhead} produces tactical success, but the
spectacle (``I love the smell of napalm in the morning'') converts victory into appetite. The
\hyperref[scene:sampan]{sampan search} yields compliance, then horror; the ``completed''
procedure reveals leanness of soul. The \hyperref[scene:french-plantation]{French plantation}
stages fulfillment frozen: the colonists achieved everything the will projected (land, dynasty,
permanence) and discover that possession does not pacify. They live among their buried dead,
clinging to what they won, trapped by their own achievement. The will got its object and
cannot let go even as fulfillment becomes living death. In Schopenhauer's terms, each
fulfilled want punctures its own promissory aura and re-installs the need to want.

\phantomsection
\subsubsection*{3) Representation, will, and the aesthetic \& ethical brakes that fail in war}
\addcontentsline{toc}{subsubsection}{Representation and brakes that fail}

Schopenhauer distinguishes the world as representation (ordered by the principle of sufficient
reason) from the world as will (blind striving) \parencite[pp.~3--5]{SchopenhauerWWR1969}.
Two brakes can mitigate the will's tyranny. First, aesthetic contemplation suspends willing by
fixing consciousness on the Idea—to ``lose oneself in the object''
\parencite[p.~178]{SchopenhauerWWR1969}. Second, compassion reframes the other not as instrument
but as a fellow bearer of suffering \parencite[pp.~372--374]{SchopenhauerWWR1969}.
Both brakes fail in the film's wartime economy. \hyperref[scene:kilgore-beach]{Wagner's
	``Ride of the Valkyries''} is mobilized as a stimulant for domination, not as will-suspending
contemplation; the sampan protocol subordinates pity to procedure. The very mechanisms that
could have cooled willing are conscripted by it.

\phantomsection
\subsubsection*{4) Boredom, repetition, and the river as pendulum}
\addcontentsline{toc}{subsubsection}{Boredom and repetition}

If pain signals unfulfilled desire, boredom signals desire's deflation after satisfaction.
Schopenhauer's claim that joy fades into ennui is not a counsel of mood but an analysis
of the will's metabolism \parencite[pp.~312--320]{SchopenhauerWWR1969}. The
\hyperref[scene:upriver-journey]{river sequences} enact this metabolism: periods of frantic
danger (pain) alternate with slack stretches of waiting (incipient boredom), and Willard's
narration re-ignites the need for the next trial.
The \hyperref[scene:do-lung-bridge]{Do Lung Bridge} sequence literalizes the pendulum:
building by day, destruction by night; every ``achievement'' immediately generates its
contrary.

\phantomsection
\subsubsection*{5) ``For my sins I got one'': the inner necessity of punishment}
\addcontentsline{toc}{subsubsection}{The inner necessity of punishment}

Schopenhauer does not require an external punisher. The punishment is inner: to ``get what one
wants'' is to have the will's insatiability revealed to oneself. Hence the tone of Willard's
clause; the ``sin'' is not only moral guilt but attachment to the fantasy that a mission could
deliver more than recurrence. The gift is therefore a judgment.

At \hyperref[scene:kurtz-compound]{Kurtz's compound}, this inner necessity is complete. Kurtz
has arranged the conditions for maximal satisfaction (command, removal of obstacles) and finds
that mastery produces only a
clarified view of the will's void: possession does not pacify. Willard's approach—labor, danger,
deprivation—does not redeem the will, but it strips away the last illusions about what
fulfillment can do. In Schopenhauer's lexicon, the world is disclosed as will precisely when
desire succeeds.

\phantomsection
\subsubsection*{6) Objection \& counterpoint: is there any release?}
\addcontentsline{toc}{subsubsection}{Is there any release?}

One might object that Schopenhauer allows two releases. First, aesthetic states offer
``deliverance'' from the will's press \parencite[p.~178]{SchopenhauerWWR1969}. Second,
compassion grounds ethics beyond egoistic striving \parencite[pp.~372--374]{SchopenhauerWWR1969}.
The film acknowledges both in negative: music becomes a tool for domination rather than
contemplation; pity is subordinated to protocol. The point is not that release is metaphysically
impossible, but that this narrative world is structured to block it; thus, fulfillment returns
as exposure.

\phantomsection
\subsubsection*{7) Payoff for the thesis}
\addcontentsline{toc}{subsubsection}{Payoff for the thesis}

Schopenhauer thus illuminates the first half of Willard's sentence: \emph{everyone} (because
willing is universal) \emph{gets} (because objects are available) \emph{everything} (because
the will projects ``the all'' onto finite objects) he \emph{wants} (because wanting, not the
wanted, is fundamental). The second half—``for my sins I got one''—expresses the cognate of this
metaphysics: punishment is not denial but grant that unmasks the will's structure. The mission
is not a deviation from desire's grammar; it is the very form in which that grammar is made
visible.

Willard's claim that afterward he ``never wanted another'' might seem to contradict
Schopenhauer's pendulum model, which predicts satisfaction immediately generates new lack. But
Schopenhauer allows one reading: the will can be stilled through total disillusionment. Not
aesthetic contemplation or compassion, but the exhaustive exposure of every object's
emptiness. If so, ``never wanted another'' is not liberation but the death of willing
itself---a grim confirmation of Schopenhauer's diagnosis that the will's only true peace is
its negation.
