\phantomsection
\subsection*{IV.10—Heidegger: Finitude, Anticipatory Resoluteness, and Why Completion Is
	Ontologically Out of Reach}
\addcontentsline{toc}{subsection}{IV.10—Heidegger: Finitude and Resoluteness}
\label{ssec:iii-heidegger}
Martin Heidegger's account of existence (Dasein) makes completion a category mistake. Dasein is
essentially being-possible—a projecting that never coincides with itself as a finished thing;
its wholeness is disclosed only in being-toward-death \parencite[pp.~279--311]{HeideggerBT1962}.
Death is not (primarily) a future event to be scheduled but the ownmost, nonrelational
possibility that individualizes Dasein now, stripping away the illusions of totalization
\parencite[pp.~294--307]{HeideggerBT1962}. Thus, any project that promises narrative wholeness—that
a mission will ``make it come together''—misreads existence. When Willard says ``Everyone gets
everything he wants,'' the Heideggerian gloss is brutal: the world may indeed supply objects and
tasks, but existence is not something an object can finish. ``\ldots And for my sins I got one''
is the moment the project's alleged telos collides with finitude.

Heidegger's analysis of everydayness and the They (\emph{das Man}) clarifies why projects so
easily wear the mask of necessity. In average everydayness, Dasein takes over possibilities ``as
one does,'' letting anonymous norms dictate what counts as urgent, clean, or right
\parencite[pp.~149--168]{HeideggerBT1962}. The \hyperref[scene:briefing]{Saigon briefing}'s
procedural tone—dossiers, signatures, the grammar of sanitation—exemplifies this absorption in
\emph{das Man}: the mission
shows up as what ``one'' does when a file reads anomalous. To take it up as such is not yet
resolute choice; it is fallenness into the ready-made interpretation. Fulfillment then
``punishes'' by disclosing that the accomplished sequence was never a route to owned wholeness;
it was a they-self rhythm all along.

The film's temporality maps onto Heidegger's account of ecstatic time. Dasein's temporality is
not a string of nows but an ``ahead-of-itself'' (future), already-in (past), and being-alongside
(present) \parencite[pp.~373--383]{HeideggerBT1962}. The Do Lung Bridge cycle—construction by
day, destruction by night—stages a caricature of inauthentic time: a serial present that never
gathers. Anticipatory resoluteness does not end such cycles; it interprets them soberly by
owning death as the limit that prevents totalization \parencite[pp.~307--311]{HeideggerBT1962}.
By this light, the climactic ``success'' cannot heal the fracture; it can only remove the alibi
that failure once provided. The felt judgment of the line is that clarity: the project is
complete and therefore unable to hide the truth that existence cannot be.

Heidegger's conscience and guilt intensify the point. Conscience ``calls'' Dasein from \emph{das
	Man} to its ownmost possibility; guilt (\emph{Schuld}) names not juridical fault but
being-the-basis of a nullity—that our thrown projection always leaves something out and cannot
guarantee innocence \parencite[\S\S 57--60, pp.~311--354]{HeideggerBT1962}. When procedures run
perfectly (the sampan inspection as ``by the book'') and still yield devastation, what is
revealed is not only moral failure but ontological mismatch: the attempt to secure existential
rightness via technical closure. Anticipatory resoluteness would require owning that mismatch,
not masking it with narratives of cleansing. The confession—``for my sins I got one''—is a
resolute sentence in this sense: it drops the promise of narrative wholeness and accepts
finitude as the horizon that renders completion impossible.
