\phantomsection
\subsection*{IV.13—Comparative Discussion: Convergences and Tensions}
\addcontentsline{toc}{subsection}{IV.13—Comparative Discussion}
\label{ssec:iv-comparative-discussion}

The preceding twelve analyses reveal not a single philosophical account of Willard's line but
a field of competing and complementary diagnoses. Some tensions are productive: they refine
the verdict by forcing precision about what kind of failure fulfillment exposes. Other
convergences are striking: across metaphysical, existential, and ethical vocabularies, the
philosophers agree that ``getting what one wants'' punishes because it reveals the will's
prior orientation or structure. What follows maps the key debates and their implications for
reading the film.

\subsubsection*{Is Recurrence Curse or Opportunity? Schopenhauer, Nietzsche, Camus}

Schopenhauer's phenomenology is precise: satisfaction ``at once makes room for a new one,'' so
life swings ``between pain and boredom'' \parencite[pp.~312, 319]{SchopenhauerWWR1969}. The
\hyperref[scene:upriver-journey]{upriver sequence} confirms this---each checkpoint delivers
relief that immediately becomes renewed lack. Nietzsche objects that such pessimism
misconstrues the task: recurrence is not a curse if the will has the courage to revalue
itself, to create new measures rather than repeat
old consumption \parencite[\S\S 34, 283]{NietzscheBGE1990}. What corrodes willing is not its
repetition but its dishonesty---domination disguised as truth.

The film tests both claims. Nietzsche is right that the mission wears the mask of cognition
(dossiers, rationality, surgical necessity), and that mask licenses command. Yet the narrative
never transvalues. After the \hyperref[scene:sampan]{sampan}, after the
\hyperref[scene:do-lung-bridge]{bridge}, no new measure emerges. Schopenhauer's
pendulum reasserts itself: fulfillment disenchants, and the will swings back to lack. Camus
cuts between them with lucidity: even a creative will must live ``without appeal,'' and no
final sanction redeems completion \parencite[pp.~28, 54, 121--123]{CamusMyth1991}. The film
sides with Camus---the world delivers objects, but what returns is knowledge, not meaning. The
convergence is grim: whether the problem is metaphysical mechanism (Schopenhauer), dishonest
transvaluation (Nietzsche), or absurdist silence (Camus), fulfillment cannot heal the will.

\subsubsection*{Does Success Ever Vindicate? Kant, Nietzsche, Kierkegaard}

Kant denies that outcomes certify worth: the good will is ``good \ldots\ in itself,'' not
``because of what it effects'' \parencite[p.~27]{KantGroundwork1996}. The
\hyperref[scene:sampan]{sampan inspection}, executed flawlessly, still fails the humanity
constraint---persons treated merely as means cannot be rescued by efficient procedure
\parencite[pp.~36--37]{KantCPrR1996}. Nietzsche
counters that Kantian morality can itself be a will to command in disguise: the ``desire for
`truth''' becomes a tool of domination when it pretends neutrality
\parencite[\S 34]{NietzscheBGE1990}.

Both critiques land. The briefing room stages Nietzsche's suspicion---rational necessity as
rhetorical cover for institutional will. Yet Kant supplies the verdict that still condemns:
even if we unmask the rhetoric, the maxim (eliminate persons designated as obstacles) fails
universalizability. Kierkegaard adds an internal dimension: even if the maxim somehow passed,
absolutizing a finite project as the self's ground thickens despair
\parencite[pp.~69--83]{KierkegaardSUD1980}. The triple pressure is severe: success cannot
vindicate (Kant), unmasking cannot excuse (Nietzsche), and structural rightness cannot cure
misrelation (Kierkegaard). ``For my sins I got one'' thus reads as the removal of every alibi.

\subsubsection*{Freedom as Burden or Condemnation? Sartre, Beauvoir, Dostoevsky}

Sartre's radical claim---we are ``condemned to be free''
\parencite[pp.~34--36]{SartreBN2003}---makes every project an authorship with absolute
responsibility. The mission cannot be blamed on
orders or necessity; it is freely chosen and owned. Beauvoir specifies the ethical constraint:
freedom is authentic only when it wills the freedom of others \parencite[p.~73]{Beauvoir1976}.
A project that systematically instrumentalizes contradicts freedom's structure. Together, they
condemn the mission on two grounds: it is bad faith (Sartre) and it violates reciprocity
(Beauvoir).

Dostoevsky complicates this by insisting that agency itself can be the disease. The
Underground Man wants ``independent desire, whatever that independence may cost''
\parencite[p.~131]{DostoevskyNFU1994}---he would rather act destructively than be a ``piano
key'' in a rational system. The mission-form threatens precisely this: absorption into
mechanism. Yet agency defended as pure negation (``I go on because I refuse the system'')
corrodes itself into spite. The film stages both traps: obedience performed as authorship
(bad faith) and continuation without measure (self-consuming agency). Fulfillment exposes that
neither path preserves genuine freedom.

\subsubsection*{Completion as Ontological Error: Sartre, Heidegger}

Sartre and Heidegger converge that completion is impossible, but their reasons differ. For
Sartre, the \emph{pour-soi} is perpetual transcendence; it ``is what it is not and not what
it is'' \parencite[pp.~100--110]{SartreBN2003}. The hidden ``project to be God''---to fuse
facticity and transcendence into self-grounding plenitude---cannot succeed
\parencite[pp.~586--604]{SartreBN2003}. Heidegger roots the error differently: Dasein's
wholeness is disclosed only in being-toward-death, which individualizes now and strips
totalization-fantasies \parencite[pp.~294--307]{HeideggerBT1962}. Where Sartre diagnoses a
wish for ontological closure, Heidegger diagnoses fallenness into \emph{das Man}---the fantasy
that doing ``what one does'' could yield authentic wholeness
\parencite[pp.~149--168]{HeideggerBT1962}.

The difference matters for interpretation. Sartre reads the mission's end as exposing bad
faith; Heidegger reads it as removing the alibi of average everydayness. Both see the vacuum
after clean procedures, but Sartre emphasizes free choice's responsibility, while Heidegger
emphasizes the they-self's inauthenticity. The film allows both: Willard freely chose the
project (Sartre) and absorbed it as ``what one does'' (Heidegger). Fulfillment punishes both
ways.

\subsubsection*{Reciprocity or the Face's Command? Beauvoir, Levinas}

Beauvoir and Levinas both condemn instrumental projects, but from different starting points.
Beauvoir builds the other into freedom's structure: authentic willing must will the other's
freedom \parencite[p.~73]{Beauvoir1976}. Projects are justified when they found situations
where others can transcend \parencite[pp.~145--153]{Beauvoir1976}. Levinas argues this comes
too late: the face's prohibition (``Thou shalt not kill'') precedes all projects and resists
assimilation into reciprocal frameworks \parencite[pp.~199, 21--24]{LevinasTI1969}. Ethics is
asymmetrical---I am responsible for the other beyond contract.

The tension is productive. Beauvoir's framework can critique the mission's failure to build a
common world; Levinas can indict the very mode of approach (dossier, protocol) as a refusal of
the face. Beauvoir worries Levinas risks ethical purity without political efficacy; Levinas
warns Beauvoir's world-building easily re-totalizes. The film confirms both critiques:
procedures flatten alterity (Levinas) and successes never found shared institutions
(Beauvoir). ``Everyone gets everything he wants'' names efficiency without co-agency; ``for my
sins I got one'' marks the double failure.

\subsubsection*{Objects or Recognition? Hegel, Kojève}

Hegel's master-slave dialectic reveals why possessing things cannot satisfy:
``self-consciousness achieves its satisfaction only in another self-consciousness''
\parencite[\S 175]{HegelPhenomenology1977}. The Master gets obedience but finds it empty---coerced
submission is not free recognition \parencite[\S\S 187--189]{HegelPhenomenology1977}. Truth
lies with work that transforms the world into a space of mutual acknowledgment
\parencite[\S 196]{HegelPhenomenology1977}. Koj{\`e}ve radicalizes this: human desire is
``desire of another's desire'' \parencite[p.~6]{KojeveIRH1980}---we want to be desired/recognized,
not merely to possess.

Applied to the film, this explains the hollowness of each ``win.'' Willard secures objectives,
but objectives do not recognize him. The currency is wrong: dominion over things and
submissions cannot purchase what human desire seeks (free acknowledgment from an equal). The
mission-form systematically forecloses recognition by reducing others to obstacles or
instruments. Hence ``getting what one wants'' delivers everything except what was unconsciously
sought. Hegel and Koj{\`e}ve converge with Levinas's asymmetry and Beauvoir's reciprocity from
a different angle: all four insist the other's freedom cannot be bracketed without voiding
satisfaction.

\subsubsection*{Conditions for Fulfillment That Satisfies}

The preceding analyses diagnose how fulfillment punishes. What would allow it to satisfy
instead? The debates converge on systematic requirements that the film's mission violates at
every turn:

\textbf{(1) Anti-instrumentality (Kant, Beauvoir, Levinas):} Treat persons as ends, will
others' freedom, respect the face's prohibition.

\textbf{(2) Anti-bad-faith (Sartre, Heidegger):} Own the project as freely chosen, not as
``what one does.''

\textbf{(3) Anti-totalization (Hegel, Kojève):} Seek recognition through work that founds a
common world, not mastery that silences.

\textbf{(4) Anti-absolutization (Kierkegaard, Dostoevsky):} Do not stake the self's ground on
a finite project; preserve agency's measure.

\textbf{(5) Anti-stasis (Nietzsche, Camus):} Revalue maxims when outcomes strip rhetoric; live
without appeal to final vindication.

The mission fails every test. It treats persons as means, disguises choice as necessity, seeks
dominion not recognition, absolutizes a finite end, and never transvalues. That it
``succeeds'' procedurally is precisely why it punishes existentially. The line's two halves
lock: the world delivers what systems can deliver; the will discovers that delivery was not
what it needed.

\subsubsection*{Implications for the Film's Moral Thesis}

These philosophical debates constrain what Willard's line can mean. The film's structure---a
journey where every success generates new lack, where efficient means yield moral emptiness,
where completion does not redeem---confirms the convergent diagnosis across the traditions.
Whether the vocabulary is Schopenhauer's pendulum, Sartre's bad faith, Kant's heteronomy,
Levinas's totality, or Hegel's empty mastery, the pattern holds: getting what one wants
exposes the wanting's misdirection.

Yet the philosophers also preserve hope, if severely qualified. Kant's duty, Beauvoir's
reciprocity, Levinas's asymmetrical responsibility, Nietzsche's transvaluation, Camus's
lucidity without appeal---each offers a discipline for willing differently. The film refuses
this path. Willard's final silence is not conversion but paralysis. He has seen the mirror but
cannot alter what it shows. The essay thus reads the film as tragedy in the philosophers'
sense: not the defeat of a good will by external forces, but the exposure of a will whose very
structure ensured that fulfillment would punish. The line is not wisdom but epitaph.
