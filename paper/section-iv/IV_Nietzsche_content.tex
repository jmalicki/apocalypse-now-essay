\phantomsection
\subsection*{Nietzsche: Fulfillment as Style of Will—Affirmation, Domination, Transvaluation}
\addcontentsline{toc}{subsection}{Nietzsche: Fulfillment as Style of Will}
\label{ssec:iii-nietzsche}
Friedrich Nietzsche does not dispute Schopenhauer's observation that willing does not rest; he
revalues it. The problem is not desire's recurrence but our craving for a terminal perch that
would end
the need to will. In this light, ``Everyone gets everything he wants'' becomes a diagnostic:
fulfillment reveals whether the will has the style to affirm its own recurrence, or whether it
smuggles domination in under the names of truth and duty. ``For my sins I got one'' is the
moment the mask of those names slips.

Nietzsche's stark claim—``man would rather will nothingness than not will at all''—positions
the refusal of willing as more intolerable to life than suffering
\parencite[III.28, p.~162]{NietzscheGenealogy1994}. The danger, then, is not the intensity of
volition but its self-deception: the will valorizes itself as ``truth'' so it can command
without admitting it. ``The desire for `truth' has hitherto been the most dangerous of all
possessions'' because it disguises a need to impose \parencite[\S 34]{NietzscheBGE1990}.
The Saigon briefing's cool rationality—dossiers, maps, a narrative of ``removing an
aberration''—is exemplary of this danger: a project of command is presented as neutral cognition.
When Willard ``wants a mission,'' the wanting is not epistemic; it is a pledge of will stamped
with the authority of ``truth.'' The sentence's first clause (``Everyone gets everything he
wants'') thus records not luck but the world's capacity to deliver the objects that our
valuations already framed as necessary; the second clause (``for my sins\ldots'') signals
the after-knowledge that those valuations were life-denying.

Against such self-deception, Nietzsche sets style—the capacity to shape one's evaluations when
reality exposes them as evasions. He urges a ``revaluation of all values''
\parencite[\S\S 203--211]{NietzscheBGE1990}, and the call to ``live dangerously!''
\parencite[\S 283]{NietzscheBGE1990} names a refusal of anesthetized security rather than a
cult of risk. In this register, fulfillment is not possession of the object but self-formation:
the will confirms itself by changing its own measure. The
\hyperref[scene:upriver-journey]{upriver progression} continuously offers occasions for such
revaluation—each checkpoint turning success into a new claim on the self. When compliance with
procedure at the \hyperref[scene:sampan]{sampan} yields horror, a Nietzschean response would be
to transvalue the maxim that licensed it. Instead, the will prefers continuity of command;
it ``gets what it wants'' (control, clarity) and is punished by the disclosure that its wanting
is reactive—obedience to inherited values that present themselves as necessity.

Nietzsche's psychology of ressentiment further clarifies the moralizing energies that travel
with domination. The weak, unable to act, invert impotence into virtue by calling the strong
``evil'' and themselves ``good'' for not doing what they cannot do
\parencite[I.10--14]{NietzscheGenealogy1994}. Yet he also describes a noble pathos that wants
to expand and test itself \parencite[\S\S 260--265]{NietzscheBGE1990}. In the film's middle
movements, these vectors cross: theatrical sovereignty stages itself as exuberance
(\hyperref[scene:kilgore-beach]{``I love the smell of napalm in the morning''}), while the
bureaucratic ``we'' that dispatches the assassin wraps elimination in the moral language of
purification. Both are forms of wanting
that the sentence anatomizes: one wants spectacle of command, the other wants justification for
command—but neither shows the transvaluative courage to alter its measure when outcomes strip
the rhetoric bare.

Kurtz, often read as the one who has gone ``beyond good and evil,'' in fact illustrates
Nietzsche's worry about the will's last refuge: after unseating inherited norms, it longs for
a final verdict that would secure mastery once more. Nietzsche's ``beyond good and evil'' is
not a license for cruelty; it is lucidity about the genealogy of one's values and the refusal
to enthrone a new absolute \parencite[\S\S 259--260]{NietzscheBGE1990}. Kurtz's pronouncement
of ``the horror'' behaves like that new absolute—a metaphysical seal on judgment that would
still the will's vulnerability. If he has ``got what he wants'' (freedom to rule, pronounce,
and be obeyed), fulfillment punishes by revealing the emptiness of mastery that will not
relinquish its last metaphysical crutch.

The health-criterion, for Nietzsche, is severe and simple: does this willing increase one's
capacity to affirm life—including ambiguity and pain—or does it shrink that capacity under a
rhetoric of truth and duty? A project may be perfectly ``true'' by institutional measures and
yet sick by this criterion. When the assignment is executed to the letter, and nothing
redemptive follows—no enlargement of perspective, no transvaluation of maxims—the confession
(``\ldots for my sins I got one'') reads as recognition that fulfillment has exposed the willing
as life-denying. In Nietzsche's terms, the punishment is not the mission's cost but its clarity:
getting what one wanted shows which kind of will one is.
