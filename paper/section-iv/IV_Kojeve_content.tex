\phantomsection
\subsection*{Koj{\`e}ve: Desire of Desire, History as Recognition, and Why Mission-Form
	Fulfillment Recurs as Lack}
\addcontentsline{toc}{subsection}{Kojève: Desire of Desire and Recognition}
\label{ssec:iii-kojeve}
Alexandre Koj{\`e}ve radicalizes Hegel's insight in anthropological terms: human desire is
``desire of another's desire''—a need to be desired/recognized by a free other
\parencite[p.~6]{KojeveIRH1980}.
The object mediates this relation, but it is not the final aim. Hence the lordship/bondage
dialectic reads, in Koj{\`e}ve's gloss, as the matrix of history: the Master obtains things
(and obedience) but not the recognition that would satisfy a human desire; the Slave, through
fearful work, transforms the world and, in so doing, becomes the bearer of truth
\parencite[pp.~27--34, 158--164]{KojeveIRH1980}. If the film's world can reliably deliver
missions and outcomes—``everyone gets everything he wants''—what it cannot deliver, by those
same means, is the desire of the other freely given.

Koj{\`e}ve's portrait of the Master maps neatly onto the mission-form that prizes clean
execution and visible effects. The Master ``gets what he wants,'' but his world is populated by
things and submissions, not by interlocutors who can confirm him
\parencite[pp.~27--34]{KojeveIRH1980}. In such a regime, success increases dependence on further
success, because each attainment fails to supply the missing confirmation. The appetite becomes
serial: new targets, new proofs, new shows of power. This is the historical engine that drives
the ``pendulum'' of operations: each completion—however perfect—returns as renewed lack, not
because the agent is psychologically thin, but because the form of fulfillment excludes the kind
of acknowledgment that could end the sequence.

By contrast, Koj{\`e}ve sees the Slave's work as the slow route to recognitive stability: work
shapes a common world in which self and other can appear to each other as free
\parencite[pp.~158--164]{KojeveIRH1980}. That is why he can speak of the ``end of history'' as
a horizon of universal recognition, not maximal accumulation
\parencite[pp.~158--164]{KojeveIRH1980}. Measured against this horizon, the
\hyperref[scene:upriver-journey]{upriver procedures} create no institutions of mutual address;
they routinize asymmetry. Even the \hyperref[scene:assassination]{final act}—eliminating
the figure who has refused the institution—seeks restoration of order without creating the space
in which recognition could be mutual. Fulfillment thus returns as judgment: the very evidence of
technical success is the evidence that the recognitive aim was never in view.

Koj{\`e}ve's reading also explains the peculiar tone of mastery's self-knowledge. Once the mask
of ``truth'' and ``necessity'' falls, the Master must either convert—accept that what he wanted
cannot be had by command—or double down, seeking ever more unchallengeable evidence of
sovereignty. The line ``\ldots for my sins I got one'' registers the first path as insight
without conversion: one sees that the delivery of ends cannot deliver recognition, yet one has
already acted in the Master's grammar. The punishment is temporal: the completed project does
not close history; it lengthens the sequence of unsatisfying confirmations.

In Koj{\`e}ve's terms, then, the film's world is historically stuck between mastery's emptiness
and the slow, dangerous labor that could found a recognitive order. ``Everyone gets everything
he wants'' names a high-functioning apparatus for producing things and effects; ``\ldots and for
my sins I got one'' names the self-knowledge that, within that apparatus, the human desire—desire
for the other's free acknowledgment—was never addressed. What returns is not failure but the
truth about what was really wanted.
