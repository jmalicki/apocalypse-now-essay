\phantomsection
\subsection*{Sartre: Freedom as Condemnation, the Impossible Synthesis, and Fulfillment as
	Exposure}
\addcontentsline{toc}{subsection}{Sartre: Freedom as Condemnation}
\label{ssec:iii-sartre}
Jean-Paul Sartre's ontology makes ``Everyone gets everything he wants'' a trap built into
freedom. For him, human reality (\emph{pour-soi}) is a lack that projects itself toward being;
it is ``what
it is not and not what it is,'' a perpetual surpassing of itself
\parencite[pp.~100--110]{SartreBN2003}. Desire therefore aims, at bottom, at an ontological
closure it can never attain. The will does not simply seek objects; it seeks to abolish its
lack by becoming a settled being. That is the hidden horizon against which the mission takes on
its peculiar glow. ``For my sins I got one'' names the moment the project's promised closure
reveals itself as structurally impossible.

\phantomsection
\subsubsection*{Condemned to Be Free: Pour-soi and the Project}
\addcontentsline{toc}{subsubsection}{Condemned to Be Free}

The project-form is central to Sartre's account of freedom. Freedom is not a privilege but the
very structure of consciousness: we are ``condemned to be free,'' without essence to excuse or
guarantee our choices \parencite[pp.~34--36]{SartreBN2003}. Because the \emph{pour-soi} is
nothing but transcendence beyond the given (facticity), every life is a project—a coherent
orientation that confers meaning retroactively on its acts
\parencite[pp.~561--569]{SartreBN2003}. The Saigon acceptance scene reads here as the decisive
orientation of a freedom in flight from its drift: a project chosen to still contingency by
giving it a vector. But, in Sartre's grammar, such orientation never stills the source; it
intensifies responsibility. Once the mission is chosen, there are no alibis left.

\phantomsection
\subsubsection*{The Project to Be God: Impossible Ontological Synthesis}
\addcontentsline{toc}{subsubsection}{The Project to Be God}

Beneath every finite project, Sartre identifies a secret, universal temptation—the ``project to
be God'': to fuse our throwness (facticity) and our transcendence into a single, self-grounding
plenitude \parencite[pp.~586--604]{SartreBN2003}. That synthesis is impossible. The
\emph{pour-soi} can never coincide with itself as the \emph{en-soi} does; it can only nihilate
the given and project beyond it. When a mission is taken as the end that would reconcile what
we are (situated, limited) with what we intend (sovereign authorship), fulfillment must punish
because its very success exposes the misconceived telos: the project could not, even in
principle, provide what the will implicitly asked of it—ontological peace.

\phantomsection
\subsubsection*{Bad Faith, the Look, and Absolute Responsibility}
\addcontentsline{toc}{subsubsection}{Bad Faith and Responsibility}

Sartre's analysis of bad faith illuminates how institutional roles mask this impossibility.
``Bad faith'' names the flight from freedom by posing oneself as either pure thing (just
obeying orders) or pure transcendence (unconditioned author), disowning the inseparable unity
of both \parencite[pp.~86--116]{SartreBN2003}. The procedural rhetoric of dossiers, signatures,
and necessity tempts the agent to occupy the role of function—a thing among things—while
narrating himself as a lucid executor. In fact, the act is freely chosen under a maxim that the
agent owns. The \hyperref[scene:french-plantation]{French plantation} stages bad faith as
frozen ritual: the colonists maintain elaborate forms (dinners, refinement, cultural identity)
despite the empire's collapse. They achieved their project (land, dynasty) and now face that
nothing in the world compelled it or justifies keeping it, yet they refuse this recognition by
clinging to the role. The forms remain, but the alibi (imperial mandate, civilization's burden)
has evaporated. Fulfillment punishes because once the project is complete, the alibi of role
collapses: nothing in the world compelled this project as mine. The confession (``\ldots for my
sins I got one'') reads as a crack in bad faith: a recognition that the necessity was staged.

The problem deepens when we consider Sartre's account of the Look (\emph{le regard}), which
shows how others reveal our facticity while tempting us to convert them into means for our
project \parencite[pp.~252--302]{SartreBN2003}. A mission-form that objectifies faces as
obstacles or instruments produces a world of being-for-others devoid of reciprocity. Each
``efficient'' success therefore deepens alienation: it multiplies acts in which the other's
freedom is suppressed to maintain the project's clarity. The more cleanly the procedure runs,
the more legible the structure becomes: meaning has been outsourced to instrumental success,
not grounded in a shared world. Fulfillment is thus a revelation: what we wanted was not meaning
but the uninterrupted sovereignty of a plan.

Sartre's relentless claim is that responsibility remains absolute. Causal explanation never
cancels authorship. Situations ``are what they are,'' but they are what they are for a freedom
that chooses what to make of them \parencite[pp.~553--561]{SartreBN2003}. This is why the end
of a project often feels accusatory. When nothing redemptive follows a technically perfect
execution, the agent confronts the naked fact that the project's value was not in the world but
in the choice that sustained it. ``Everyone gets everything he wants'' then means: the world is
reliable at delivering objects for our projects; ``\ldots and for my sins I got one'' means:
once delivered, the project reflects my choice back at me without the cushion of failure.

One might object, Sartreanly, that lucid perseverance—owning the act without appeal—could
transfigure the mission into authenticity. But for Sartre authenticity is not stubbornness; it
is lucidity about the impossibility of completion and refusal of bad faith in either direction
(no hiding in role; no fantasy of omnipotent authorship). In a world where the project's
structure systematically instrumentalizes others, lucidity would require altering the project or
abandoning it, not merely executing it honestly. Where no such alteration occurs, fulfillment
cannot be redemptive: it is an X-ray of the willing that carried it.

Sartre thus underwrites both halves of Willard's sentence. ``Everyone gets everything he
wants'': the world provides ample situations in which freedom can adopt ends and see them
through. ``For my sins I got one'': because the end was implicitly a bid for the impossible
synthesis (completion, immunity from ambiguity), getting it reveals the project as bad-faith
flight from freedom's structure. Punishment is not failure but clarity—the clarity that
completion was never on offer.
