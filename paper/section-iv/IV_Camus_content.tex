\phantomsection
\subsection*{Camus: Absurd Lucidity, Revolt ``Without Appeal,'' and Completion as Knowledge
	(Not Meaning)}
\addcontentsline{toc}{subsection}{Camus: Absurd Lucidity and Revolt}
\label{ssec:iii-camus}
Albert Camus begins with a refusal of consolations. ``There is but one truly serious philosophical
problem, and that is suicide'' \parencite[p.~3]{CamusMyth1991}. The claim sets the tone: the
question is not whether life can be made coherent, but whether one can live honestly when it
cannot. The absurd is the name for this standoff—``born of the confrontation between the human
need and the unreasonable silence of the world'' \parencite[p.~28]{CamusMyth1991}. Read against
Willard's line, ``Everyone gets everything he wants,'' the absurd warns that the delivery of
objects and outcomes has no built-in power to answer the need that generated them. ``\ldots And
for my sins I got one'' sounds, in Camus's vocabulary, like an onset of lucidity: completion
gives knowledge, not meaning.

Camus is suspicious of what he calls philosophical suicide—any leap (religious, metaphysical,
or ideological) that smuggles meaning back in after the absurd has been recognized
\parencite[pp.~53--58]{CamusMyth1991}. To live ``without appeal'' is to refuse that leap
\parencite[p.~54]{CamusMyth1991}. Much of the film's rhetoric—dossier certainties, the mission's
hygienic narrative—functions as an appeal to an order that would dissolve ambiguity. As the
journey upriver strips those narratives away, the world retains its ``unreasonable silence,''
yet the project continues. Camus would say: the willing, deprived of its fictions, is now
exposed to the task of revolt—not overthrow, but a ``permanent confrontation'' with
meaninglessness \parencite[p.~55]{CamusMyth1991}. If the revolt does not transvalue its maxims,
completion will punish by revealing that the act was, after all, an appeal in disguise.

Camus reframes fulfillment by recoding value as lucidity, freedom, passion—the three modalities
of living the absurd \parencite[pp.~54--71]{CamusMyth1991}. Lucidity means remaining with what
the world actually grants; freedom means recognizing that, if meanings are not given, our
projects are ours without metaphysical guarantees; passion means intensifying experience rather
than seeking a terminal sanction. In this light, the film's most efficient scenes (where things
work) are its least meaningful. The Do Lung Bridge cycle—building by day, destruction by
night—reads like the myth of Sisyphus in military dress: strenuous labor without appeal, the
task's perfection indifferent to significance. Camus's verdict on Sisyphus—``One must imagine
Sisyphus happy'' \parencite[p.~123]{CamusMyth1991}—does not romanticize toil; it claims that
honesty about the task's finitude can be a site of dignity. The catch is that such dignity
requires abandoning the promise that the task will redeem. Where Willard's project is still
mortgaged to a redemptive story (purge the aberration, restore sense), its successful completion
must recoil as knowledge that no redemption follows.

The figure most tempted by appeal is the one who seeks a final verdict on existence. Camus's
polemic targets precisely that longing: the desire to seal the world with an ultimate judgment
that would still the need to will \parencite[pp.~53--60]{CamusMyth1991}. In this register,
Kurtz's pronouncement—``the horror''—behaves like a metaphysical seal, an ultimate word that
would turn lucidity into law. Camus would demur: the absurd forbids the last word. To honor the
absurd is to continue acting without that word—no appeal to a transcendent rule, no enthronement
of the self as tribunal. ``Everyone gets everything he wants'' thus becomes, for Camus, a
litmus: if what one wanted was an ultimate exoneration, getting it will read as punishment—the
world remains silent.

Camus's portrait of the absurd hero helps explain the tone of the confession. The absurd hero
does not seek to solve the absurd; he keeps faith with it through measured revolt
\parencite[pp.~54--60, 121--123]{CamusMyth1991}. He does not deny limits, and he does not
pretend his acts are guaranteed meaning by a higher court. Where the mission-form equates
success with justification, Camus severs that link. After the deed, what remains is clarity:
we know what the world is (silent), what we are (beings who will without guarantee), and what
action can be (finite, accountable, unredeemed). If the project has been conducted under the
illusion that completion = meaning, then completion unveils the illusion. ``For my sins I got
one'' is exactly that unveiling—the point at which the will, faced with the absurd, loses its
alibi.

Finally, Camus's injunction—live ``without appeal''—tightens the essay's thesis. If the world
can reliably supply objects for our projects (hence ``everyone gets\ldots''), and if those
projects often carry tacit appeals (to necessity, to cleansing narratives, to final judgments),
then the punitive feel of fulfillment is simply the return of the real: the object arrives; the
appeal fails; lucidity remains. The task, if there is one, is to convert willing from a demand
for consummation into a discipline of revolt—a way of acting that neither lies about meaning
nor abdicates it. Absent that conversion, getting what one wants will continue to accuse the
will that wanted it.
